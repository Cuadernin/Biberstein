
\documentclass[12pt, letterpaper, twoside]{book}
\usepackage{amsfonts, amsmath, amssymb, color, fancyhdr, graphicx, latexsym, makeidx, mathrsfs, multicol, textcomp, theorem}

\usepackage{XTOCInc}

\usepackage[spanish]{babel}

\usepackage{ulem}
\usepackage[shortlabels]{enumitem}
\usepackage{enumitem}
\usepackage{stackrel}


%%%%%%%%%%%%%% begin fonts %%%%%%%%%%%%%%

\usepackage{ccfonts,eulervm}
\usepackage[T1]{fontenc}

%%%%%%%%%%%%%% end fonts %%%%%%%%%%%%%%

\DeclareFontFamily{\encodingdefault}{default}{\hyphenchar\font=`\-}

\DeclareMathOperator\sen{sen}
\DeclareMathOperator\arctanh{\tt arc\,tanh}
\DeclareMathOperator\Arctan{\tt Arc\,tan}
\DeclareMathOperator\Arccos{\tt Arc\,cos}
\DeclareMathOperator\expo{\tt exp}
\DeclareMathOperator\Homo{\tt Hom}

\DeclareMathOperator\Det{\mathop{\hbox{\tt det}}}
\DeclareMathOperator\Vol{\mathop{\hbox{\tt vol}}}
\DeclareMathOperator\Diam{\mathop{\hbox{\tt di\'{a}m}}}
\DeclareMathOperator\Diag{\mathop{\hbox{\tt diag}}}
\DeclareMathOperator\Sgn{\mathop{\hbox{\tt sgn}}}

\DeclareMathOperator*{\miCup}{\cup}

\def\milimSup{\mathop{\hbox{{\tt l\I m\,Sup}}}}
\def\miliminf{\mathop{\hbox{{\tt l\I m\,\I nf}}}}

\def\miMax{\mathop{\hbox{{\tt M\'ax}}}}
\def\mimin{\mathop{\hbox{{\tt m\I n}}}}

\def\miSup{\mathop{\hbox{{\tt Sup}}}}
\def\miinf{\mathop{\hbox{{\tt \I nf}}}}

\def\milim{\mathop{\hbox{{\tt l\I m}}}}

\def\miSop{\mathop{\hbox{{\tt Spt}}}}

%\font\seventeentt=cmr10 at 17pt

%\theoremheaderfont{\scshape}

\hyphenation{sobre}
\hyphenation{normado}
\hyphenation{Muestre}
\hyphenation{tales}
\hyphenation{bola}
\hyphenation{conjunto}
\hyphenation{Existe}

%\def\bedis{$\begin{displaystyle}}

%\def\endis{\end{displaystyle}$}

\def\bedis{$\displaystyle }

\def\endis{$ }



\def\then{\ \Longrightarrow \ }

\newcommand{\F}{\ensuremath{\mathbb{F}}}
\newcommand{\G}{\ensuremath{\mathbb{G}}}
\newcommand{\I}{\'\i}
\newcommand{\N}{\ensuremath{\mathbb{N}}}
\newcommand{\Q}{\ensuremath{\mathbb{Q}}}
\newcommand{\R}{\ensuremath{\mathbb{R}}}
\newcommand{\C}{\ensuremath{\mathbb{C}}}
\newcommand{\Z}{\ensuremath{\mathbb{Z}}}
\newcommand{\K}{\ensuremath{\mathbb{K}}}

\newcommand{\Rn}{\ensuremath{\mathbb{R}^n}}
\newcommand{\Rm}{\ensuremath{\mathbb{R}^m}}
\newcommand{\flecha}{\ensuremath{\rightarrow}}
\newcommand{\en}{\ensuremath{\in}}
\newcommand{\todo}{\ensuremath{\forall}}
\newcommand{\lam}{\ensuremath{\lambda}}
\newcommand{\Lam}{\ensuremath{\Lambda}}
\newcommand{\Rtres}{\ensuremath{\mathbb{R}^3}}
\newcommand{\x}{\ensuremath{x}}
\newcommand{\fxla}{\ensuremath{f(x,\lam)}}
\newcommand{\cp}{\ensuremath{C^p}}
\newcommand{\Ga}{\ensuremath{\Gamma}}
\newcommand{\Li}{\ensuremath{L_\infty}}
\newcommand{\xraya}{\ensuremath{\overrightarrow{x}}}
\newcommand{\yraya}{\ensuremath{\overrightarrow{y}}}
\newcommand{\zraya}{\ensuremath{\overrightarrow{z}}}
\newcommand{\uraya}{\ensuremath{\overrightarrow{u}}}


\newcommand{\alge}[1]{\ensuremath{\mathfrak{#1}}}
\newcommand{\campo}{\ensuremath{\mathbb{K}}}
\newcommand{\campon}{\ensuremath{\mathbb{K}^n}}
\newcommand{\evec}[1]{\ensuremath{\boldsymbol{\mathcal{#1}}}}
\newcommand{\field}[1]{\ensuremath{\mathbb{#1}}}
\newcommand{\ie}{\mbox{\it i. e.}}
\newcommand{\indice}[2]{\index{#1} \mbox{\sc{#2}}}
\newcommand{\irra}{\ensuremath{\mathbb{I}}}
\newcommand{\norma}[1]{\ensuremath{\left\| \vec{#1} \right\|}}
\newcommand{\prodesc}[2]{\ensuremath{\left\langle \vec{#1} \, \left\lvert \, \vec{#2} \right. \right\rangle}}
\newcommand{\set}[1]{\ensuremath{\mathbf{#1}}}
\newcommand{\st}{\ensuremath{\rotatebox[origin=c]{90}{$\ni$}}}
\newcommand{\ud}{\mathtt{d}}
\newcommand{\ve}[1]{\ensuremath{\boldsymbol{\vec{\mathtt{#1}}}}}

%\newcommand{\suchthat}{\mathrel{\ooalign{$\ni$\cr\kern-1pt$-$\kern-6.5pt$-$}}}

\def\demo{\noindent\mbox{\sc Demostraci\'on:}\\}
\def\notac{\noindent\mbox{\sc NOTACI\' ON}\\}
\def\fin{\hbox{\vrule height6pt width8pt depth0pt}}


\def\titulo{
    \begin{tabular}{ccc}
      &{\Huge Integraci\'on}                                              & \\
      &--------------------------- {\Large de} ---------------------------& \\
      &{\Huge Lebesgue}                                                   & \\[.6in]
      &{\Large -- Versi\'on 0 --}
    \end{tabular}}

\def\autor{\Large Dr. Olgierd Alf Biberstein             \\
           \Large Herschd\"orfer                         \\[.6in]
           INSTITUTO POLIT\'ECNICO NACIONAL              \\
            Escuela Superior de F\I sica y Matem\'aticas \\
            Departamento de Matem\'aticas                \\[.1in]
           }

\def\nonumchapter#1{%
    \chapter*{#1}
    \addcontentsline{toc}{chapter}{#1}}

\def\prefacesection#1{%
    \chapter*{#1}
    \addcontentsline{toc}{chapter}{#1}}

\makeindex
\author{\autor}
\date{\sc M\'exico\\ 2019-02-28}%{\bf Date:} { Unknown}}

%%%%%%%%%%%%%%%%%%%%%%%%%%%%%%%%%%%%%%%%%%%%%%%%%%%%%%%%%%%%%%%%%%%%%%%%%%%%%%%%%%%%%%%%%%%%%%%%%%%%%%%%%%%%%%%%%%%%%%%%
%1 one inch + \hoffset
%2 one inch + \voffset
%\paperwidth     = 597.00pt  %
\paperheight    = 845.00pt  %
%\hoffset        =  -0.45in  %
%\voffset        =  -0.60in  %
%\marginparwidth =   0.00pt  %
%\marginparpush  =   5.00pt  %
%\marginparsep   =   7.00pt  %
%\textheight     =   8.90in  %
%\textwidth      =   6.20in  %
%\evensidemargin =  -0.10in  %pares
%\oddsidemargin  =  -0.10in  %impares
%\baselineskip   =  18.00pt  %
%\headheight     =   1.25cm  %
%\headsep1.12truecm          %
%\topmargin-0.51truecm       %
%\footskip       =  27.00pt  %
\linespread{1.17}
%%%%%%%%%%%%%%%%%%%%%%%%%%%%%%%%%%%%%%%%%%%%%%%%%%%%%%%%%%%%%%%%%%%%%%%%%%%%%%%%%%%%%%%%%%%%%%%%%%%%%%%%%%%%%%%%%%%%%%%%


\pagestyle{fancy}
\renewcommand{\chaptermark}[1]{\markboth{ Cap. \thechapter}{#1}}
\renewcommand{\sectionmark}[1]{\markright{\thesection\ #1}}
\fancyhf{}
\fancyhead[LE, RO]{\tt \thepage}
\fancyhead[LO]{\rightmark}
\fancyhead[RE]{\leftmark}
\renewcommand{\headrulewidth}{0.5pt}
\renewcommand{\footrulewidth}{0pt}
\addtolength{\headheight}{0.5pt}
\fancypagestyle{plain}{\fancyhead{}
\renewcommand{\headrulewidth}{0pt}}

\renewcommand{\footrulewidth}{0.3pt}
\addtolength{\headheight}{0.5pt}
\fancypagestyle{plain}{\fancyhead{}
\renewcommand{\headrulewidth}{0pt}}

\fancyfoot[LE]{\small An\'alisis V. 3}
\fancyfoot[RO]{\small Biberstein}

\setlength{\parindent}{0pt}
\newcommand{\forceindent}{\leavevmode{\parindent=1em\indent}}

\begin{document}

%\pagenumbering{roman}

%\title{\titulo}

%\maketitle

%\tableofcontents


\pagenumbering{arabic}
\chapter{Funciones definidas por integrales. Integrales impropias en \R}

\setcounter{page}{677}
\renewcommand{\theenumi}{\roman{enumi}}
%\begin{enumerate}[label = \uline{\arabic*.}]

%1
%{\setlength\itemindent{30pt}
\subsection{Funciones definidas por integrales}
\textbf{Proposición 1.} (Continuidad de una función definida por una integral.) \\
Sea $\Lambda$ un espacio métrico y sea $f$ una aplicación  \Rn $\times$  $\Lambda \rightarrow$ \F . Se supone:
\begin{enumerate}[i)]
\item $\forall$ $\in$ $\Lambda$ la función $f^{\lambda}:x$ $\rightarrow$ $f(x,\lambda)$ es integrable en \Rn .
Su integral designará por:


\begin{equation*}
\boxed{\Phi (\lambda) =: \int_{\Rn}f(x,\lambda)dx}
\end{equation*}


Desde luego $\Phi$ así definida es una aplicación $\Lambda$ $\rightarrow$ \F .
\item La función $f_x: \lambda \rightarrow$ $f(x,\lambda)$ es continua en cierto punto $\mu \in$ $\Lambda$ para casi todo $x$ $\in$ \Rn .
\item Existe $g$ $\in$ $\mathcal{L}_1(\Rn,\R)$ tal que $||f(x,\lambda)||$ $ \leq$ $g(x)$ para casi todo $x$ $\in$ \Rn \phantom{} y $\forall$ $\lambda$ $\in$ $\Lambda$.
Entonces, la función $\Phi$ es continua en $\mu$. \\
\underline{Demostración.} 

Sea $\lbrace \lambda_k \rbrace_{k \in \N}$ una sucesión de puntos de $\Lambda$ tal que $
\lim_{k \to +\infty}(\lambda_k)=\mu$. \\
Debemos probar que $\lim_{k \to +\infty}\Phi (\lambda_k)=\Phi(\mu)$. \\
Por la condición $\mathrm{II}$:  $\lim_{k \to +\infty}f^{\lambda_k}(x)=f^\mu (x)=f(x,\mu)$ para casi todo $x$ $\in$ \Rn . \\
La condición $\mathrm{III}$ dice: $||f^{\lambda_k}(x)||\leq g(x)$ para casi todo $x$ $\in$ \Rn y $\forall$ $k$ $\in$ \N .
Así pues, por el teorema de Lebesgue:
\begin{equation*}
\lim_{k \to +\infty}\Phi (\lambda_k)=\lim_{k \to +\infty}\int_{\Rn}f^{\lambda_k}=\int_{\Rn}f^\mu=\int_{\Rn}f(x,\mu)dx=\Phi (\mu) .
\end{equation*}
\end{enumerate}
\textbf{Corolario}
Sea $\Lambda$ un espacio métrico compacto y sea \K \phantom{} un subconjunto compacto de \Rn. Si la función $f:$ $K$ $\times$ $\rightarrow$ \F \phantom{} es continua en $K$ $\times$ $\Lambda$, la función $\Phi:$ $\Lambda$ $\rightarrow$ \F \phantom{} definida por:


\begin{equation*}
\boxed{\Phi (\lambda)=\int_K f(x,\lambda)dx}
\end{equation*}

es continua en $\Lambda$.\\

\underline{Demostración}\\

Definimos $f:\Rn \times \Lambda \rightarrow \F$ por la  fórmula:

\begin{equation*}
\widehat{f}(x,\lambda)= \left\{ \begin{array}{lcc}
             f(x,\lambda) &   si  & x \in K \\
             \\ 0 &  si &  x \in \Rn - K \\
             \end{array}
   \right.
\end{equation*}

\begin{enumerate}[i)]
\item $\forall$ la función $x \rightarrow$ $f(x,\lambda)$ es continua, luego integrable sobre $K$ (o sea $\Phi$ está bien definida). Esto equivale a decir que la función $f^\lambda:x$ $\rightarrow$ $\widehat{f}(x,\lambda)$ es integrable en \Rn $\forall$ $\lambda$ $\in$ $\Lambda$
\item La función $\widehat{f}_x:\lambda$ $\rightarrow$ $\widehat{f}(x,\lambda)$ es continua en $\Lambda$ $\forall$ $x \in$ \Rn .
\item Puesto que $K$ $\times$ $\Lambda$ es un espacio métrico compacto, $\exists$ $M>0$ tal que $ ||f(x,\lambda)|| \leq M$ $\forall$ $(x,\lambda) \in$ $K$ $\times \Lambda$. Luego se verifica:

\begin{equation*}
||\widehat{f}(x,\lambda)|| \leq Mx_A (x) \forall x \in \Rn ,  \forall \lambda \in K
\end{equation*}
Pero $Mx_A$ es una función integrable \Rn \flecha \R . \\
Por la proposición precedente aplicada a $\widehat{f}$, la función \lam \flecha $\int_\Rn f(x,\lam)dx$ es continua en \Lam . Pero esta función no es otra que $\Phi$, c. q. d.
\end{enumerate}

\underline{Ejemplos.}
En los ejemplos a continuación se podrá aprender algunos recursos para determinar en la practica la función mayorante $g$ de la prop. 1. 

\begin{enumerate}[1)]
\item Volvamos al ejemplo 2) del fin del cap. V. Sea $K$ un compacto de $\mathbb{R}^3$  y $\mu$ una función real, medible y acotada en $K$. \\
 Sea $||$ $||$ la norma euclidiana en  $\mathbb{R}^3$. Sabemos que \todo  \phantom{}$a$ \en  $\mathbb{R}^3$ está definida la función: 
 
$$ U(a)=\int_K \frac{\mu (x)dx}{|| x-a ||}$$ (el potencial creado en el punto $a$ por la carga de densidad $\mu$ distribuida sobre $K$).\\
Sea $M=:Sup$ $| \mu (x)|$. Sea $\delta>0$ y sea $S_\delta=\lbrace a|d(a,K)>\delta \rbrace$.\\
Aquí $d$ es la distancia euclidiana en  \Rtres . \\
Nos restringimos al caso $a$ \en $S_\delta$ y definimos $\widehat{f}:\Rtres \times$ $S_\delta$ \flecha \R \phantom{} por

\begin{equation*}
\widehat{f}(x,a)= \left\{ \begin{array}{lcc}
             \frac{\mu (x)}{||x-a||} &   si  & x \en K \\
             \\ 0 &  si &  x \in  \Rtres - K \\
             \end{array}
   \right.
\end{equation*}
Entonces $U(a)=\int_{\Rtres} \widehat{f}(x,a)dx$.\\
\todo $x$ \en  \Rtres la función $a$ \flecha $\widehat{f}(x,a)$ es continua en $S_\delta$. \\
También \todo $x$ \en $S_\delta :$ $|\widehat{f}(x,a)|\leq \frac{M}{\delta}x_K (x)$, donde el segundo miembro es una función integrable en \Rtres, independiente de $a$. \\
Por la prop. 1 . $U$ es continua en $S_\delta$. Sea ahora $a$ \en \Rtres $-K$ arbitrariamente fijado. Para $\delta<d(a,K)$ se tendrá $a \en$ $S_\delta$.\\
Puesto que $S_\delta$ es abierto y que $U$ es continua en $S_\delta$, $U$ es continua en $a$. Así pues $U$ es continua en $\Rtres - K$. \\
De hecho, se puede demostrar que $U$ es continua \Rtres \phantom{} entero, pero la prop. 1 ya no es suficiente para ello. Véase el ejercicio.

\item Sea $f$ \en  $\mathcal{L}_1(\R,\F)$. \\
\todo  \phantom{} $x$ \en \R \phantom{} la función $t$ \flecha $e^{-ixt}$ es medible y satisface (1) $||e^{-ixt} f(t)||=||f(t)||$. Luego dicha función es integrable en \R. Pongamos:

\begin{equation*}
\widehat{f}(x)=:\int_{-\infty}^{+\infty} e^{-ixt} f(t)dt, \todo  \phantom{} x \in \R
\end{equation*}
\todo  \phantom{} $t$ \en \R \phantom{} la función $x$ \flecha $e^{-ixt}f(t)$ es continua en \R y la relación (1) muestra que se puede aplicar la prop. 1. Así, pues $\widehat{f}$ es continua en \R .
\item Consideremos la función $\Gamma$ de Euler:
\begin{equation*}
\Gamma (x)=\int_0^{+\infty}e^{-t}t^{x-1}dt, \todo x>0.
\end{equation*}

\todo  \phantom{} $t$ \en $[0,+\infty[$ la función $x$ \flecha $e^{-ixt}t{x-1}$ es continua en $]0,+\infty[$.\\
Sean a,b números tales que $0<a<b$. Restrinjamos $x$ al intervalo $]a,b[$. Sea \todo \en $]0,+\infty[$:
\begin{equation*}
g(t)= \left\{ \begin{array}{lcc}
             e^{-t}t^{a-1} &   si  & 0 \leq t \leq 1 \\
             \\ e^{-t}t^{b-1} &  si &  t >1 \\
             \end{array}
   \right.
\end{equation*}
$g$ es integrable en $]0,+\infty[$ y se tiene\\
$$e^{-t}t{x-1} \leq g(t,  \phantom{}  \todo t  \phantom{} \en [0,+\infty[,  \phantom{} \todo x \en ]a,b[$$
Luego $\Gamma$ es continua en  $]a,b[$. Puesto que todo $x>0$ pertenece a un tal intervalo  $]a,b[$ convenientemente elegido, \underline{$\Gamma$ es continua en  $]0,+\infty[$.}\\
\end{enumerate}
\textbf{Proposición 2.} (Derivación bajo el signo integral).\\
Sea $\mathrm{I}$ un intervalo \R . Sea $f$ una aplicación \Rn $\times$  $\mathrm{I}$ \flecha \F .\\
Se supone:
 \begin{enumerate}[i)]
 \item \todo \phantom{}  \lam \en $\mathrm{I}$ la función $f^\lam:$ $x$ \flecha $f(x,\lambda)$ es integrable en \Rn .\\
 Se define la funcion $\Phi:$ $\mathrm{I} \flecha$ $\F$ por: 
 
\begin{equation*}
\boxed{\Phi (\lam)=: \int_{\Rn} f(x,\lam)dx, \todo \lam \en \mathrm{I}}
\end{equation*}

\item Para casi todo $x$ \en \Rn \phantom{} y \todo \phantom{} \lam \en $\mathrm{I}$ existe la derivada.\\

\begin{equation*}
f'_x(\lam)=:\lim_{h \to 0}\frac{f(x,\lam + h)-f(x,\lam)}{h}
\end{equation*}

\item $\exists$  $g$ \en $\mathcal{L}_1(\Rn,\R)$ tal que $||f'_x(\lam)||\leq g(x)$ para casi todo $x$ \en \Rn  \phantom{} y \todo \phantom{} \lam \en $\mathrm{I}$.\\
$||f'_x(\lam)||\leq g(x)$  para casi todo $x$ \en \Rn \phantom{} y \todo \phantom{} \lam \en $\mathrm{I}$.\\
Entonces la función $x$ \flecha $f'_x(\lam)$ es integrable en \Rn , $\Phi$ es derivable en $\mathrm{I}$ y \\

\begin{equation*}
\boxed{\Phi^{'} (\lam)=\int_{\Rn} f'_x(\lam)dx, \phantom{s} \todo \lam \phantom{s} \en \mathrm{I}} 
\end{equation*}

\underline{Demostración.}\\
Fijemos una vez para siempre \lam \en $\mathrm{I}$ . Sea $\lbrace h_\upsilon \rbrace$ una sucesión de números reales tales que $h_\upsilon \neq 0$, \lam $+h_\upsilon$ \en $\mathrm{I}$ , $\lim_{v \to \infty}h_\upsilon=0$.\\

Pongamos \todo \phantom{} $x$ \en \Rn : $g_\upsilon (x)=:\frac{f(x,\lam+h_\upsilon)-f(x,\lam)}{h_\upsilon}$\\
Por la hipótesis $\mathrm{I})$ $\lbrace g_\upsilon \rbrace$ es una sucesión de funciones integrables \Rn \flecha \F . Además: \\
\begin{equation*}
\frac{\Phi (\lam + h_\upsilon)-\Phi (\lam)}{h_\upsilon}=\int_{\Rn} g_\upsilon
\end{equation*}
Por la hipótesis $\mathrm{II})$ se tiene:\\
$(1)$  $\lim_{v \to +\infty}g_{\upsilon}(x)=f'_x(\lam)$ para casi todo $x$ \en \Rn. \\
Finalmente por la hipótesis $\mathrm{III}$ y la formula de los incrementos finitos:\\
$(2)$ $||g_\upsilon (x)|| \leq \stackbin[0\leq t \leq 1]{}{Sup}||f'_x(\lam+th_\upsilon)|| \leq g(x)$ para casi todo $x$ \en \Rn .\\ \\
De (1) y (2) se deduce, aplicando el teorema de Lebesgue, que la función $x$ \flecha $f^{'}_x (\lam)$ es integrable y que:\\
$$
\Phi^{'}(\lam)=\lim_{\upsilon \to +\infty}\frac{\Phi (\lam+h_\upsilon)-\Phi (\lam)}{h_\upsilon}=\lim_{\upsilon \to +\infty} \int_{\Rn} g_\upsilon = \int_{\Rn} f^{'}_x(\lam)dx
$$ \hspace{11cm} c. q. d \\

\underline{Observación.}\\
Agreguemos a las hipótesis de la prop. 2 la siguiente:
\item \underline{Para casi todo $x$ \en \Rn , la función $f'_x$ es continua en $\mathrm{I}$.}\\
Al combinar con la prop. 1 se ve que entonces \underline{$\Phi '$ es continua en $\mathrm{I})$.}
 \end{enumerate} 
 
En otras palabras \underline{$\Phi$ es de clase $C^1$ en $\mathrm{I})$.}\\

\textbf{Corolario.}\\
Sea $K$ un subconjunto compacto de \Rn y sea $\mathrm{I}$ un intervalo compacto de \R . Sea $f:K$ $times$ $\mathrm{I}$ \flecha \F . Se supone: \\

\begin{enumerate}[(i)]
\item \todo \phantom{} \lam \en $\mathrm{I}$, la función $f^\lam:x$ \flecha $f(x,\lam)$ es integrable en $K$. \\
Se define $\Phi:\mathrm{I}$ \flecha \F \phantom{} por: \\
\begin{equation*}
\boxed{\Phi(\lam)=:\int_K f(x,\lam)dx}
\end{equation*}

\item La aplicación $(x,\lam)$ \flecha $f'_x (\lam)$ existe y es continua en $K$ $\times \mathrm{I}$. \\
\underline{$\Phi$ es de clase $C^1$ en $\mathrm{I}$ y:}\\

\begin{equation*}
\boxed{\Phi ' (\lam)=\int_K f'_x(\lam) dx.}
\end{equation*}
\end{enumerate}

\underline{Demostración.} \\
Definamos $f:\Rn \times \mathrm{I}$ \flecha \F \phantom{} por:
\begin{equation*}
\tilde{f}(x,\lam)= \left\{ \begin{array}{lcc}
             f(x,\lam) &   si  & x \en K \\
             \\ 0 &  si &  x\en \Rn -K \\
             \end{array}
   \right.
\end{equation*}
Se tiene:
\begin{enumerate}[(i)]
\item La función $\tilde{f}:x$ \flecha $\tilde{f}(x,\lam)$ es integrable en \Rn , siendo:
$$
\Phi (\lam)=\int_{\Rn} \tilde{f}(x,\lam)dx.
$$

\item \todo $x$ \en \Rn existe la derivada $\tilde{f}'_x=\left\{ \begin{array}{lcc}
             f(x,\lam) &   si  & x \en K \\
             \\ 0 &  si &  x\en \Rn -K \\
             \end{array}
               \right.$
\item Sea $M=:\stackbin[(x,\lam) \en K\times  \mathrm{I}]{}{M\acute{a}x}||f'_x(\lam)||$ \\
Entonces \todo \phantom{} $x$ \en \Rn \phantom{} \todo \phantom{} \lam \en  $\mathrm{I}$.
$$
||\tilde{f}'_x(\lam)|| \leq Mx_K (x).$$

Aquí el segundo miembro es una función integrable en \Rn . \\

\item \todo \phantom{} $x$ \en \Rn , $\tilde{f}'_x$ es continua en  $\mathrm{I}$. \\
Por la prop. 2 y la observación que la sigue, $\Phi$ es de clase $C^1$ en   $\mathrm{I}$ y se tiene:\\
$$
\Phi ' (\lam)= \int_{\Rn} \tilde{f}'_x (\lam)=\int_K f'_x(\lam)dx, \todo \phantom{s} \lam \en  
\mathrm{I}$$   \hspace{11cm} c. q. d \\
\end{enumerate}

\underline{Ejemplo: (Derivadas sucesivas de la  $\Gamma$).} \\

Volvamos a la función $\Gamma=\int_0^{+\infty}e^{-t}t^{x-1}dx$, \todo \phantom{} \lam $x>0$. \\ \\
Afirmamos que $\Gamma$ es de clase $C^1$ en $]0,+\infty[$ y que se puede calcular su derivada mediante la derivación bajo el signo integral o sea \\

\begin{equation}
\boxed{\Gamma ' (x)=\int_0^{+\infty}e^{-t}t^{x-1}Log(t) dt, \todo \phantom{s} x\en ]0,+\infty[.}
\end{equation}
Sean a,b números tales que $0<a<b$. Sea $g: ]0,+\infty[.$ \flecha \R \phantom{} la función definida por:\\
\begin{equation*}
g(t)= \left\{ \begin{array}{lcc}
            e^{-t}t^{a-1}|Log(t)| &   si  & 0<t \leq 1 \\
             \\  e^{-t}t^{b-1}Log(t) &  si &  t>1 \\
             \end{array}
   \right.
\end{equation*}
Se tiene $e^{-t}t^{a-1}|Log(t)|$ $\sim_{t \to 0}$ $t^{a-1}|Log(t)|= o \frac{1}{t^{1-\frac{a}{2}}}$ \\
y $e^{-t}t^{b-1}Log(t)=_{t \to +\infty}o \frac{1}{t^2}$ \\
Luego la función $g$ es integrable en $]0,+\infty[$. Además, \todo $x$ \en $]a,b[$ y \todo \phantom{} $t \en ]0,+\infty[.$:
$$
e^{-t}t^{x-1}|Log(t)| \leq g(t).
$$
Como también la función $x$ \flecha $e^{-t}t^{x-1}Log(t)$ es continua, se sigue la prop. 2 y de la observación después de ella que $\Gamma$ es de clase $C^1$ en $]a,b[$ y que $\Gamma '$ está dada por la fórmula (1) \todo \phantom{} $x$ \en $]a,b[$. La arbitrariedad de $a$,$b$ entraña que $\Gamma$ es de clase $C^1$ en $]0,+\infty[$ y que $\Gamma '$ está dada por la formula (1.1) \todo \phantom{} $x$ \en $]0,+\infty[$. \\
Por inducción se demuestra fácilmente que, de hecho \\ \underline{$\Gamma$ es de clase  $C^\infty$ en $]0,+\infty[$ y que \todo \phantom{} $k$ \en \N \phantom{} y \todo \phantom{} $x$ \en $]0,+\infty[$ se tiene }\\

\begin{equation*}
\boxed{\Gamma^{(k)}(x)=\int_0^{+\infty} e^{-t}t^{x-1}{(Log(t))}^kdt.}
\end{equation*}
\phantom{recorridodobles} \underline{Variación de la función $\Gamma$.}\\
Se tiene \todo $x>0$:\\
$$
\Gamma''=\int_0^{+\infty} e^{-t}t^{x-1}{(Log(t))}^2dt>0.
$$
luego $\Gamma'$ es estrictamente creciente en $]0,+\infty[$. \\
Puesto que $\Gamma(1)=\Gamma(2)=1$, por el teorema de Rolle existe $\alpha$ \en  $]1,2[$ tal que $\Gamma' (\alpha)=0$. Por consiguiente $\Gamma'(x)<0$ si $0<x<\alpha$ y $\Gamma'>0$ \todo \phantom{} $x>\alpha$. \\
De est o se sigue que, a su vez, $\Gamma$ es estrictamente decreciente en $]0,\alpha$ y estrictamente creciente en $]\alpha,+\infty[$. Puesto que  $\Gamma(n+1)=n:$ \todo \phantom{} $n \en \N$, $\Gamma$ no es acota superiormente en $]\alpha,+\infty[$.\\
Por monotonicidad:\\
\begin{equation*}
\boxed{\stackbin[x \to +\infty]{}{\Gamma(x)} \longrightarrow +\infty}
\end{equation*}
Finalmente $x \Gamma(x)={\Gamma(x+1)}_{x \to 0^{+}}$ \flecha 1, de donde:\\
\begin{equation*}
\boxed{\stackbin[x \to 0^{+}]{}{\Gamma(x)} \sim \frac{1}{x}}
\end{equation*}
en particular $\stackbin[x \to 0^{+}]{}{\Gamma(x)} \longrightarrow +\infty$. \\
El resumen de nuestros resultados está en la siguiente: \\ \\

%%%%%%%%%% TABLA DE VARIACION %%%%%%%%%%%%
Un cálculo numérico suministra por cierto las aproximaciones \\
$$ 
\alpha \approx 1.46 
$$
$$
\Gamma (\alpha) \approx 0.88
$$
%%%%%%%%%% GRAFICA DE LA FUNCION GAMMA %%%%%%%%%%%%
\\
\subsection{Funciones de clase $C^P$ definidas por integrales.}

\textbf{Proposición 3. } \\
Sea \Lam \phantom{} un conjunto abierto en \Rm y sea $f:(x,\lam)$ \flecha $f(x,\lam)$ una función: \Rn $\times$ \Lam \flecha \F . Se supone:\\
\begin{enumerate}[(i)]
\item Para casi todo $x$ \en \Rn, la función $f_x$ \lam \flecha $f(x,\lam)$ es de clase $C^p$ en \Lam.
\item \todo \phantom{} \lam \en \Lam \phantom{} todas las funciones \x \flecha $\delta_{\alpha_1} \ldots \delta_{\alpha_k}f_x(\lam)$, donde $1\leq \alpha_i \leq m$ para $i=1,2,\ldots,k$ y $0\leq k \leq p$, son integrables en \Rn. \\
El caso $k=0$ significa que la función \x \flecha \fxla \phantom{} es integrable en \Rn . Se define la función $\Phi:\Lam$ \flecha \F \phantom{} por: \\
\begin{equation*}
\boxed{\Phi (\lam)=:\int_{\Rn}\fxla dx , \phantom{s} \todo \phantom{s} \lam \en \Lam}
\end{equation*}
\item Para todo cubo cerrado $K$ contenido en \Lam \phantom{} y toda sucesión $(\alpha_1,\ldots, \alpha_p)$ de $p$ indices en $[1,m]$ existe una función integrable:\\
$g_{\alpha_1,\ldots,\alpha_p }^{K}: \Rn$ \flecha \R \phantom{} para casi todo \x \en \Rn \phantom{} y \todo \phantom{} \lam \en $K$ se tiene $||\delta_{\alpha_1} f_x(\lam)|| \leq g_{\alpha_1,\ldots, \alpha_p}^{K}(x)$.\\
Entonces $\Phi$ es de clase \cp en \Lam \phantom{} y para $k=1,\ldots,p$ las derivadas parciales de orden $k$ de $\phi$ están dadas por:\\
\begin{equation*}
\boxed{\delta_{\alpha_1} \ldots \delta_{\alpha_k}\Phi(\lam)=\int_{\Rn} \delta_{\alpha_1} \ldots \delta_{\alpha_k}f_x(\lam)dx, \phantom{s} \todo \phantom{s} \lam \en \Lam}
\end{equation*}
\end{enumerate}

\underline{Demostración.}\\
\begin{enumerate}[a)]
\item Mostremos que una mayoración análoga a $(\mathrm{III})$ vale para todas las derivadas parciales de orden $k\leq p$. Bata admitir este resultado para las derivadas de orden $k+1$ y probarlo para las derivadas de orden $k$. Para abreviar pongamos $D=\delta_{\alpha_1} \ldots \delta_{\alpha_k}$.\\
Sea $K$ un cubo cerrado contenido en \Lam , sea $\lam_o=(\lam_1^o,\ldots,\lam_m^o)$ el centro de $K$ y sea $a$ la media arista de $K$.\\
Por la formula de incrementos finitos se tiene para casi todo \x \en \Rn \phantom{} y \todo \phantom{} \lam$=(\lam_1,\ldots,\lam_m)\en K$:\\
$$
||Df_x(\lam)-Df_x(\lam_o)|| \leq \sum_{\upsilon=1}^{m}||Df_x(\lam_1,\ldots,\lam_{\upsilon},\lam_{\upsilon+1}^o,\ldots,\lam_m^o)-
$$
$$
Df_x(\lam_1,\ldots,\lam_{\upsilon -1},\lam_{\upsilon}^o,\lam_{\upsilon +1},\ldots,\lam_m^o)
$$
\begin{equation*}
a \sum_{\upsilon =1}^{m}\stackbin[0 \leq t \leq 1]{}{Sup} || \alpha_\upsilon Df_x(\lam_1,\ldots,\lam_{\upsilon-1},t \lam_{\upsilon}^o+(1-t)\lam_\upsilon,\lam_{\upsilon+1},\ldots,\lam_\upsilon)
\end{equation*}

$
\leq a \sum_{\upsilon =1 }^{m}g_{\upsilon^K,\alpha_1,\ldots,\alpha_k}(x)$, de ahí que:\\

\begin{equation*}
||Df_x(\lam)|| \leq a \sum_{\upsilon=1}^{m}g_{\upsilon^K,\alpha_1,\ldots,\alpha_k}(x)+||Df_x(\lam_o)||
\end{equation*}
o, al poner \\
$$
g_{\upsilon,\alpha_1,\ldots,\alpha_k}^K(x)=\sum_{\upsilon =1}^{m}g_{\upsilon,\alpha_1,\ldots,\alpha_n}^K(x)+||Df_x(\lam_o)||$$
se tendrá bien \todo \phantom{} \lam \en $K$: $|| \delta_{\alpha_1},\ldots,\delta_{\alpha_k}f_x(\lam)|| \leq || \leq g_{\upsilon,\alpha_1,\ldots,\alpha_k}^K(x)$ donde $g_{\upsilon,\alpha_1,\ldots,\alpha_k}^K$ es una función integrable en \Rn , independiente de \lam .\\
\item  Probemos por inducción sobre $k$, que es legitimo calcular las derivadas parciales de orden $k$ por derivación bajo el signo integral.\\
\begin{enumerate}[1)]
\item \underline{Caso $k=1$} \\
Sea $\lam_o$ \en \Lam \phantom{} y sea $K$ un cubo cerrado de centro $\lam_o$ \phantom{} contenido en \Lam .\\
Sea $a$ la media arista de $K$. Fijemos un indice $i \en [1,m]$ y consideremos la función:\\
\begin{equation}
=\lam_i \flecha \int_{\Rn}f(x,\lam_1^o, \ldots,\lam_{i-1}^o,\lam_i,\lam_{i+1}^o,\ldots,\lam_m^o) dx.
\end{equation}
donde $-a+\lam_i^o<\lam_i<a+\lam_i^o$.\\
Para casi todo \x \en \Rn \phantom{} y \todo \phantom{} $\lam_i$ \en $]-a+\lam_i^o,a+\lam_i^0[$ se tiene: \\
$||\delta_if(x,\lam_1^o,\ldots,\lam_{i-1}^o,\lam_i,\lam_{i+1}^o,\ldots,\lam_m^o)||\leq g_i^K(x)$, donde $g_i^K(x)$ es una función integrable en \Rn , independiente de $\lam_i$. \\ 
Por prop. 2 aplicada a la función (1.2) en el punto $\lam_i$ \phantom{} se obtiene bien:\\
\begin{equation*}
\delta_i(\lam_o)=\int_{\Rn}\delta_i f_x(\lam_o)dx.
\end{equation*}

\item Sea $k \leq p$. Supongamos el resultado ya establecido para las derivadas de orden $k-1$.\\
Sea $\delta_{\alpha_1},\delta_{\alpha_2},\ldots,\delta_{\alpha_k}$ una derivación parcial de orden $k$. Por hipótesis de inducción:\\
$$
\delta_{\alpha_2} \ldots \delta_{\alpha_k}\Phi(\lam)=\int_{\Rn}\delta_{\alpha_2} \ldots f_x(\lam)dx, \phantom{s} \todo \phantom{s} \lam \en \Gamma
$$
Sea $\lam_o \en \Gamma$ y sea $K$ un cubo cerrado de centro $\lam_o$ contenido en $\Gamma$. Sea $a$ la media arista de $K$. Consideremos la función: \\
\begin{equation}
=\lam_{\alpha_1} \flecha \int_{\Rn}\delta_{\alpha_2},\ldots,\delta_{\alpha_k}f_x(\lam_1^o,	\ldots,\lam_{\alpha_1-1}^o,\lam_{\alpha_1},\lam_{\alpha_1+2}^o,\ldots,\lam_m^o) dx=
\end{equation}

\begin{equation*}
\delta_{\alpha_2},\delta_{\alpha_k}\Phi(\delta_1^o,\ldots,\lam_{\alpha_1-1}^o,\lam_{\alpha_1},\lam_{\alpha_1+1}^o,\ldots,\lam_m^o), donde
\end{equation*}

$-a+\lam_{\alpha_1}^o<\lam_{\alpha_1}<a+\lam_{\alpha_1}^o$ \\

Para casi todo \x \en \Rn \phantom{} y para todo $\lam_{\alpha_1} \en ]-a+\lam_{\alpha_1}^o,a+\lam_{\alpha_1}^o[$ se tiene $||\delta_{\alpha_2},\delta_{\alpha_k}\Phi(\delta_1^o,\ldots,\lam_{\alpha_1-1}^o,\lam_{\alpha_1},\lam_{\alpha_1+1}^o,\ldots,\lam_m^o)||\leq$
$g_{\upsilon,\alpha_1,\ldots,\alpha_k}^K(x)$, donde $g_{\upsilon,\alpha_1,\ldots,\alpha_k}$ es una función integrable independiente de $\lam_{\alpha_1}$.\\
Por la prop. 2 aplicada a la función (1.3) en el punto $\lam_{\alpha_1}^o$ se consigue bien:\\
\begin{equation*}
\delta_{\alpha_1}\delta_{\alpha_2}\ldots \delta_{\alpha_k}\Phi(\lam_o)=\int_{\Rn}\delta_{\alpha_1}\delta_{\alpha_2}\ldots \delta_{\alpha_k}f_x(\lam_o)dx.
\end{equation*}

\item Queda por mostrar que todas las derivadas parciales $\delta_{\alpha_1}\ldots \delta_{\alpha_k}\Phi$, $1\leq k \leq p$, son continuas en $\Gamma$.\\
Sea  $\lam_o \en \Gamma$ y sea $K$ un cubo cerrado de centro $\lam_o$ contenido en $\Gamma$. Se tiene\\
$$
\delta_{\alpha_1}\ldots \delta_{\alpha_k}\Phi(\lam)=\int_{\Rn}\delta_{\alpha_1}\ldots \delta_{\alpha_k}f_x(\lam)dx, \todo \phantom{s} \lam \en K
$$
\end{enumerate}
\end{enumerate}
Para casi todo \x \en \Rn , la función \lam \flecha $\delta_{\alpha_1}\ldots \delta_{\alpha_k}f(x,\lam)$ es continua en $\lam_o$. \\
Además, para casi todo \x \phantom{} en \Rn \phantom{} y \todo \phantom{} \en \K :\\
$|| \delta_{\alpha_1}\ldots \delta_{\alpha_k}f_x(\lam)|| \leq g_{\alpha_1,\ldots,\alpha_k}^K (x)$, donde $g_{alpha_1,\ldots,\alpha_k}$ es una función integrable independiente de $\Gamma$.\\
Por la prop. 1. $\delta_{\alpha_1}\ldots \delta_{\alpha_k}\Phi$ es continua en $\lam_o$. \\ \\
\textbf{Coralario.}\\
Sean $A$ un subconjunto compacto de \Rn \phantom{} de \Rn , $\Gamma$ un abierto en \Rm , $f$ una aplicación $A \times \lam$ \flecha \F .\\ 
Se supone que \todo \phantom{} \x \en $A$ y \todo \phantom{} \lam \en $\Gamma$ existen las derivadas parciales $\delta_{\alpha_1}\ldots \delta_{\alpha_k}f_x(\lam)$, $1\leq k \leq p$ y que todas las funciones $(x,\lam)\flecha \delta_{\alpha_1}\ldots \delta_{\alpha_k} f_x(\lam)$ con $0 \leq k \leq p$, en particular la función $(x,\lam) \flecha f(x,\lam)$, son continuas en $A \times \lam$. \\
Sea 

\begin{equation*}
\boxed{\Phi (\lam)=\int_A f(x,\lam)dx}
\end{equation*}
Entonces $\Phi$ es de clase \cp \phantom{} en $\Gamma$ y, para $1\leq k \leq p$:

\begin{equation*}
\boxed{\delta_{\alpha_1}\ldots \delta_{\alpha_k}\Phi (\lam)=\int_A \delta_{\alpha_1}\ldots \delta_{\alpha_k}f_x(\lam)dx.}
\end{equation*}

\underline{Demostración.}\\
Sea $\tilde{f}:\Rn \times \Gamma \flecha \F$ definida por:\\

\begin{equation*}
\tilde{f}(x,\lam)= \left\{ \begin{array}{lcc}
            f(x,\lam) &   si &  x\en A \\
             \\  0 &  si & x\en \Rn-A \\
             \end{array}
   \right.
\end{equation*}

\begin{enumerate}[i)]
\item Si $0 \leq k \leq p$ tocas las funciones  \lam \flecha $\delta_{\alpha_1}\ldots \delta_{\alpha_k}\tilde{f}_x(\lam)$ existen y son continuas en \Ga \phantom{} para todo \x \en 
\Rn . Luego la función $\tilde{f}_x$ es de clase \cp \phantom{} en \Ga \phantom{} para todo \x \en \Rn . 
\item Para $0 \leq k \leq p$ y \todo \phantom{} \lam \en \Ga \phantom{} todas las funciones \x \flecha  $\delta_{\alpha_1}\ldots \delta_{\alpha_k}f_x(\lam)$ son continuas luego integrables en $A$. Esto significa que las funciones \x \flecha  $\delta_{\alpha_1}\ldots \delta_{\alpha_k}\tilde{f}_x(\lam)$ son integrables en \Rn , verificándose por cierto
\begin{equation*}
\int_{\Rn}\delta_{\alpha_1}\ldots \delta_{\alpha_k}\tilde{f}_x(\lam)dx=\int_A \delta_{\alpha_1}\ldots \delta_{\alpha_k}f_x(\lam)dx
\end{equation*}
En particular\\
$$
c(\lam)=\int_{\Rn}\tilde{f}(x,\lam)dx. \phantom{s} \todo \phantom{s}\lam \en \Ga
$$
\item Sea $K$ un cubo cerrado contenido en \Ga \phantom{} y sea $(\alpha_1,\ldots,\alpha_p)$ una sucesión de $p$ indices en $[1,m]$. Puesto que la función $(x,\lam) \flecha \delta_{\alpha_1}\ldots \delta_{\alpha_k}f_x(\lam)$ es continua en el compacto $A \times K$, existe:
\begin{equation*}
M_{\alpha_1,\ldots,\alpha_p}^K=:\stackbin[(x,\lam)\en A \times K]{}{M\acute{a}x}|| \delta_{\alpha_1}\ldots \delta_{\alpha_k}f_x(\lam)||
\end{equation*}
Se tiene, por lo tanto la mayoracion
$$
||\delta_{\alpha_1}\ldots \delta_{\alpha_k}f_x(\lam)|| \leq M_{\alpha_1,\ldots,\alpha_p}^Kx_A(x), \phantom{s} \todo \phantom{s} x \en \Rn \phantom{s} y \phantom{s} \todo \phantom{s} \lam \en K.
$$
\end{enumerate}
Aquí $M_{\alpha_1,\ldots,\alpha_p}^Kx_A$ es una función integrable en \Rn independiente de \Ga . Aplicando ahora la prop. 3 concluimos que $\Phi$ es de clase \cp \phantom{} en \Ga \phantom{} y que, para $1\leq k \leq p$:
\begin{equation*}
\delta_{\alpha_1}\ldots \delta_{\alpha_k}\Phi(\lam)=\int_{\Rn}\delta_{\alpha_1}\ldots \delta_{\alpha_k}\tilde{f}_x(\lam)dx=
\end{equation*}
$$
=\int_A \delta_{\alpha_1}\ldots \delta_{\alpha_k}f_x(\lam)dx
$$
\hspace{10cm} c. q. d

\underline{Ejemplo.}\\
Sea $m \geq 3.$ Para todo \lam$=(\lam_1,\ldots,\lam_m)$ tal que $\lam_1>0,\ldots, \lam_m>$ se pone:
\begin{equation*}
\boxed{\mathrm{I}(\lam)=:\int_0^{+\infty}\frac{dx}{\sqrt{(x+\lam_1)\ldots (x+\lam_m)}}}
\end{equation*}
La integral existe, puesto que $\frac{dx}{\sqrt{(x+\lam_1)\ldots (x+\lam_m)}} \stackbin[x \to +\infty]{} {\sim} x^{\frac{1}{m/2}}$ y $m/2>1$. Para $\alpha_1 \geq , \ldots, \alpha_m \geq 0$ hallamos:\\
$$
\delta_1^{\alpha_1}\cdots \delta_m^{\alpha_m}f_x(\lam)=\frac{(-1)\sum_{i=1}^{m}\alpha_i}{2^{\sum_{i=1}^{m}\alpha_i}}\frac{(2\alpha_1-1)!! \ldots (2\alpha_m-1)!!}{(x+\lam_1)^{\alpha_1+1/2}\cdots (x+\lam_m)^{\lam_m+1/2}}
$$
con la convención $(2\alpha_1-1)!!=1$, si $\alpha_i=0$.\\
Todas las funciones \lam \flecha $\delta_{1}^{\alpha_1}\cdots \delta_m^{\alpha_m}f_x(\lam)$ son continuas en $\Lam =: \phantom{}\underbrace{]0,+\infty [x\ldots x]0,+\infty[}_m$ si \x \en $]0,+\infty[$. 
\begin{enumerate}[i)]
\item Puesto que $\delta_1^{\alpha_1}\cdots \alpha_m^{\alpha_m}f_x(\lam)\stackbin[x \to +\infty]{}{\sim} \frac{1}{x^{m/2}+\sum_{i=1}^{m}\alpha_i}$ todas las funciones \x \flecha $\delta_1^{\alpha_1}\cdots \alpha_m^{\alpha_m}f_x(\lam)$ son integrables en $]0,+\infty[$.\\
Pongamos $c(\alpha_1,\ldots,\alpha_m)=:\frac{(-1)^{\sum_{i=1}^{m}\alpha_1}}{2^{\sum_{i=1}^{m}\alpha_i}}(2\alpha_1-1)!! \ldots (2\alpha_m-1)!!$
\item Fijemos arbitrariamente $a>0$. Se tiene la mayoracion:\\
\begin{equation}
\delta_1^{\alpha_1}\cdots \alpha_m^{\alpha_m}f_x(\lam) \leq c(\alpha_1,\ldots,\alpha_m) \frac{1}{(x+a)^{m/2+\sum_{i=1}^{m}\alpha_i}}, \todo \phantom{s} x>0
\end{equation}
y para $\lam_1 \geq a, \ldots , \lam_m \geq a$. La función \x \flecha $\frac{1}{(x+a)^{m/2+\sum_{i=1}^{m}\alpha_i}}$ es integrable en $]0,+\infty[$.\\
\end{enumerate}
Por la prop. 3 la función $\mathrm{I}$ es de clase $C^{\infty}$ en $\underbrace{]a,+\infty [x \ldots x]a}_m+!!!$ y sus derivadas parciales se calculan por derivación bajo el signo integral. Como $a$ es arbitrario, $\mathrm{I}$ es de clase $C^{\infty}$ en $\underbrace{]0,+\infty [x\ldots x]0,+\infty[}_m$ y se tiene:\\
$$
\underline{\delta_1^{\alpha_1}\cdots \alpha_m^{\alpha_m}\mathrm{I}(\lam)=c(\alpha_1,\ldots,\alpha_m)\infty_0^{+\infty}\int_0^{+\infty}\frac{dx}{(x+\lam_1)^{\alpha_1+1/2}\ldots (x+\lam_m)^{\alpha_m+1/2}}}$$ 
se encuentra p. ej.\\
$$
(\delta_1+\ldots+\delta_m)(\mathrm{I}(\lam)-\frac{1}{2}\int_0^{+\infty}\frac{1}{\sqrt{(x+\lam_1)\ldots (x+\lam_m)}}(\frac{1}{x+\lam_1}+\ldots+\frac{1}{x+\lam_m})dx=
$$
$$
\left.  \frac{1}{\sqrt{(x+\lam_1)\ldots (x+\lam_m)}} \right|_{x=0}^{+\infty}=-\frac{1}{\sqrt{\lam_1 \cdots \lam_m}}
$$

\subsection{Funciones monótonas.}

\textbf{Lema 1.} \\
Sea $\mathrm{I}$ un intervalo de \R \phantom{} y sea $f:\mathrm{I} \flecha \R$ una función decreciente (no necesariamente estrictamente decreciente) en $\mathrm{I}$. Sea $\upsilon \en \N$ dado. Entonces existe una función $psi_\upsilon$ "numerablemente escalonada" \\
$psi_\upsilon=\sum_{k \en \Z} c_k^\upsilon x_{J_k^\upsilon}$, donde $\lbrace J_k^\upsilon \rbrace_{k \en \Z}$ es una sucesión de intervalos disjuntos a pares cuya reunión es $\mathrm{I}$, tal que:

\begin{enumerate}[1)]
\item La restricción $psi_\upsilon |_{\mathrm{I}}$ es decreciente en $\mathrm{I}$.
\item \todo \phantom{} \x \en $\mathrm{I}$, $0 \leq f(x)-psi_\upsilon (x) < \frac{1}{\psi}$

\item $\psi_\upsilon \geq 0$ si $f \geq 0$. 
\end{enumerate}

\underline{Demostración.}\\
Dado $\upsilon \en \N$, \todo \phantom{} $k$ \en \Z \phantom{} definamos $J_k^\upsilon=\lbrace x| x\en \mathrm{I}, \frac{K}{\upsilon} \geq f(x) < \frac{k+1}{\upsilon} \rbrace$. \\
Afirmamos que $J_k^\upsilon$ es un intervalo contenido en $\mathrm{I}$, posiblemente reducido a un punto o vació. Para demostrarlo basta probar que si $x_1,x_2 \en J_k^\upsilon$, $x_1 < x_2$ y $x \en ]x_1,x_2[$, entonces $x \en J_k^\upsilon$. La hipótesis sobre \x \phantom{} y el decrecimiento de $f$ implican $\frac{k}{\upsilon} \leq f(x_2) \leq f(x) \leq f(x_1) < k$ o sea efectivamente \x \en $J_k^\upsilon$.\\
Los intervalos $J_k^\upsilon$ son disjuntos a pares e $\mathrm{I}=\underline{U}_{k \en \Z}J_k^\upsilon$ \\
Definamos $\psi_\upsilon: \R \flecha \R$, por:

\begin{equation*}
\psi_\upsilon= \sum_{k \en \Z} \frac{k}{\upsilon} x_{J_k^\upsilon}
\end{equation*}
\begin{enumerate}[1)]
\item Sean \x ,$y \en \mathrm{I}, x < y$. Si \x \en $J_k^\upsilon$, $y \en J_1$, se tiene $\frac{k}{\upsilon} \leq f(x) < \frac{k+1}{\upsilon}$ y $\frac{1}{\upsilon} \leq f(y) < \frac{1+1}{\upsilon}$. Pero $f(y) \leq f(x)$ luego $1 \leq j$, por lo tanto $\psi_\upsilon(y)= \frac{1}{\upsilon} \leq \frac{k}{\upsilon}=\psi_\upsilon (x)$. Así pues la función $\psi_\upsilon |_I$ es decreciente.
\item Es claro que \todo \phantom{} \x \en $\mathrm{I}$, $0 \leq f(x)-\psi_\upsilon (x) < \frac{1}{\upsilon}$.
\item De 2) se sigue $\psi_\upsilon (x) > f(x)-\frac{k}{\upsilon}$. Si $f \geq 0$ se tiene pues $\psi_\upsilon (x) > -\frac{1}{\upsilon}$. Ya que $\psi_\upsilon (x)=\frac{k}{\upsilon}$ para algún $k \en \Z$, necesariamente $k \geq ?$, luego $\psi_\upsilon (x) \geq 0$, \todo x.
\end{enumerate}

\textbf{Proposición 4.} \\
Sea $\mathrm{I}$ un intervalo de \R \phantom{} y sea $f: \mathrm{I} \flecha \R$ una función  monótona en $\mathrm{I}$. Entonces:
\begin{enumerate}[I)]

\item $f$ es medible en $\mathrm{I}$.
\item $f$ es integrable en todo subintervalo compacto de $\mathrm{I}$.
\end{enumerate}
\underline{Demostración. }\\
\begin{enumerate}[i)]
\item Al cambiar si hace falta, $f$ en $-f$, se puede suponer que $f$ es decreciente en $\mathrm{I}$. Por el lema 1, \todo \phantom{} $\upsilon \en \N$ exista una función numerablemente escalonada $\Phi_\upsilon =\sum_{k \en \Z}c_k^{\upsilon}x_{J_k}^{\upsilon}$ tal que $0 \geq f(x)- \Phi_\upsilon(x) < \frac{1}{\upsilon}$ \todo \x \en \phantom{} $\mathrm{I}$. Si $\tilde{f}$ es la amplificación canónica de $f$ a \R  , se tiene también $0 \leq f(x)- \Phi_\upsilon(x) < \frac{1}{\upsilon}$ \todo \x \R .Luego $\tilde{f}$ es limite (aún uniforme) en \R \phantom{} de la sucesión $\lbrace \Phi_\upsilon \rbrace$. Pero $\Phi_\upsilon =\lim_{m \to +\infty}\sum_{k=-m}^{m}c_k^{\upsilon}x_{J_k}^{\upsilon}$ en todo punto de \R .\\
Luego $\Phi_\upsilon$ es medible. Por conseguiente $\tilde{f}$ es medible. En otras palabras $f$ es medible en $\mathrm{I}$.

\item Sean $a,b \en \mathrm{I}$, $a<b$. Se tiene \todo \phantom{}\x \en $[a,b]:f(b) \leq f(x) \leq f(a)$. Luego $f$ es medible y acotada en $[a,b]$, por lo tanto $f$ es integrable en $[a,b]$.\\
\end{enumerate}
\hspace{11cm} c. q. d\\

\underline{Lema 2}\\
Sean $\mathrm{I}=[a,b]$ un intervalo compacto de \R \phantom{} y $f: \mathrm{I} \flecha \R$ una función decreciente en $\mathrm{I}$. Entonces \todo \phantom{} $\upsilon \en \N$ existe una función escalonada $\Phi_\upsilon :\R \flecha \R$, nula fuera de $\mathrm{I}$ tal que:
\begin{enumerate}[1)]
\item $\Phi_\upsilon |_\mathrm{I}$ es decreciente en $\mathrm{I}$.
\item $0 \leq f(x)-\Phi_\upsilon (x) \leq \frac{1}{\upsilon}$ \todo  \phantom{} $\upsilon$ y \todo \phantom{} \x \en $[a,b]$.
\item $\Phi_\upsilon \geq 0$ si $f \geq 0$
\end{enumerate}

\underline{Demostración.} \\
Con las notaciones de la demostración del lema 1 definimos, como allí,$\Phi_\upsilon=:\sum_{k \en \Z}\frac{k}{\upsilon} x_{J_k^{\upsilon}}$. Puesto que \todo \phantom{} \x \en $\mathrm{I}$, $f(b) \leq f(x) \leq f(a)$, el intervalo $J_\upsilon^k$ es vacío si $k>\upsilon f(a)$ y si $k \leq \upsilon f(b)-1$. Así pues, tan solo para un número finito de indices $k$, $x_{J_k^{\upsilon}} \neq 0$. Por consiguiente $\Phi_\upsilon$ es una función escalonada \R \flecha \R . Lo demás sigue de la demostración del lema 1. \\

\textbf{Proposición 5.}\\
\underline{Sea $g:[a,b] \flecha \F$ una función integrable en $[a,b]$.} \\
\underline{Sea $f:[a,b] \flecha \R$ una función decreciente, no negativa en $[a,b]$.}\\
Entonces la función $fg:[a,b] \flecha \F$ es integrable en $[a,b]$ y se tiene:
\begin{equation*}
\underline{| \int_a^b f(x) g(x) dx| | \leq f(a) \stackbin[a \leq x \leq b]{}{M\acute{a}x} | | \int_a^x g || }
\end{equation*}
En el caso particular de ser $\F =3$.\\
$$
\exists \epsilon \en [a,b] \backepsilon \int_a^b f(x) g(x)dx=f(a) \int_a ^{\epsilon}g(x)dx
$$
\underline{Demostración.}\\
La función $fg$ es medible en $[a,b]$ como producto de funciones medibles. Ademas, siendo $|fg| \leq f(a)$, donde el segundo miembro es una función integrable en $[a,b]$, $fg$ es integrabe en $[a,b]$.\\
Definamos la funcion $G$, una integral indefinida de $g$ en $[a,b]$ por: \\
\begin{equation*}
\underline{G(x)=: \int_a^x g \phantom{s} \todo \phantom{s} x\en [a,b]}
\end{equation*}
Sabemos que $G$ es continua en $[a,b]$. Pongamos:\\
\begin{equation*}
K=:\stackbin[a \leq x \leq b] {}{M\acute{a}x} || G(x) ||
\end{equation*}

\begin{enumerate}[a)]
\item Consideremos primero el caso particular de ser $f$ escalonada, además de ser no negativa y decreciente en $[a,b]$.
\item Existe pues una subdivisión $a=a_o < a_1 < \ldots < a_m=b$ del intervalo $[a,b]$ tal que $f$ es de valor constante $d_k$ en todo subintervalo abierto $]a_{k-1},a_k[$, $k=1,\ldots , m$. Por cierto $d_k \leq d_{k-1}$ para $k=2, \ldots,m$. \\
\end{enumerate}
Podemos escribir:\\
$$
\int_a^b fg=\sum_{k=1}^m \int_{a_{k-1}}^{a_k}fg= \sum_{k=1}^m d_k (G(a_k)-G(a_{k-1}))=
$$
$$
=\sum_{k=1}^m d_k G(a_k)-\sum_{k=1}^m d_k G(a_{k-1})=\sum_{k=1}^m d_k G(a_k)- \sum_{k=1}^{m-1}d_{k+1}G(a_k)
$$
o sea, teniendo en cuneta que $G(a_o)=G(a)=0:$
\begin{equation}
\int_a^b fg=d_mG(a_m) + \sum_{k=1}^{m-1}(d_k -d_{k+1})G(a_k)
\end{equation}

Recordando que aquí $d_m \geq 0$ y $d_k -d_{k+1} \geq 0$ para $k=1, \ldots, m-1$, obtenemos de (1.5) por la desigualdad triangular:
\begin{equation}
|| \int_a^b fg || \leq k(d_m +\sum_{k=1}^{m-1} (d_k -a_{k+1}))=kd_1 \leq k f(a)
\end{equation}
como afirmamos.
\begin{enumerate}[I)]
\item Siempre con la hipótesis de ser $f$ escalonada, consideremos el caso particular \F$=$ \R .\\
Introduzcamos los números $M=: \stackbin[a \leq x \leq b]{}{M\acute{a}x} G(x)$ y $m=:\stackbin[a \leq x \leq b]{}{Min G(x)}$ \\
De la formula (1.5), siempre atendiendo a la positividad de los coeficientes $d_m$ y $a_k-a_{k+1}$, se sigue: 
\begin{equation}
md_1 \leq \int_a ^ bfg \leq Md_1
\end{equation}
Ya que $G(a)=0$, se tiene $M \geq 0$ y $m \leq 0$.Como $d_1 \leq f(a)$, la formula (1.6) implica:

\begin{equation}
mf(a) \leq \int_a^b fg \leq Mf(a).
\end{equation}
\item En segundo lugar pasemos al caso general de $f$ decreciente no negativa en $[a,b]$, abandonando la hipótesis de ser $f$ escalonada.
\item Por el lema 2. \todo \phantom{} $\upsilon \en \N$ existe una función escalonada $\Psi_\upsilon: \R \flecha \R$ nula fuera de $[a,b]$ tal que la restricción de $\Psi_\upsilon$ a $[a,b]$, y vale $\Psi_\upsilon \geq 0$, \todo \phantom{} \x \en $[a,b]$.\\
A la función $\Psi_\upsilon$ podemos aplicarle la formula (1.7) de la parte $\mathrm{I})$ obteniendo:
\begin{equation}
|| \int \psi_\upsilon g || \leq K  \Phi_\upsilon (a)
\end{equation}
Para $\upsilon+ $ el segundo miembro de (1.9) converge a $Kf(a)$. Se tiene también $\lim_{v \to +\infty} \Phi_\upsilon g=fg$ en todo punto de $[a,b]$. Además $| \Phi_\upsilon g| =\Phi_\upsilon |g| \leq f |g| \leq f(a) $, donde el último miembro es una función integrable en $[a,b]$.\\
Por el teorema de Lebesgue se puede puede pasar el límite bajo el signo integral para $\upsilon \to +\infty$ en el primer miembro de (1.9). Tomando los limites en (1.9) se consigue:\\
$ || \int_ \exists fg || \leq Kf(a)$, como afirma el enunciado.\\
\item Queda el caso particular \F = \R.
\end{enumerate}
Definamos $\Phi_\upsilon$ como en $\mathrm{II}$ y apliquemos a la función $\Phi_\upsilon$ la fórmula (1.6) de la parte $\mathrm{I}$:
\begin{equation*}
m\Phi_\upsilon (a) \leq \int_a^b \Phi_\upsilon g \leq M \Phi \upsilon (a)
\end{equation*}
Pasando aquí al limite para $\upsilon \to +\infty$ obtenemos:
\begin{equation*}
mf(a) \leq \int_a^b fg  \leq M f(a)
\end{equation*}
Descartando el caso trivial $f(a)=0$, esto se escribe:
\begin{equation*}
m \leq \frac{1}{f(a)} \int_a ^b fg \leq M
\end{equation*}
De ahí, por el teorema de los valores intermedios de Bolzano, aplicado a la función continua $G$ existe $\epsilon \en [a,b]$ tal que
$$
G(\epsilon)=\frac{1}{f(a)} \int_a^b fg, \phantom{s} \int_a ^b fg=f(a) \int_a^\epsilon g
$$
\hspace{11cm} c. q. d \\
He aquí una generalización de la prop. 5. \\

\textbf{Proposición 6.}(\underline{SEGUNDO TEOREMA DEL VALOR MEDIO DE O. RONNEY})\\
\underline{Sea $g:[a,b] \flecha \F$ una función integrable en $[a,b]$.}\\
\underline{Sea $f:[a,b] \flecha \R$ una función monótona en $[a,b]$.}\\
Entonces:
\begin{equation*}
\boxed{|| \int_a ^b fg || \leq (| f(a) | + 2|f(b)|) \stackbin[a \leq x \leq b]P{}{M\acute{a}x} || \int_a ^x g ||}
\end{equation*}
Si $\F=\R$, existe $\epsilon \en [a,b]$ tal que:\\
\begin{equation*}
\boxed{\int_a ^b fg=f(a) \int_a^\epsilon g + f(b) \int_\epsilon^b g}
\end{equation*}
\underline{Demostración.}\\
Siendo las fórmulas por demostrar invariantes bajo el cambio de $f$ en $-f$ podemos sin pérdida de generalidad suponer $f$ decreciente en $[a,b]$. Entonces la función $f-f(b)$  será decreciente y no negativa en$ [a,b]$. Podemos pues aplicarle los resultados de la prop. 5.
\begin{enumerate}[a)]
\item Sea, como antes, $K=\stackbin[a \leq x \leq b]{}{M\acute{a}x} || \int_a^x g||$ 
Por la prop. 5\\
\begin{equation*}
|| \int_a^b (f-f(b) g || \leq (f(a)-f(b))K \leq (|f(a)|+|f(b)|)K
\end{equation*}
De ahí que
\begin{equation*}
|| \int_a^b fg || \leq (|f(a)|+|f(b)|)K +|f(b)| \phantom{s}||\int_{a}^b g|| \leq 
\end{equation*}
$ \leq (|f(a)| +2|f(b)|)K$, como afirmamos.
\item Supongamos ahora $\F=\R$. Por la prop. 5 existe $\epsilon \en [a,b]$ tal que
\end{enumerate}
$$\int_a^b (f-f(b)g)=(f(a)-f(b))\int_a^\epsilon g$$
Es decir:
\begin{equation*}
\int_a^bfg=f(a)=\int_a^\epsilon g+f(b)(\int_a^bg-\int_a^\epsilon g)=f(a)\int_a^\epsilon g+f(b)\int_\epsilon^bg
\end{equation*}
\hspace{11cm} c. q. d.
\subsection{Integrales impropias.}
\underline{Definiciones.}\\
Consideremos un intervalo abierto $]a,b[$ de \R . Aquí permitiremos $a=-\infty$ ó $b=+\infty$.\\
Sea $f:]a,b[ \flecha \F$.\\
\begin{enumerate}[a)]
\item \underline{Supongamos $f$ integrable en todo intervalo $]a,\beta[$, donde $a<\beta <b$.}\\
Si existe $ \stackbin[\beta \en ]a,b[]{}{\lim_{\beta \to b}} \int_a ^\beta f$, este límite se designa por $\int_a ^{\to b}f$. 
\item \underline{Supongamos $f$ integrable en todo intervalo $]\alpha,\beta[$, donde $a< \alpha <b$.}\\
Si existe $ \stackbin[\beta \en ]a,b[]{}{\lim_{\alpha \to b}}\int_\alpha ^\beta f$, este límite se designa por $\int_{\to a}^b f$.\\
\item \underline{Supongamos $f$ integrable en todo subintervalo compacto de $]a,b[$.}\\
Sea $x_o\en ]a,b[$. La existencia de $\int_{\to a}^{x_o}f$ y la de $\int_{x_o}^{\to b}f$ es, cada una, independiente de la elección del punto $x_o$. Si existen $\int_{\to a}^{x_o}f$ y $\int_{x_o}^{\to b}f$ la suma $\int_{a}^{x_o}f+\int_{x_o}^{b}f$ es también independiente de la elección de $x_o$ y se designa por $\int_{\to a}^{\to b}f$.\\
$\int_{\to a}^{b}f,\int_{a}^{\to b}f,\int_{\to a}^{\to b}f$, se llaman INTEGRALES IMPROPIAS.
\end{enumerate}

En vez de decir que una integral impropia \textit{existe}, se suele decir por tradición que dicha INTEGRAL IMPROPIA es CONVERGENTE.\\
Puesto que \F \phantom{} es un espacio métrico completo, podemos obtener un criterio de "convergencia" de integrales impropias, aplicando el criterio de Cauchy para la existencia de un límite (prop. 16 del cap. $\mathrm{I})$:\\ \\
\textbf{Proposición 7.} \underline{CRITERIO DE CAUCHY PARA LA CONVERGENCIA DE} \\
\underline{INTEGRALES IMPROPIAS}\\
Sea $f:]a,b[ \flecha \F$ una función integrable en todo intervalo $]a,\beta[$ donde $a<\beta < b$.
\begin{enumerate}[I)]
\item \underline{Si $b < +\infty$, la integral impropia $\int_a^{\to b}f$ es convergente, si y solo si: \todo $\epsilon>0$}
\underline{ $\exists \delta>0$ tal que $x_1,x_2 \en ]b-\delta,b[\cap ]a,b[\Rightarrow|| \int_{x_1}^{x_2}f||< \epsilon$}
\item \underline{Si $b=+\infty$ la integral impropia $\int_a^{+\infty}f$ es convergente, si y solo si:  \todo $\epsilon>0$}
\underline{ $\exists R>a$ tal que $x_1>R,x_2>R \Rightarrow|| \int_{x_1}^{x_2}f||< \epsilon$}
\end{enumerate}
\underline{Demostración.}\\
Definamos $G(x)=:\int_{a}^xf$, \todo \phantom{} \x \en $]a,b[$. Decir que la integral impropia $\int_a^{\to b}f$ es convergente, es decir que existe $ \stackbin[x \en ]a,b[]{}{\lim_{x\to b}G(x)}$.\\
Supongamos $b<+\infty$. Por el criterio de Cauchy, el límite considerado existe, si y sólo si: \todo $\epsilon>0 \exists>0$ tal que $x_1,x_2 \en ]b-\delta,b[ \cap ]a,b[$ $\Rightarrow || G(x_2)-G(x_1)|| < \exists$. Pero $G(x_2)-G(x_1)=\int_{x_1}^{x_2}f$. La demostración en el caso $b=+\infty$ es análoga.\\
El lector enunciará y demostrará condiciones semejantes para la convergencia de la integral impropia $\int_{\to a}^b f$.\\
De aquí en adelante, hasta el fin del capítulo, salvo en ejemplos, nos limitaremos a considerar las integrales impropias $\int_a^{\to b}$ con $b<+\infty$, si esta condición es pertinente. Dejemos al lector la tarea de enunciar y demostrar los resultados en los demás casos.\\ \\
\textbf{Proposición 8 y definición.}\\
Sea $f:]a,b[ \flecha \F$ una función integrable en todo intervalo $]a,\beta[$, donde $a<A<b$.\\
Si la integral impropia $\int_a^{\to b} |f|$ es convergente, la integral impropia $\int_a^{\to b}f$ es convergente. En tal caso se dice que la INTEGRAL IMPROPIA $\int_a^{\to b}f$ es ABSOLUTAMENTE CONVERGENTE.\\ \\
\underline{Demostración.}\\
Supongamos que la integral impropia $\int_a^{\to b}|f|$ es convergente. En virtud de la prop. 7 esto equivale a decir que: 
\todo \phantom{} $\epsilon >0$ $\exists \delta>0$ tal que $x_1,x_2 \en ]b-\delta,b [ \cap ]a,b[$, $x_1 <x_2 \Rightarrow \int_{x_1}^{x^2}|f| < \exists$.\\
Puesto que $|| \int_{x_1}^{x_2}f || \leq \int_{x_1}^{x_2}|f|$, se tiene a mayor abundamiento, la implicación:\\
 $x_1,x_2 \en ]b-\delta,b [ \cap ]a,b[ \Rightarrow || \int_{x_1}^{x_2}f || < \epsilon$. Así pues la integral impropia $\int_a^{\to b}f$ es convergente por el criterio de Cauchy (prop. 7).\\
\textbf{Proposición.}\\
Sea $f:]a,b[ \flecha \F$ una función integrable en todo intervalo $]a,\beta [$, donde $a<\beta <b$. Para la integral impropia $\int_a^{\to b}f$ sea absolutamente convergente, es necesario y suficiente que la función $f$ sea integrable en $]a.b[$ y entonces se tiene:\\
$$
\underline{\int_a^{\to b}f=\int_a^b f.}
$$
\underline{Demostración.}\\
\begin{enumerate}
\item Afirmamos que $f$ es medible en $]a,b[$. En efecto sea $\lbrace \beta_\upsilon \rbrace$ una sucesión de puntos de $]a,b[$ que converge a $b$. Se tiene $\tilde{f}x_{]a,b[}=\lim_{\upsilon \to +\infty}\tilde{f}x_{]a,\beta_\upsilon[}$ en todo punto de \R \phantom{} y, puesto que $\tilde{f}x_{]a,\beta_\upsilon[}$ es una función integrable, a mayor abundamiento medible, $\tilde{f}x_{]a,b[}$ es medible, o sea $f$ es medible en $]a,b[$.\\
Por consiguiente $f$ será integrable en $]a,b[$ si y sólo si $|f|$ es integrable en $]a,b[$.
\item Supongamos $f$ integrable en $]a,b[$, equivalentemente $|f$ integrable en $]a,b[$. Sea  $\lbrace \beta_\upsilon \rbrace$ una sucesión de puntos de $]a,b[$ que converge a $b$.\\
Se tiene \todo $\upsilon$:\\
\begin{equation}
\int_a^{\beta_\upsilon}|f|=\int_\R |\tilde{f}|x_{]a,\beta_\upsilon[}
\end{equation}

Además $\lim_{\upsilon \to + \infty} \tilde{f}|x_{]a,\beta_\upsilon[}=|\tilde{f}|x_{]a,\beta_\upsilon[}$ en todo punto de \R \phantom{} y $|\tilde{f}| x_{]a,\beta_\upsilon[} \leq | \tilde{f}|=$ función integrable independiente de $\upsilon$.\\
Por el teorema de Lebesgue se puede pasar al límite para $\upsilon \to +\infty$ bajo el signo integral en el segundo miembro de (1.10). Se consigue:\\
\begin{equation*}
\lim_{\upsilon \to +\infty}\int_{a}^{\beta_\upsilon}|f|=\int_\R |\tilde{f}| x_{]a,b[}.
\end{equation*}
Así pues existe la integral impropia $\int_a^{\to b}|f|$ y es igual a $\int_a^b |f|$. Esto quiere decir que la integral impropia $\int_a^{\to b}f$ es absolutamente convergente.\\
Con la misma significación de $\lbrace \beta_\upsilon \rbrace$ escribamos ahora:\\
\begin{equation}
\int_a^{\beta_\upsilon}f=\int_\R \tilde{f}x_{]a,\beta_\upsilon[}
\end{equation}
Se tiene $\lim_{\upsilon \to +\infty}\tilde{f}x_{]a,\beta_\upsilon[}=\tilde{f}x_{]a,b[}$ en todo punto de \R. \phantom{} y $|\tilde{f}x_{]a,\beta_\upsilon[}| \leq |\tilde{f}|=$función integrable independiente de $\upsilon$. \\
Por el teorema de Lebesgue se puede pasar al límite para $\upsilon \to +\infty$ en el segundo miembro de (1.11). Se consigue:
$$
\lim_{\upsilon \to +\infty}\int_a^{\beta_\upsilon}f=\int_\R \tilde{f}x_{]a,b[}
$$
$$
\int_a^{\to b}f=\int_a^b f.
$$
\item Recíprocamente, supongamos la integral impropia $\int_a^{\to b}f$ absolutamente convergente. Sea $\lbrace \beta_\upsilon$ una sucesión creciente de puntos de $]a,b[$ tal que $\lim_{\upsilon \to +\infty}\beta_\upsilon=b$.\\
Se tiene $\int_a^{\to b}f=\lim_{\upsilon \to +\infty} \int_a^{\beta_\upsilon}|f|=\lim_{\upsilon \to +\infty}\int_R |\tilde{f}|x_{]a,\beta_\upsilon[}$.\\
Ahora bien $\lbrace  |\tilde{f}|x_{]a,\beta_\upsilon} \rbrace$ es una sucesión creciente de funciones integrables que converge a  $|\tilde{f}|x_{]a,\beta_\upsilon}$ en todo punto de \R .\\
Por el teorema de B. Levi la función límite $ |\tilde{f}|x_{]a,b[}$ es integrable. En otras palabras $|f|$ es integrable en $]a,b[$. Por lo tanto en $f$ integrable en $]a,b[$.
\hspace{9cm} c. q. d
\end{enumerate} 
La prop. 9 hace patente que, en el marco de la teoría actual de integración, es inútil considerar aparte las integrales impropias absolutamente convergentes pues son iguales a integrales \textit{propias} y en su escritura se puede suprimir la advertidora flecha. Gracias a esta circunstancia podemos aplicar a las integrales absolutamente  convergentes, si preocupación adicional, los resultados válidos para integrales \textit{propias}, p. ej. los teoremas de convergencia y los teoremas relativos a las funciones definidas por integrales. Todo criterio de integrabilidad es al mismo tiempo un criterio de convergencia absoluta de integrales impropias.\\
Sin embargo, como se verá en ejemplos al fin de esta sección, existen integrales impropias convergentes que no son absolutamente convergentes. Por la prop. 9 las integrales propias correspondientes no existen. Dichas integrales impropias constituyen un fenómeno peculiar de \R \phantom{} que no parece tener análogo interesante en \Rn \phantom{} para $n>1$. A ellas no se pueden aplicar sin preocupaciones los teoremas sobre integrales \textit{genuinas} y hay que estudiarlas aparte.\\
Debemos buscar antes de todo criterios de convergencia de integrales impropias que se apliquen al caso de no haber convergencia absoluta. \\ \\

\textbf{Proposición 10.}\underline{(Criterios de ABEL para la convergencia de integrales impropias.)}\\
\underline{Sea $f:[a,b[ \to \R$ una función monótona en $[a,b[$.}\\
\underline{Sea $g:[a,b[ \to \F$ una función integrable en todo intervalo $[a,\beta]$}\\
\underline{donde $a<\beta <b$. Sea $G(x)=: \int_a^x g$, \todo \x \en $[a,b[$.}\\
La integral impropia $\int_a^{\to b}fg$ será convergente en cada uno de los casos siguientes:\\
\begin{enumerate}[1)]
\item $\lim_{x \to b}f(x)=0$ y $G$ es acotada en $[a,b[$.
\item $f$ es acotada en $[a,b[$ y la integral impropia $\int_a^{\to  b}g$ es convergente.\\
\end{enumerate}
\underline{Demostración.}\\

Sean $x_1,x_2$ números tales que $a<x_1<x_2<b$. Por el teorema de Bonnet se tiene la desigualdad: 
\begin{equation}
|| \int_{x_1}^{x_2}fg || \leq (|f(x_1)|+2|f(x_2)|) \stackbin[x_1 \leq x \leq x_2]{}{M\acute{a}x}|| \int_{x_1}^x g ||
\end{equation}

\begin{enumerate}
\item Supongamos que se cumplen las hipótesis 1).  \\
Sea $K=\stackbin[\x \en [a,b[]{}{Sup}||G(x)||.$ Se tendrá para todo $\x \en [x_1,x_2]$:
\begin{equation}
|| \int_{x_1}^x g|| = || G(x)-G(x_1) || \leq || G(x) || + || G(x) || \leq 2K.
\end{equation}
Sea $\epsilon >0$ dado arbitrario. Por hipótesis $\exists \delta>0$ tal que $x \en ]b-\delta,b[ \cap ]a,b[ \Rightarrow |f(x)| <\epsilon$.\\
En virtud de la mayoracion (1.12) se tendrá la implicación:\\
$$
x_1,x_2\en ]b-\delta [ \cap ]a,b[ \Rightarrow || \int_{x_1}^{x_2}fg || < 6K\epsilon
$$
Luego la integral impropia $\int_a^{\to b}fg $ es convergente por el criterio de Cauchy.\\

\item Supongamos que se cumplen las hipótesis 2). \\
Sea $M=:\stackbin[\x \en [a,b[]{}{Sup}|f(x)|$. Sea $\epsilon>0$ dado arbitrario.\\
Puesto que la integral impropia $\int_a^{\to b}g$ es convergente:\\

$\exists \delta >0$ tal que $y_1,y_2 \en ]b-\delta [ \cap ]a,b[ \Rightarrow || \int_{y_1}^{y_2}g|| < \epsilon$
\end{enumerate}
Al tomar $x_1,x_2 \en  ]b-\delta [ \cap ]a,b[$, tendremos $\stackbin[x_1 \leq x \leq x_2]{}{M\acute{a}x} || \int_{x_1}^x g || \leq \epsilon$ \\
luego, por la mayoracion (1): \\
$$
|| \int_{x_1}^{x_2}fg|| \leq 3M \epsilon
$$
De nuevo la integral impropia $\int_a^{\to b}fg$ es convergente  por el criterio de Cauchy. \\
\phantom{llenado doble y múltiple ya que necesito espacio para escribir:} c. q. d \\
\underline{Ejemplos.}\\
Las integrales $\int_1^{\to +\infty}\frac{\sen (x)}{x^\alpha}dx$ y $\int_1^{\to +\infty} \frac{\cos (x)}{x^\alpha}dx$, $\alpha \en \R$. \\ \\
\begin{enumerate}[1)]
\item \underline{Caso $\alpha>1$.}\\
Escribamos $| \frac{\sen (x)}{x^\alpha}| < \frac{1}{x^\alpha}$ y $|\frac{\cos (x)}{x^{\alpha}}| \leq \frac{1}{x^\alpha}$.\\
Por la prop. 29. del cap. V, la función \x \flecha $\frac{1}{x^\alpha}$ es integrable en $[1,+\infty[$ en nuestro caso. Luego también son integrables en $[1,+\infty[$ las funciones \x \flecha $\frac{\sen (x)}{x^\alpha}$ y \x \flecha $\frac{\cos (x)}{x^\alpha}$. Dicho de otro modo:\\
\underline{Las integrales impropias $\int_1^{\to +\infty}\frac{\sen (x)}{x^\alpha}dx$ y $\int_1^{\to + \infty}\frac{\cos (x)}{x^\alpha}dx$ son}\\
\underline{absolutamente convergentes.}\\
\item Afirmamos que $\int_1^{\to +\infty}\frac{|\sen (x)|}{x}dx=+\infty$.\\
En otras palabras la función \x \flecha $\frac{\sen (x)}{x}$ no es integrable en $[1,+\infty[$.\\
\todo $n \en \N$ se tiene:\\
\begin{equation*}
\int_1^{+\infty}\frac{|\sen (x)|}{x}dx \geq \int_\pi^{(n+1)\pi}\frac{|\sen (x)|}{x}dx=\sum_{k=1}^n\int_{k\pi}^{(n+1)\pi}\frac{|\sen (x)|}{x}dx
\end{equation*}
\begin{equation*}
\geq \sum_{k=1}^n \frac{1}{(k+1)\pi}\int_0^\pi \sen (x)dx=\frac{2}{\pi} \sum_{k=1}^n \frac{1}{
k+1}
\end{equation*}
Pero sabemos que la serie de término general $\frac{1}{k+1}$ diverge, de donde la conclusión anunciada. \\
De ahí deducimos que si $\alpha \leq 1$:\\
$\int_1^{+\infty} \frac{|\sen (x)|}{x^\alpha}dx \geq \int_1^{+\infty}\frac{|\sen x|}{x}dx= +\infty$
\\
Análogamente se prueba que en este caso:\\
\begin{equation*}
\int_1^{+\infty}\frac{|\cos (x)}{x^\alpha}dx=+\infty
\end{equation*}
Así pues, si $\alpha \leq 1$ las funciones \x \flecha $\frac{\sen (x)}{x^\alpha}$ y $x \flecha \frac{\cos (x)}{x^\alpha}$ no son integrables en $[1,+\infty[$.\\
\item \underline{Firmamos que para $0< \alpha \leq 1$ las integrales impropias
$\int_1^{\to +\infty}\frac{\sen (x)}{x^\alpha}$} y \underline{$\int_1^{\to +\infty}\frac{\cos (x)}{x^\alpha}$ son convergentes.}\\
Basta observar que la función $x \flecha \frac{1}{x^\alpha}$ es decreciente en $[1,+\infty[$: $\lim_{x \to +\infty}\frac{1}{x^\alpha}=0$ y las funciones $x \flecha \int_1^x \sen (t)dt=cos (1)-cos(x)$ y \x \flecha $\int_1^x \cos (t)dt=\sen x - \sen 1$ son acotadas en $[1,+\infty[$.\\
En virtud del primer criterio de Abel las integrales impropias $\int_1^{\to +\infty}\frac{\sen (x)}{x^\alpha}$ y $\int_1^{\to +\infty}\frac{\cos (x)}{x^\alpha}$ son convergentes. \\
\item Afirmamos que para $\alpha \leq 0$ las integrales impropias consideradas "divergen" (es decir no son convergentes). \\
Lo afirmado equivale a decir que las integrales $\int_1^{\to +\infty}x^\beta \sen (x) dx$ y $\int_1^{\to +\infty}x^\beta \cos (x) dx$ divergen si $\beta \geq 0$.\\
Mostremoslo p. ej. para la primera integral. Sea $R>1$. La función $R \flecha \int_1^{R} sen(x)dx=\cos (1)-\cos(R)$ no tiende a ningún límite para $R \to +\infty$. Luego la integral impropia $\int_1^{\to +\infty} \sen (x)dx$ diverge. \\
Sea ahora $\beta >0$ y supongamos que la integral impropia $\int_1^{+\infty} x^\beta \sen (x) dx$ convergiera. Puesto que la función \x \flecha $\frac{1}{x^\beta}$ es monótona y acotada en $[1,+\infty[$, por el segundo criterio de Abel seria convergente la integral impropia $\int_1^{\to +\infty}\frac{1}{x^\beta}(x^\beta \sen (x))dx=\int_1^{\to +\infty}\sen (x)dx$, lo que , como acabamos de mostrar es falso.
\\
Así pues la integral impropia $\int_1^{\to +\infty} x^\beta \sen (x)dx$ diverge.\\
\underline{Este ejemplo muestra como se puede empleados los criterios de Abel}
\underline{para demostrar la divergencia de ciertas integrales impropias.}
\item Consideremos las llamadas INTEGRALES IMPROPIAS de FRESNEL:\\
$$\int_0^{+\infty}\sen(x^2)dx \phantom{y}y \phantom{s}  \int_1^{\to +\infty}\cos (x^2)dx.$$\\
Por el cambio de variables $t=x^2, x=\sqrt{t}$, hallamos:\\
\begin{equation}
\int_0^R \sen(x^2)dx=\frac{1}{2}\int_0^{R^2}\frac{\sen (t)}{\sqrt{t}}dt.
\end{equation}
Para $R \flecha +\infty$, el segundo miembro de $(1)$ tiende a un límite en virtud de la parte 3), luego la integral impropia \\
\underline{$\int_0^{+\infty}\sen(x^2)dx$ es convergente.} Asimismo, se ve que la integral impropia \\
\underline{$\int_0^{+\infty}\cos(x^2)dx$ es convergente.} Por cierto la relación (1.14) suministra:\\
$$
\int_0^{+\infty}\sen (x^2)dx=\frac{1}{2}\int_0^{\to +\infty}\frac{\sen (t)}{\sqrt{t}dt}.
$$
Sin embargo, \underline{dichas integrales no son absolutamente convergentes.}\\
Pues si p. ej. la función \x \flecha $\sen x^2$ fuese integrable en $[0,+\infty[$, en virtud del teorema de cambio de variables también la función $t\flecha \frac{\sen (t)}{\sqrt{t}}$ sería integrable en $[0,+\infty[$ y esto no es el caso conforme a lo establecido en 2).\\
\item Consideremos p. ej. la integral impropia:
\begin{equation*}
\underline{\int_1^{+\infty}\arctan (x)\frac{\sen (x)}{x^\alpha}dx, 0 < \alpha \leq 1}
\end{equation*}
La función $\arctan$ es creciente y acotada en $[1,+\infty[$, la integral impropia $\int_1^{+\infty}\frac{\sen (x)}{x^\alpha}dx$ es convergente en virtud de la parte 3), luego la integral impropia $\int_1^{+\infty}  \arctan (x) \frac{\sen (x)}{x^\alpha}dx$ es convergente por el segundo criterio de Abel.\\
Por otra parte:\\
$$
\int_1^{+\infty}\arctan (x)\frac{|\sen (x)|}{x^\alpha}dx \geq \frac{\pi}{4}\int_1^{+\infty}\frac{|\sen (x)|}{x^\alpha}dx=+\infty
$$
en virtud de la parte 2). Luego \underline{nuestra integral es absolutamente}\\
\underline{convergente.}
\subsection{Funciones definidas por integral impropias.}
\underline{(Convergencia uniforme de integrales impropias dependientes de un} \\
\underline{parámetro).}\\
Sea \Lam \phantom{} un conjunto y sea $f:]a,b[ \times \Lam \flecha \F$. Se supone que \todo \lam \en \Lam \phantom{} la función \x \flecha $f(x,\lam)$ es integrable en todo intervalo $]a,\beta[$, donde $a <\beta <b$. Se dice que la INTEGRAL IMPROPIA $\int_a^{\to b }f(x,\lam)dx$ es UNIFORMEMENTE CONVERGENTE si existe $\stackbin[\beta \en ]a,b[]{}{\lim_{\beta \to b}}\int_a^{\beta}f(x,\lam)dx$ uniformemente en \Lam .\\
Explicitamente: \underline{Para todo $\epsilon>0$ existe $\delta >0$ (independiente de \lam)}\\
\underline{tal que $\beta \en ]b-\delta,b[ \cap ]a,b[ \Rightarrow || \int_\beta^{\to b}f(x,\lam)dx|| < \epsilon$ , $\todo$ $\lam \en \Lam$}
\end{enumerate} 

\textbf{Proposición 11. } \underline{(Criterio de Cauchy para la convergencia uniforme de integrales impropias).}\\
Sea $f$ como en la definición precedente. Para que la integral impropia $\int_{a}^{\to b} f(x,\lam)dx$ sea uniformemente convergente es necesario y suficiente que para toda $\epsilon>0$ exista $\delta >0$ \underline{independiente de $\lam$} tal que:\\
$$
\underline{x_1,x_2 \en ]b-\delta,b[ \cap ]a,b[ \Rightarrow || \int_{x_1}^{x_2}f(x,\lam)dx||< \epsilon, \phantom{} \todo \lam \en \Lam}
$$
\underline{Demostración.}\\
\begin{enumerate}[a)]
\item Supongamos que la integral impropia $\int_a^{\to b}f(x,\lam)dx$ es uniformemente convergente. En virtud de la definición:\\
$\exists \delta >0$ (independiente de \lam) tal que $\beta \en ]b-\delta,b[ \cap ]a,b[ \Rightarrow \int_\beta^{\to b}f(x,\lam)dx|| < \frac{\epsilon}{2}, \todo \lam \in \Lam.$\\
Si $x_1,x_2 \en ]b-\delta,b[ \cap ]a,b[$, se sigue:\\
$$\int_{x_1}^{x_2}f(x,\lam)dx= || \int_{x_1}^{\to b}f(x,\lam)dx-\int_{x_2}^{\to b}f(x,\lam)dx|| \leq \int_{x_1}^{\to b}f(x,\lam)dx ||$$
$$+|| \int_{x_2}^{\to b}f(x,\lam)dx || < \frac{\epsilon}{2}+\frac{\epsilon}{2}=\epsilon$$
Luego se cumple la condición de Cauchy. \\
\item Supongamos que se cumple la condición de Cauchy. Sea $\epsilon>0$ dado. Por hipótesis existe $\delta>0$ (independiente de \lam) tal que si $x_1,x_2 \en ]b-\delta,b[ \cap ]a,b[$, entonces:
\end{enumerate}
\begin{equation}
|| \int_a^{x_1}f(x,\lam)dx-\int_a^{x_2}f(x,\lam)dx || < \epsilon, \todo \lam \en \Lam
\end{equation}
Puesto que \F \phantom{} es un espacio métrico completo, en virtud de la prop. 16 del cap. $\mathrm{I}$ esto implica en particular la existencia del límite en $b$ de la función $\beta \flecha \int_a^\beta f(x,\lam)dx$. Existe pues la integral impropia $\int_a^{\to b} f(x,\lam)dx$.\\
Pasemos al límite en (1.15) para $x_2 \to b^{-}$, manteniendo $x_1$ fijo. Se encuentra:\\
$$
|| \int_a^{x_1}f(x,\lam)dx-\int_a^{\to b}f(x,\lam)dx || \leq \epsilon
$$
Como esto se verifica \todo \phantom{} $x_1>\delta$ hemos demostrado la integral impropia $\int_a^{\to b}f(x,\lam)dx$ es uniformemente convergente.\\ \\

\textbf{Proposición 12.}\underline{(Integración impropia de sucesiones de funciones).}\\
Sea $\lbrace f_\upsilon \rbrace$ una sucesión de funciones: \\
\underline{$f_\upsilon:]a,b[ \flecha \F$.}\\
Se supone:\\

\begin{enumerate}[1)]
\item \underline{$f_\upsilon$ es integrable en $]a,\beta[$ $\todo \upsilon \en \N$ y $\todo \beta ]a,b[$.}
\item \underline{$\lim_{\upsilon \to +\infty}f_\upsilon=f$ c. t. p en $]a,b[$/}
\item \underline{$f$ es integrable en $]a,\beta[$ $\todo \en ]a,b[$ y se verifica:}\\
$$
\lim_{\upsilon \to +\infty}\int_a^{\beta}f_\upsilon=\int_a^\beta f, \phantom{s} \todo \phantom{s} \beta \en ]a,b[.
$$
Esta condición se cumplirá en particular si para todo $\beta \en ]a,b[$ existe una función $g_{\beta}:]a,b[ \flecha \R$, integrable en $]a,\beta[$ tal que $|| f_\upsilon(x)|| \leq g_{\beta}(x)$ $\todo$ $\upsilon \en N$ y $\todo$ \x \en $]a,\beta$\\
También se cumplirá si, siendo finito, $\lbrace f_\upsilon\rbrace$ converge a $f$uniformemente en $]a,\beta[$ $\todo$ $\beta \en ]a,b[$.
\item \underline{La integral impropia $\int_a^{\to b} f_\upsilon (x)dx$ es uniformemente convergente con respecto a $\upsilon\en \N$.}\\
Entonces la integral impropia $\int_a^{\to b} f (x)dx$ es convergente y $\int_a^{\to b} f (x)dx=\lim_{\upsilon \to +\infty}\int_a^{\to b}f_\upsilon (x)dx$
\end{enumerate}
\underline{Demostración.}
Pongamos $\Phi(\beta,\upsilon)=: \int_a^\beta f_\upsilon$ $\todo \beta \en ]a,b[$ y $\todo \upsilon \en \N$.\\

La hipótesis 3) significa: 
\begin{equation}
\lim_{\upsilon \to +\infty}\Phi (\beta,\upsilon)=\int_a^\beta f, \phantom{s} \todo \beta \en ]a,b[
\end{equation}
La hipótesis 4) significa:
\begin{equation}
\stackbin[\beta \en ]a,b[]{}{\lim_{\beta \to b}}\Phi (\beta, \upsilon)=\int_a^{\to b}f_\upsilon(x)dx
\end{equation}
uniformemente con respecto a $\upsilon$.
Como \F \phantom{} es un espacio métrico completo, las relaciones (1.16)y (1.17) implican en virtud de la prop. 17 del cap. $\mathrm{I}$ que existen los límites $\stackbin[\beta \en ]a,b[]{}{\lim_{\beta \to b}}\int_a^\beta f=\int_a^{\to b}f$, $\lim_{\upsilon \to +\infty}\int_a^{\to b}f_\upsilon(x)dx$ y que dichos límites son iguales. Pero esto es precisamente lo que afirma la proposición. \\ \\
\textbf{Proposición 13.}\underline{(Continuidad de una función definida por una integral impropia).}\\
Sea \Lam \phantom{} un espacio métrico. Sea $f:]a,b[ \times \Lam \flecha \F$.\\
Se supone:\\
\begin{enumerate}[1)]
\item La función \x \flecha \fxla \phantom{} es integrable en $]a,\beta[$ $\todo \lam \en \Lam$ y $\todo \beta \en ]a,b[$.\\
\item Para así todo \x \en $]a,b[$ la aplicación \lam \flecha \fxla \phantom{} es continua en un punto $\mu \en \Lam$.
\item \todo $\beta  \en ]a,b[$ la aplicación \lam \flecha $\int _a^{\beta} \fxla dx$ es continua en el punto $\mu$.\\
Esta condición se cumple en particular si $\todo \beta \en ]a,b[$ existe una función $g_\beta:]a,b[ \flecha \R$, integrable en $]a,\beta[$ tal que para casi todo $x \en ]a,\beta[$ y $\todo \lam \en \Lam$ se tiene $|| \fxla || \leq g_\beta (x)$. 
\item La integral impropia $\int_a^{\to b}\fxla dx$ es uniformemente convergente en \Lam .\\
\end{enumerate}
Definamos la función $\Phi: \Lam \flecha \F$ por:
\begin{equation*}
\boxed{\Phi (\lam)=\int_a^{\to b}\fxla dx, \phantom{s} \todo \lam \en }
\end{equation*}
\underline{Entonces $\Phi$ es continua en el punto $\mu$.}\\
\underline{Demostración.}\\
Sea $\lbrace \lam_\upsilon \rbrace$ una sucesión de puntos de \Lam \phantom{} tal que $\lim_{\upsilon \to +\infty}\lam_\upsilon=\mu$\\
Debemos mostrar que $\lim_{\upsilon \to +\infty} \Phi(\lam_\upsilon)=\Phi (\mu)$.\\
Pongamos: $f_\upsilon (x)=: f(x,\lam_\upsilon) \phantom{s} \todo \upsilon \en \N$ y $\todo x \en ]a,b[$.\\
Por hipótesis 2)  : $\lim_{\upsilon \to +\infty}f_\upsilon (x)=f(x,\mu)$ para casi todo \x \en $]a,b[$. \\
Por hipótesis 3) : $\lim_{\upsilon \to +\infty}\int_{a}^\beta f_\upsilon =\int_a^\beta f(x,\mu)dx, \phantom{} \todo \beta ]a,b[.$ \\
Por hipótesis 4): la integral impropia $\int_a^{\to b}f_\upsilon$ es uniformemente convergente con respecto $\upsilon$.\\
De estos resultados se sigue por la prop. 12 que\\
$$\int_a^{\to b}  f(x,\mu)dx=\lim_{\upsilon \to +\infty}\int_a^{\to b}f_\upsilon$$
Es decir
$$
\Phi (\mu)=\lim_{\upsilon \to + \infty}\Phi (\lam_\upsilon)
$$
\phantom{separado para que funcione bine jajajajaja xd xd} c. q. d
\\ \\
\underline{Integración de funciones definidas por integrales impropias.}\\
\textbf{Proposición 14.} \\
Sea $f:]a,b[ \times \Rn \to flecha$.\\
Se supone:
\begin{enumerate}[i)]
\item \todo $\beta \en ]a,b[$, $f$ es integrable en $]a,\beta[ \times \Rn$.
\item Para casi todo $\lam \en \Rn$, existe la integral impropia $\int_a^{\to b}\fxla dx$.
\item Existe $\beta_o \en ]a,b[$ y $h\en \mathcal{L}_1(\Rn,\R)$ tal que \\
\underline{$||\int_a^\beta \fxla dx || \leq h(\lam)$ para casi todo \lam \en \Rn \phantom{} y \todo $\beta \en [\beta_o,b[$.}
\end{enumerate}
Entonces\\
\begin{equation*}
\boxed{\int_{\Rn}d\lam \int_a^{\to b}\fxla dx=\int_a^{\to b}dx\int_{\Rn}\fxla dx}
\end{equation*}
\underline{Comentario}\\
Sea $\lbrace \beta_\upsilon \rbrace$ una sucesión de puntos 
de $]a,b[$ tal que $\lim_{\upsilon \to +\infty}\beta_\upsilon=b$. Por la hipótesis $\mathrm{I}$ y el teorema de Fubini existe un conjunto despreciable $Z_\upsilon$ de \Rn \phantom{} tal que si \lam \en \Rn$-Z_\upsilon$, la función \x \flecha \fxla \phantom{} es integrable en $]a,\beta_\upsilon$. Sea $Z=:\cup_{v=1}^{+\infty}Z_\upsilon$. Si \lam \en \Rn $- Z_\upsilon$ , la función \x \flecha \fxla \phantom{} es integrable en $]a,\beta[$ $\todo \beta \en ]a,b[$. Las hipótesis $\mathrm{II}$ y $\mathrm{III}$ tienen pues sentido.
También por el teorema de Fubini, \todo $\upsilon$ existe un conjunto despreciable $N_\upsilon$ , en $]a,\beta_\upsilon$ tal que si \x \en $]a,\beta_\upsilon[-N_\upsilon$ la función \lam \flecha \fxla \phantom{} es integrable en \Rn . Así pues dicha función es integrable en \Rn \phantom{} para casi todo \x \en $]a,b[$.\\ \\
\underline{Demostración.}\\
Sea $\lbrace \beta_\upsilon \rbrace$ una sucesión de puntos de $]a,b[$ tal que $\lim_{\upsilon \to +\infty}\beta_\upsilon =b$. Tenemos \todo $\upsilon$:
\begin{equation}
\int_{\Rn}d\lam \int_a^{\beta_\upsilon}\fxla dx= \int_a^{\beta_\upsilon}dx \int_{\Rn}\fxla dx.
\end{equation}
siendo ambas integrales reiteradas iguales a la integral $\int_{]a,\beta_\upsilon[ \times \Rn}\fxla dx d\lam$ en virtud del teorema de Fubini.\\
Pongamos $g_\upsilon(\lam)=: \int_a^{\beta_\upsilon} \fxla dx$. \\
Las funciones $g_\upsilon$ están definidas c. t. p. en \Rn \phantom{} y son integrables en \Rn . Además, por hipótesis $\mathrm{II}$):
$$
\lim_{\upsilon \to +\infty}g_\upsilon (\lam) =\int_a^{\to b} \fxla dx
$$ 
para casi todo \lam \en \Rn .\\
También por hipótesis $\mathrm{III}$ si $\upsilon$ es lo suficientemente grande para que $\beta_\upsilon \en [\beta_o,b[$ se tiene:
$|| g_\upsilon (\lam)|| \leq h(\lam)$ donde $h$ es una función integrable \Rn \flecha \R .\\
Por el teorema de Lebesgue la función límite \lam \flecha $\int_a^{\to b}\fxla d\lam$ es integrable en \Rn \phantom{} y el primer miembro de (1.18) converge para $\upsilon \to +\infty$ a: $\int_{\Rn}d\lam \int_a^{\to b}\fxla dx$. Esto implica que el segundo miembro de (1.18) también converge y, por definición de una integral impropia se tiene:\\
\begin{equation*}
\int_{\Rn}d\lam \int_a^{\to b}\fxla dx=\int_a^{\to b}dx \int_{\Rn}\fxla d\lam
\end{equation*}
\phantom{separado para que funcione bine jajajajajaja xd xd x} c. q. d \\
Si restringimos el recorrido de $\lam$ a un subconjunto integrable de \Rn , podemos acudir, en vez del teorema de Lebesgue, a la convergencia uniforme:\\ \\
\textbf{Proposición 15.}\\
Sea $S$ un subconjunto integrable de \Rn . Sea $f:]a,b[ \times S \flecha \F$. \\
Se supone:
\begin{enumerate}[1)]
\item \underline{\todo $\beta \en ]a,b[$ $f$ es integrable en $]a,\beta[ \times S$.}
\item \underline{\todo \lam \en $S$ y $\todo \beta \en ]a,b[$ existe la integral $\int_a^\beta \fxla dx$.}
\item \underline{La integral impropia $\int_a^{\to b}\fxla dx$ es uniformemente convergente en $S$.}
\end{enumerate}
Entonces 
\begin{equation*}
\boxed{\int_S d\lam \int_a^{\to b}\fxla dx=\int_a^{\to b}\fxla d\lam}.
\end{equation*}
\underline{Demostración} \\
Sea $\lbrace \beta_\upsilon \rbrace$ una sucesión de puntos de $]a,b[$ tal que $\lim_{\upsilon \to +\infty}\beta_\upsilon =b$. Por el teorema de Fubini se verifica $\todo \upsilon$:\\
\begin{equation}
\int_S d\lam \int_a^{\beta_\upsilon}\fxla dx=\int_a^{\beta_\upsilon}dx\int_S \fxla d\lam
\end{equation}
$\todo \upsilon$ definimos $g_\upsilon:S\flecha \F$ por: \\
$$
g_\upsilon (\lam)=: \int_a^{\beta_\upsilon}\fxla dx, \phantom{} \todo \lam \en S
$$
La hipótesis 3) implica que la sucesión $\lbrace g_\upsilon \rbrace$ converge uniformemente en $S$ a la función \lam \flecha $\int_a^{\to b} \fxla dx$. Ya que $S$ es un conjunto integrable, sigue de ahí por la prop. 11 del cap. $\mathrm{III}$ que dicha función es integrable en $S$ y que, para $\upsilon \to +\infty$, el primer miembro de (1.19) converge a $\int_S d\lam \int_a^{\to b}\fxla dx$. \\
Pasando al límite en (1.19) para $\upsilon +\infty$ obtenemos: \\
\begin{equation}
\int_S d\lam \int_a^{\to b}\fxla dx=\int_a^{\to b}dx\int_S \fxla dx
\end{equation}
\phantom{separado para que funcione bine jajajajajajaa xd xd} c. q. d \\ \\
\underline{Derivación de funciones definidas por integrales impropias}\\
Recordemos el siguiente teorema clásico del cálculo diferencial sobre la derivación de sucesiones de funciones: \\
Sea $\mathrm{I}$ un intervalo de \R , \F \phantom{} un espacio de Banach. Sea $\lbrace f_\upsilon \rbrace$ una sucesión de funciones $\mathrm{I} \flecha \F$, derivaciones en todo punto de $\mathrm{I}$. 	\\
Se supone:\\
\begin{enumerate}[i)]
\item Existe $x_o \en \mathrm{I}$ tal que la sucesión $\lbrace f_\upsilon (x_o) \rbrace$ es convergente en \F .
\item La sucesión $\lbrace f_\upsilon' \rbrace$ converge uniformemente en $\mathrm{I}$ a cierta función $g:\mathrm{I}\flecha \F$.
\end{enumerate}
Entonces la sucesión $\lbrace f_\upsilon \rbrace$ converge uniformemente en $\mathrm{I}$ a una función $f:\mathrm{I} \flecha \F$ y se tiene $f'(x)=g(x)$ $\todo x \en \mathrm{I}$.\\
El lector encontrará la demostración p. ej. en H. Cartan: \textit{Cálculo Diferencial}.\\ \\
\textbf{Proposición 16.}\\
Sea $\mathrm{I}$ un intervalo de \R \phantom{} y sea $f:]a,b[ \times \mathrm{I} \flecha \F$.\\
Se supone:
\begin{enumerate}[1)]
\item $\todo \lam \en \mathrm{I}$ la función \x \flecha \fxla \phantom{} es integrable en $]a,\beta[$ \todo $\beta\en ]a,b[$. Se pone: $\Phi_\beta (\lam)=: \int_a^\beta \fxla dx$.
\item Para casi todo \x \en$]a,b[$ y \todo \lam \en $\mathrm{I}$ existe la derivada
\item Se tiene $\Phi_\upsilon' (\lam)=\int_a^\beta f_x'(\lam)dx$ $\todo \beta \en ]a,b[$ y $\todo \lam \en \mathrm{I}$.\\
Esta condición se cumple en particular si $\todo \beta \en ]a,b[$ existe una función $g_\beta:]a,\beta[ \flecha \R$, integrable en $]a,\beta[$ tal que:\\
$|| f_x'(\lam)|| \leq g_\beta (x)$ para casi todo \x \en $]a,\beta [$ y $\todo \lam \en \mathrm{I}$.
\item Existe $\lam_o \en \mathrm{I}$ tal que la integral impropia $\int_a^{\to b}f(x,
\lam_o)dx$ es convergente. 
\item La integral impropia $\int_a^{\to b}f_x' (\lam)dx$ es uniformemente convergen en $\mathrm{I}$.
\item La integral impropia $\int_a^{\to b}\fxla dx$ es uniformemente convergente en $\mathrm{I}$ y al poner:\\
\end{enumerate}

\begin{equation*}
\boxed{\Phi (\lam)=: \int_a^{\to b}\fxla dx, \phantom{s} \todo \lam \en \mathrm{I}}
\end{equation*}
Se tiene:
\begin{equation*}
\boxed{\Phi '(\lam)= \int_{a}^{\to}f_x(\lam)dx, \todo \lam \en \mathrm{I}}
\end{equation*}
\underline{Demostración.} \\
Sea $\lbrace \beta_\upsilon \rbrace$ una sucesión de puntos de $]a,b[$ tal que $\lim_{\upsilon \to +\infty}\beta_\upsilon=b$. Pongamos $\todo \upsilon \en \N$ y $\todo \lam \en \mathrm{I}$:
$$
\Psi_\upsilon (\lam)=:\int_a^{\beta_\upsilon}\fxla dx.
$$
Entonces en virtud de la condición 3):
\begin{equation}
\Psi_\upsilon' (\lam)=\int_a^{\beta_\upsilon}f'_x(\lam)dx \phantom{s} \todo \upsilon \en N,  \todo \lam \en \mathrm{I}
\end{equation}
La condición 4) implica:
\begin{equation}
\lim_{\upsilon +\infty}\Psi_\upsilon (\lam_o)=\int_a^{\to b}f(x,\lam_o)dx.
\end{equation}
La condición 5) entraña:
\begin{equation}
\lim_{\upsilon +\infty}\Psi_\upsilon' (\lam)=\int_a^{\to b}f_x'(\lam)dx.
\end{equation}
uniformemente en $\mathrm{I}$.\\
Por el teorema citado sobre la derivación de sucesiones de funciones, las relaciones (1.21), (1.22), (1.23) implican que la sucesión $\lbrace \Psi \rbrace$ converge uniformemente en $\mathrm{I}$ a una función $\Phi$, es decir, como se ve fácilmente por un razonamiento por contradicción, la integral impropia $\int_a^{\to  b}\fxla dx$ convergen uniformemente en $\mathrm{I}$ y $\Phi (\lam)=\int_a^{\to b}\fxla dx$ $\todo \lam \en \mathrm{I}$. Finalmente: \\
$$
\Phi ' (\lam)=\int_a^{\to b}f'_x(\lam)dx \phantom{s} \todo \lam \en \mathrm{I}
$$
\underline{Dos ejemplos clásicos} \\
\begin{enumerate}[1)]
\item Cálculo de la integral impropia $\int_0^{\to +\infty}\frac{\sen (x)}{x}dx$. \\
Sea $f:[0,+\infty[ \flecha \R$ la función definida por: \\

\begin{equation*}
\fxla= \left\{ \begin{array}{lcc}
            e^{-\lam x}\frac{\sen (x)}{x} &   si  & x \neq 0 \\
             \\  1 &  si &  x=0 \\
             \end{array}
   \right.
\end{equation*}
\end{enumerate}
$f$ es continua en su dominio. \\
\begin{enumerate}[a)]
\item Consideremos \todo \lam \en $[0,+\infty[$ la función continua $f^\lam :[0,+\infty[ \flecha \R$ definida por $f^\lam (x)=\fxla$ $\todo x \en [0,+\infty[$ \\
¿Es dicha función integrable en $[0,+\infty[$?\\
Supongamos primero $\lam>0$.\\
Notemos que las funciones \x \flecha \x$+\sen (x)$ y \x \flecha \x$-\sen (x)$ tienen derivadas $\geq 0$ luego son crecientes en $[0,+\infty[$. Puesto que se anulan en 0, estas funciones son $\geq 0$ en $[0,+\infty[$. De ahí sigue:\\
$$
| \sen (x) | \leq x \phantom{s} \todo \x \en [0,+\infty[
$$
Se tiene pues $|\fxla | \leq e^{-\lam x} \phantom{s} \todo x \geq 0.$\\
Por cierto en la hipótesis actual $\lam>0$, la función \x \flecha $e^{\lam x}$ es integrable en $[0,+\infty[$. En efecto, tratando dicha función como medible positiva, obtenemos:
\begin{equation*}
\left. 
\int_0^{+\infty}e^{-\lam x}dx=-\frac{e^{-\lam x}}{\lam}\right|_{x=0}^{+\infty}=\frac{1}{\lam}<+\infty
\end{equation*}
Así pues para $\lam>0$, $f^\lam$ es integrable en $[0,+\infty[$, o, si se prefiere, la integral impropia $\int_0^{+\infty}e^{-\lam x}\frac{\sen (x)}{x}dx$ es absolutamente convergente.\\
Para $\lam=0$ se tiene $f^o(x)=\frac{\sen (x)}{x}$ y la función $f^o$, como sabemos de los ejemplos después de la prop. 10, no es integrable en $[0,+\infty[$. Sin embargo la integral $\int_0^{+\infty}\frac{\sen (x)}{x}$ es convergente.\\
Podemos pues definir
\begin{equation}
\Phi (\lam)=\int_0^{+\infty}e^{-\lam x}\frac{\sen (x)}{x}dx \phantom{s} \todo \lam \en [0,+\infty[
\end{equation}
\item Vamos a mostrar que \todo \lam >0 existe la derivada $\Phi ' (\lam)$ y calcular dicha derivada. Para $\lam >0$ se puede omitir la flecha en la integral (1.24) y tratarla como integral propia.\\
Hallamos: $f_x' (\lam)=-e^{-\lam x}\sen (x)$.
Sea $\lam_o$, arbitrariamente fijado. Para $\lam>\lam_o$ tenemos $|f_x ' (\lam) | \leq e^{-\lam_o x}$ y $x \flecha e^{-\lam_o x}$ es una función integrable en $[0,+\infty[$, independiente de \lam . Por la prop. 2 .(derivación bajo el signo integral de integrales propias), obtenemos:
\begin{equation}
\Phi '(\lam)=-\int_0^{+\infty}e^{-\lam x}\sen (x)dx, \phantom{s} \todo \lam>0
\end{equation}
Ya que $\lam_o$ es arbitrario, la relación (1.25) es válida para todo $\lam>0$.\\
Por métodos elementales encontramos que una primitiva de la función \x \flecha $e^{- \lam x}\sen (x)$ es la función:
\begin{equation*}
x\flecha \frac{-e^{-\lam x}(\cos (x)+\lam \sen (x)}{1+\lam^2}
\end{equation*}
Luego por (1.25):\\
\begin{equation*}
\left.
\Phi ' (\lam)=\frac{e^{- \lam x}(\cos (x)+ \lam \sen (x))}{1+\lam^2}\right|_{x=0}^{x=+\infty}=-\frac{1}{1+\lam^2}
\end{equation*}
\item De ahí deducimos que $\todo \lam >0$:
\begin{equation}
\Phi (\lam)=C-\arctan (\lam)
\end{equation}
donde $C$ es una constante. Calculemos $C$.\\
Se tiene $\todo \lam >0$:\\
$$|\Phi (\lam)|=|\int_0^{+\infty}e^{-\lam x}\frac{\sen (x)}{x}dx| \leq \int_0^{+\infty}e^{-\lam x}|\frac{\sen (x)}{x}|dx $$
$$\leq \int_0^{+\infty}e^{- \lam x}dx=\frac{1}{\lam} \stackbin[\lam \flecha +\infty]{}{\flecha}0$$
Luego, pasando al límite en (1.26) para $\lam +\infty$, obtenemos:\\
$C=\frac{\pi}{2}$. La relación (1.26) se convierte pues en:\\
\begin{equation}
\Phi (\lam) =\frac{\pi}{2}-\arctan (\lam), \phantom{s} \todo \lam >0
\end{equation}
\item El caso $\lam=0$ nos interesa particularmente. Mostremos que $\Phi$ es continua en $0$.
\end{enumerate}
\begin{enumerate}[1)]
\item La función \x \flecha \fxla \phantom{} es continua, luego integrable en $[0,R]$ $\todo \lam \en [0,+\infty[$ y $\todo R>0$/
\item La función \lam \flecha \fxla \phantom{} es continua en $C$ $\todo \x \en [0,+\infty[$. 
\item Fijemos $L>0$ arbitrariamente. La función $(x,\lam)\flecha \fxla$ es continua en $[0,R]\times [0,L]$ $\todo R>0$. Luego, por el corolario de la prop.  1., la función \lam \flecha $\int_0^R e^{- \lam x}\frac{\sen (x)}{x}dx$ es continua en $[0,L]$, en particular en el punto $0$.
\item Afirmamos que la integral impropia $\int_0^{+\infty}e^{- \lam x}\frac{\sen (x)}{x}dx$ es uniformemente convergente con respecto a $\lam$ en $[0,+\infty[$.
Para todo $\lam \en [0,+\infty[$ la función \x\flecha $e^{-\lam x}$ es decreciente no negativa. Por la prop. 5 se tiene $\todo$ $x_1,x_2$, $0 \leq x_1 \leq x_2$:\\
\begin{equation}
| \int_{x_1}^{x_2}e^{- \lam x}\frac{\sen (x)}{x}dx | \leq e^{\lam x_1} \stackbin[x_1 \leq \epsilon \leq x_2]{}{Sup}| \int_{x_1}^\epsilon \frac{\sen (x)}{x}dx|
\end{equation}
\begin{equation*}
 \stackbin[x_1 \leq \epsilon \leq x_2]{}{Sup}| \int_{x_1}^\epsilon \frac{\sen (x)}{x}dx|
\end{equation*}
Sea $\epsilon >0$ dado arbitrario. Puesto que la integral impropia $\int_0^{\to +\infty}\frac{\sen (x)}{x}$ es convergente, existe $R>0$ (evidentemente independiente de $\lam$) tal que $y_2>y_2\geq R \flecha | \int_{y_1}^{y_2}\frac{\sen (x)}{x}| < \epsilon$.
\end{enumerate}
Al tomar $x_2 >x_1 \geq R$, el último miembro de (1.28) ser ..
a mayor abundamiento: \\
$$x_2 >x_1  \geq R \flecha | \int_{x_1}^{x_2}e^{- \lam x}\frac{\sen (x)}{x}dx| \leq \epsilon$$
Se cumple pues la condición de Cauchy para la convergencia de la integral impropia $\int_0^{\to +\infty}e^{- \lam x}\frac{\sen (x)}{x}dx$.\\
De 1), 2), 3), 4) deducimos, por la prop.13 que $\Phi$ es continua en $0$. Por consiguiente la fórmula (1.27) sigue siendo válida en cero, con lo cual hemos demostrado: 
\begin{equation*}
\boxed{\int_0^{+\infty}\frac{\sen (x)}{x}dx=\frac{\pi}{2}}
\end{equation*}

\underline{Cálculo de las integrales de Fresnel $\int_0^{\to +\infty}\sen (x^2)dx$ y $\int_0^{\to +\infty}\cos (x^2)dx$.}\\ \\
Para $t>0$ evaluemos la integral $\int_0^{+\infty}e^{-t u^2}du$. Por el cambio de variables $u=\frac{1}{\sqrt{t}}\int_0^{+\infty}e^{-z^2}dz=-\frac{1}{\sqrt{t}}\frac{\sqrt{\pi}}{2}$  (cf. ejemplo). \\
el comienzo del cap. V). Notaremos la relación equivalente:\\

\begin{equation}
\frac{1}{\sqrt{t}}=\frac{2}{\sqrt{\pi}}\int_0^{+\infty}e^{-tu^2}du
\end{equation}
En el ejemplo 5. después de la prop. 13 demostramos la relación $\int_0^{\to +\infty}\sen (x^2)dx=\frac{1}{2}\int_0^{\to +\infty}\frac{\sen (t)}{\sqrt{t}}dt$. Substituyendo en el segundo miembro $\frac{1}{\sqrt{t}}$ por su expresión (1.29) conseguimos:
\begin{equation}
\boxed{\int_0^{\to +\infty}\sen (x^2)dx=\frac{1}{\sqrt{\pi}}\int_0^{\to +\infty}\sen (t)dt\int_0^{\to +\infty}e^{-tu^2}du}
\end{equation}
Afirmamos que se puede invertir el orden de las integraciones en el segundo miembro de (1.30). En efecto: 
\begin{enumerate}[i)]
\item Sea $R>0$ arbitrario. Consideremos la función $(t,u) \flecha \sen (t)$. $e^{-tu^2}$ de $[0,R]\times [0,+\infty[$ en \R .\\
Se tiene para $u\neq 0: \int_0^R| \sen (t)| e^{-t u^2}dt \leq 
 \int_0^Re^{-t u^2}dt=\frac{1-e^{-Ru^2}}{u^2}$. Ya que $\lim_{u \to 0}\frac{1-e^{-Ru^2}}{u^2}=R$, la función $u \flecha \frac{1-e^{-Ru^2}}{u^2}$ será continua en $0$, si le atribuimos allí el valor de $R$. También $\frac{1-e^{-Ru^2}}{u^2} < \frac{1}{u^2}$ y la función $u \flecha \frac{1}{u^2}$ es integrable en $[1,+\infty[$. Así pues la función $u \flecha \frac{1-e^{-Ru^2}}{u^2}$es integrable $[0,+\infty[$. Luego se tiene:\\
$$
\int_0^{+\infty}du\int_0^R | \sen (t) | e^{-tu^2}dt \leq \int_0^{+\infty}\frac{1-e^{-Ru^2}}{u^2}du < +\infty.
$$
Por el teorema de Tonelli, la función $(t,u) \to \sen (t). e^{-tu^2}$ es integrable en $[0,R] \times [0,+\infty[$ $\todo R>0$.
\item $\todo u>0$ existe la integral impropia $\int_0^{\to +\infty}\sen (t)e^{-tu^2}dt$. En efecto se obtiene por un cálculo directo: 
\begin{equation}
\left.
 \int_0^{\to +\infty}\sen (t)e^{-tu^2}dt=e^{-tu^2}\frac{-u^2 \sen (t)-\cos (t)}{1+u^4}\right|_{t=0}^{+\infty}=\frac{1}{1+u^4}
\end{equation}
Por cierto dicha integral es aún absolutamente convergente (se puede suprimir la flecha).
\item Para $R>0$ arbitrario se tiene la mayoración: 
\end{enumerate} 
\begin{equation*}
| \int_0^R \sen (t).e^{-tu^2}dt|=|\frac{1}{1+u^4}-e^{-Ru^2}\frac{u^2 \sen (R)+\cos (R)}{1+u^4}|
\end{equation*}
$ \leq \frac{2+u^2}{1+u^4}$, donde la función $u \flecha \frac{2+u^2}{1+u^4}$ es una función integrable en $[0.+\infty[$, independiente de $R$.\\ 
Los resultados $\mathrm{I}), \mathrm{II}),\mathrm{III})$ implican, en virtud de la prop. 14., que efectivamente, se puede intercambiar el orden de las integraciones en (1.31). Se , teniendo en cuenta (1.31):
\begin{equation}
\int_0^{\to +\infty}\sen (x^2)dx=\frac{1}{\sqrt{\pi}}\int_0^{+\infty}\int_0^{\to +\infty}\sen (t).e^{-tu^2}dt=
\end{equation}
$$
=\frac{1}{\sqrt{\pi}}\int_0^{+\infty}\frac{du}{1+u^4}
$$
Por métodos elementales se consigue $\int_0^{+\infty}\frac{du}{1+u^4}=\frac{\pi}{2\sqrt{2}}$, luego por (1.32):
\begin{equation*}
\boxed{\int_0^{+\infty}\sen (x^2)dx=\frac{1}{2}\sqrt{\frac{\pi}{2}}}.
\end{equation*}
Análogamente se halla:
\begin{equation*}
\boxed{\int_0^{+\infty}\cos (x^2)dx=\frac{1}{2}\sqrt{\frac{\pi}{2}}}.
\end{equation*}



%%%%%%%%%%%%%%%%%%%%%%

\chapter{Espacios Hilbertianos \Li \phantom{} como dual de $L_1$}

\section{Generalidades. Proyecciones ortogonales.}
\underline{Definición.} \\
Sea $H$ un espacio vectorial sobre \K . $H$ se llamará ESPACIO PREHILBERTIANO, si está dada una aplicación $(\overrightarrow{x},\overrightarrow{y}) \rightarrow (\overrightarrow{x},\overrightarrow{y})$ de $H \times H$ en \K \phantom{} llamada PRODUCTO ESCALAR que satisface los siguientes axiomas:

\begin{enumerate}[1)]
\item \todo $\overrightarrow{y} \en H$ la aplicación $\overrightarrow{x} \flecha (\overrightarrow{x},\overrightarrow{y})$ de $H$ en \K \phantom{} es lineal.
\item $(\overrightarrow{x}|\overrightarrow{y})=\overrightarrow{(\overrightarrow{y},\overrightarrow{x})}$ $\todo \overrightarrow{x},\overrightarrow{y} \en H$.
\item $(\overrightarrow{x}|\overrightarrow{x})$ es real $ \geq 0$ $\todo \overrightarrow{x} \en H$.
\item $(\overrightarrow{x}|\overrightarrow{x})=0 \to \overrightarrow{x}=\overrightarrow{0}$

\end{enumerate}
La barra en el axioma 2) designa la conjugación compleja. Si \K =\R  este axioma reza simplemente $(\overrightarrow{x}|\overrightarrow{y})=(\overrightarrow{y}
|\overrightarrow{x})$ $\todo \overrightarrow{x},\overrightarrow{y} \en H$. En este caso $\todo \overrightarrow{x} \en H$ la aplicación \yraya \flecha $(\xraya|\yraya)$ es una aplicación lineal de $H$ en \K \phantom{} o sea $(\xraya , \yraya) \flecha (\xraya | \yraya)$ es una aplicación bilineal de $H \times H$ en \K . \\
Sin embargo, en el caso \K = \C \phantom{} rigen las reglas: 
\begin{equation*}
\boxed{(\xraya | \yraya_1 + \yraya_2)=(\xraya|\yraya_1)+(\xraya|\yraya_2)}
\end{equation*}
\begin{equation*}
\boxed{(\xraya|\alpha \yraya)=\overrightarrow{\alpha}(\xraya|\yraya) \phantom{s} \alpha \en \C}
\end{equation*}
Si $E,F$ son espacios vectoriales complejos una aplicación $L:E \flecha F$ se llama APLICACIÓN SEMILINEAL si:
$$
L(\xraya_1+\xraya_2)=L\xraya_1+L\xraya_2, \phantom{s} \todo \xraya_1, \xraya_2 \en E
$$
$$
L(\alpha \xraya)=\overrightarrow{\alpha}L\xraya, \phantom{s} \todo \alpha \en C, \phantom{s} \todo \xraya \en E.
$$
Así pues si \K = \C , la aplicación $\yraya \flecha (\xraya | \yraya)$ es, $\todo \xraya \en H$, una aplicación semilinea de $H$ en \C . Se dice en este caso que la aplicación $(\xraya,\yraya) \flecha (\xraya|\yraya) $ de $H \times H$ en \C \phantom{} es SESQUILINEAL (o sea " una vez y medio lineal").  \\
\underline{En todas las demostraciones a continuación supondremos que \K=\C .} Deberá ser evidente al lector como adaptarlas al caso más fácil \K=\R . \\
\underline{Definición.}\\
\todo \xraya \en $H$ se pone: 
\begin{equation*}
\boxed{|| x || =: \sqrt{(\xraya | \xraya)}}
\end{equation*}
\underline{Ejemplos.}\\
\begin{enumerate}[1)]
\item En \Rn , para $x=(x_1.\ldots, x_n), y=(y_1,\ldots, y_n)$ pongamos $(x|y)=: \sum_{k=1}^{n}x_k \overline{y_k}$. La aplicación $(x,y) \flecha (x|y)$ es un producto escalar que hace de $\K^{n}$ un espacio prehilbertiano. Aquí $|| x || =| (\sum_{k=1}^{n} |x_k|^2)^\frac{1}{2}.$
\item Sea $S$ un subconjunto medible de \Rn . Sean $f,g \en L_2(S,\K)$. (Confundimos aquí intencionalmente funciones con clases de equivalencia de funciones). Ponemos:
\begin{equation*}
\boxed{(f|g)=:\int_S f\overline{g}}
\end{equation*}
\end{enumerate}
La desigualdad de Holder con $p=p*=2$ implica que la existencia de la integral en el segundo miembro. Se comprueba fácilmente que la aplicación $(f,g) \flecha (f|g)$ es un producto escalar. Hace de $L_2 (S,\K)$ un espacio prehilbertiano. Notemos que 
$$
||f||=(\int_S |f|^2)^\frac{1}{2}=N_2(f), \phantom{s} \todo f \en L_2(S,\K).
$$
A continuación $H$ designa un espacio prehilbertiano. \\\
\underline{Definición.}\\
Vectores \xraya , \yraya \en $H$ se dicen ORTOGONALES y se escribe $\xraya\perp \yraya$ si $(\xraya | \yraya )=0$.\\
Notemos que si para cierto $\xraya \en H$ se tiene $\xraya \perp \overrightarrow{z}$ $\todo  \overrightarrow{z} \en H$, entonces $\xraya=\overrightarrow{0}$. En efecto se tendrá en particular $(\xraya | \xraya)=0$ y la conclusión sigue del axioma 4).\\ \\
\textbf{Proposición 1.}(Teorema de Pitágoras).\\
Si $(\xraya_1,\ldots ,\xraya_r)$ son vectores de $H$ ortogonales a pares se tiene $|| \xraya_1+ \ldots + \xraya_r ||^2= ||\xraya_1 ||^2+ \ldots + || \xraya ||^2.$

\underline{Demostración.}\\
En el caso $r=2$ se tiene $|| \xraya_1 + \xraya_2 ||^2 =(\xraya_1+\xraya_2 |\xraya_1+\xraya_2)=(\xraya_1 | \xraya_1)+(\xraya_1 | \xraya_2)+(\xraya_2|\xraya_1)+(\xraya_2 | \xraya_2)$. Puesto que $(\xraya_1 | \xraya_2)=(\xraya_2 | \xraya_1)=0$ queda bien $|| \xraya_1 + \xraya_2 ||^2 = || \xraya_1 ||^2+|| \xraya_2 ||^2$. El caso general sigue fácilmente por inducción. \\ \\
\textbf{Proposición 2.} \\
$\todo \xraya, \yraya \en H$, se tiene 
\begin{equation*}
\boxed{
|| \xraya +\yraya ||^2+ || \xraya -\yraya ||^2 = 2(|| \xraya ||^2 + || \yraya||^2)}
\end{equation*}
(IDENTIDAD DEL PARALELOGRAMO). \\
La demostración es inmediata. \\ \\
\textbf{Proposición 3.} \\ 
Se tiene $\todo \xraya, \yraya \en H$: 
\begin{equation*}
\boxed{(\xraya | \yraya )| \leq || \xraya || \phantom{s} || \yraya ||}
\end{equation*}
(DESIGUALDAD DE SCHWARZ).
Hay igualdad si y sólo si el sistema $(\xraya, \yraya)$ es linealmente dependiente. \\
\underline{Demostración.}\\
Descartando un caso trivial, supondremos $\yraya \neq 0$.\\
$\todo \lam \en \C$ se tiene: 
\begin{equation}
0 \leq (\xraya + \lam \yraya | \xraya +\lam \yraya)=| \lam|^2 \phantom{s} || \yraya ||^2 + 2 Re(\lam (\yraya | \xraya ))+ || \xraya ||^2
\end{equation}
Tomemos en particular:

\begin{equation*}
\lam= \left\{ \begin{array}{lcc}
            t\frac{(\xraya | \yraya)}{| (\xraya | \yraya)| } &   si  & (\xraya | \yraya ) \neq 0 \\
             \\  t &  si &  (\xraya | \yraya )=0 \\
             \end{array}
   \right.
\end{equation*}
La desigualdad (2.1) implica:
\begin{equation}
t^2 || \yraya || ^2+ 2t | (\xraya |\yraya)| + || \xraya ||^2 \geq 0, \phantom{s} \todo t\en \R.
\end{equation}
El primer miembro de (2.1) es un trinomio de segundo grado en $t$ con coeficientes reales. La desigualdad (2.2) se cumple si y sólo si el discriminante de dicho trinomio es $\leq 0$, o sea: \\
$| (\xraya | \yraya ) |^2 - || \xraya ||^2 \phantom{s} || \yraya|| ^2 \leq 0$, lo que equivale bien a: 
$$
| (\xraya | \yraya ) \leq || \xraya || \phantom{s} || \yraya ||.
$$
La igualdad aquí entraña que el considerado trinomio tiene una raíz doble o sea que existe \lam \en \C \phantom{s} tal que \xraya +\lam \yraya=0 . En otras palabras el sistema $(\xraya, \yraya)$ es linealmente dependiente. Es inmediato que esta condición es también suficiente para que la desigualdad de Schwarz sea una igualdad. \\
\phantom{sadasdaadsaadasdasdasdasdasdasdvsadasdasdadasssasaa} c. q. d. \\
\underline{Observación.}\\
En el caso $H=L_2 (S,\K)$ la desigualdad de Schwarz no es otra que la desigualdad de Holder con $p=p*=2$. \\ \\
\textbf{Proposición 4.} \\
Se tiene $\todo \xraya, \yraya \en H$: 
\begin{equation*}
\boxed{ || \xraya + \yraya || \leq || \xraya || + ||\yraya ||}
\end{equation*} (DESIGUALDAD TRIANGULAR).
Hay igualdad si y sólo si uno de los vectores \xraya , \yraya \phantom{} es múltiplo no negativo del otro.\\
\underline{Demostración.}\\
Valiéndonos de la desigualdad de Schwarz obtenemos: \\
$|| \xraya + \yraya ||^2 =|| \xraya ||^2 + ||\yraya ||^2+2Re (\xraya | \yraya ) \leq || \xraya ||^2 +2|(\xraya | \yraya)| \leq  || \xraya ||^2 + || \yraya ||^2+2 || \xraya ||^2 \phantom{s} || \yraya || = ( || \xraya ||+ || \xraya ||)^2$, de donde $|| \xraya + \yraya|| \leq || \xraya ||^2+|| \yraya ||$.
La igualdad tiene lugar si y sólo si $|(\xraya | \yraya ) = || \xraya || \phantom{s} || \yraya ||$ y $(\xraya | \yraya) \geq 0$, o sea si y sólo si $(\xraya, \yraya)$ es linealmente dependiente y $(\xraya | \yraya) \geq 0$. Esto significa bien que uno de los vectores es múltiplo no negativo del otro. \\ \\

\textbf{Proposición 5.} \\
\underline{  es una norma sobre $H$.}
\underline{Demostración.}\\
Es claro que $ || \xraya || \geq 0$. La relación $|| \lam \xraya ||=|\lam| \phantom{s} ||\xraya ||$ es inmediata. La relación $|| \xraya + \yraya || \leq || \xraya || + ||\yraya||$ se acaba de demostrar. Finalmente la implicación $||\xraya || =0 \flecha \xraya=0$ sigue del axioma 4) de producto escalar. \\
 \\
 \textbf{Proposición 6.}\\
 La aplicación $(\xraya, \yraya) \flecha (\xraya | \yraya)$ es una aplicación continua de $H \times H$ en $H$.  \\
 \underline{Demostración.} (Cf. ejercicio). \\
 Fijemos $\xraya, \yraya \en H$ y sean $\lbrace \xraya_\upsilon \rbrace$, $\lbrace \yraya_\upsilon \rbrace$ sucesiones en $H$ que convergen a sendos $\xraya, \yraya$. Debemos probar que $\lbrace (\xraya_\upsilon | \yraya_\upsilon)  \rbrace$ converge a $(\xraya | \yraya )$. Ahora bien, aplicando la desigualdad de Schwarz. 
\begin{equation}
|(\xraya | \yraya)-( \xraya_\upsilon | \yraya_\upsilon)| =| (\xraya | \yraya) \phantom{s} (\xraya_\upsilon | \yraya) + (\xraya_\upsilon| \yraya)- (\xraya_\upsilon | \yraya_\upsilon)=
\end{equation}
\begin{equation*}
=|(\xraya-\xraya_\upsilon| \yraya) +(\xraya_\upsilon |\yraya - \yraya_\upsilon)| \leq |(\xraya-\xraya_\upsilon| \yraya)| + |(\xraya_\upsilon|\yraya-\yraya_\upsilon )| \leq 
\end{equation*}
\begin{equation*}
||\xraya - \xraya_\upsilon || \phantom{s} || \yraya || + || \xraya_\upsilon || || \yraya-\yraya_\upsilon ||
\end{equation*}
La hipótesis $\lim_{\upsilon \to +\infty}\xraya_\upsilon=\xraya$ implica que\\
$\exists n_0$ tal que $ \upsilon \geq n_0 \flecha$ $|| \xraya_\upsilon-\xraya || \leq 1$, de donde \\
$\upsilon \geq n_0 \flecha || \xraya_\upsilon || \leq 1+ || \xraya || $. \\
Para $\upsilon \geq n_0$ será pues por (2.3): \\
\begin{equation}
| (\xraya | \yraya) - (\xraya_\upsilon | \yraya_\upsilon)| \leq || \xraya - \xraya_\upsilon || \cdot || \yraya || + (1+|| \xraya ||)|| \yraya -\yraya_\upsilon
\end{equation}
Sea $\epsilon >0$ dado. Entonces $\exists n_1 \geq n_0$ tal que
$$
\upsilon \geq n_1 \flecha || \xraya - \xraya_\upsilon || < \frac{\epsilon}{2(1+|| \yraya ||)}, \phantom{s} || \yraya -\yraya_\upsilon || < \frac{\epsilon}{2(1+||\xraya||)}
$$
Se sigue (2.4) que 
$$
\upsilon \geq n_1=| (\xraya | \yraya)-(\xraya_\upsilon |\yraya_\upsilon)| < \epsilon
$$
\phantom{sssssssssssssssssssssssssssssssssss sasdasdasdasdadadssada} c. q. d

\underline{Definición.}\\
Un espacio prehilbertiano se llama ESPACIO HILBERTIANO si la norma $||$ $||$ hace de él un espacio de Banach ( osea un espacio normado completo).\\
\underline{Ejemplos.}
\begin{enumerate}[i)]
\item $\K^n$ provisto del producto escalar $(x|y)=\sum_{k=1}^{n}x_k\overline{y_k}$ es un espacio hilbertiano.
\item $L_2(S,\K)$ es un espacio hilbertiano.
\end{enumerate}
\textbf{Proposición 7 y definición.} \\
Sea $H$ un espacio prehilbertiano. Sea \x\en $H$. Sea $M$ un subespacio de $H$. 
\begin{enumerate}[1)]
\item Si existe $\xraya_o \en M$ tal que $\xraya-\xraya_o \perp M$ (o sea $\xraya - \xraya_o \perp \overrightarrow{v}$ $\todo \yraya \en M$), rige la implicación: \\
$\yraya \en M$, $\overrightarrow{v} \neq \xraya_o \flecha  || \xraya -\yraya || > || \xraya -\xraya_o ||$.  \\
Luego $\xraya_o$ es el único y la distancia de \xraya \phantom{} a $M$ es $d(\xraya , M)= || \xraya - \xraya_o ||$. $\xraya_o$ se llama PROYECCIÓN ORTOGONAL de \xraya \phantom{} sobre $M$. \\
Se tiene:
\begin{equation*}
\boxed{(d(\xraya,M)^2=|| \xraya ||^2+||\xraya_o ||^2}
\end{equation*}
\item Recíprocamente, si existe $\xraya_o \en M$ tal que $|| \xraya-\xraya_o||=d(\xraya,M)$, $\xraya_o$ es la proyección ortogonal de $\xraya$ sobre $M$. \\
\end{enumerate}
\underline{Demostración.}\\
\begin{enumerate}[a)]
\item Sea $\xraya_o \en M$ tal que $\xraya-\xraya_o \perp M$. Sea $\yraya \en M$, $\yraya \neq \xraya_o$. \\
Se obtiene por el teorema de Pitágoras:
$$
|| \xraya -\yraya||^2=|| (\xraya -\xraya_o) +(\xraya_o-\yraya||^2=|| \xraya -\xraya_o||^2+|| \xraya -\yraya||^2>|| \xraya - \xraya_o||^2
$$
o sea $|| \xraya -\yraya || > || \xraya - \xraya_o ||$, como afirmamos. \\
Además, de nuevo por el teorema de Pitágoras:
$$
|| \xraya||^2=|| (\xraya-\xraya_0)+\xraya_o||^2=|| \xraya-\xraya_o||^2+|| \xraya_o||^2=d(\xraya,M)^2+|| \xraya_o||^2.
$$
Así pues $d(\xraya,M)^2=|| \xraya||^2-|| \xraya_o||^2$
\item  Recíprocamente supongamos que existe $\xraya_o \en M$ tal que $d(\xraya,M)=|| \xraya-\xraya_o ||$. \\
Entonces $\todo \yraya \en M$ y $\todo \lam \en \C$ se verifica:
\end{enumerate}
$$
||  \xraya- \xraya_o+\lam \yraya||^2 \geq ||  \xraya-\xraya_o ||^2
$$
De donde, el desarrollar: \\
$$|| \xraya-\xraya_o||^2 +2Re(\lam (\yraya|\xraya-\xraya_o))+|\lam|^2 \phantom{s} || \yraya||^2 \geq || \xraya -\xraya_o ||^2
$$
Es decir
\begin{equation}
2Re(\lam(\yraya|\xraya-\xraya_o))+|\lam|^2 \phantom{s}|| y||^2 \geq 0. 
\end{equation}
Tomemos en particular $\lam=t(\xraya-\xraya_o|\yraya)$ con $t\en \R$. \\
La relación (2.5) implica: \\
\begin{equation}
2t | (\xraya-\xraya_o|\yraya)|^2+t^2|(\xraya-\xraya_o|\yraya)|2 \phantom{s} || \yraya ||^2 \geq 0, \phantom{s} \todo t \en \R
\end{equation}
Esto equivale a que el coeficiente de $t$ en (2.6) sea nulo, es decir $(\xraya-\xraya_o|\yraya)=0$ $\todo \yraya \en M$. Demostramos pues $\xraya-\xraya_o\perp M$, luego $\xraya_o$ es la proyección ortogonal de \xraya \phantom{} sobre $M$.\\
\phantom{sssssssssssssssssssssssssssssssssss sasdasdasdasdadadssada} c. q. d
\\

Si $\xraya \en H$ y $M$ es un subespacio de $H$, puede no existir la proyección ortogonal de \xraya \phantom{} sobre $M$. \\
Un subespacio $M$ de un espacio prehilbertiano H se llamará SUBESPACIO DISTINGUIDO si $\todo x \en H$ existe la proyección ortogonal de \xraya \phantom{} sobre $M$. \\ \\
\textbf{Proposición 8.} \\
Todo subespacio completo de un espacio prehilbertiano es un subespacio distinguido.\\
\underline{Demostración.}\\
Sea $M$ un subespacio completo de un espacio prehilbnertiano $H$. Sea $\xraya \en H$ y sea $a=:d(x,M)$. Por definición de esta última distancia existe una sucesión $\lbrace \yraya_\upsilon \rbrace$ de puntos de $M$ tal que $\lim_{\upsilon + \infty}||\xraya -\yraya_\upsilon||=a$. Sean $\upsilon, \mu$ dos índices arbitrarios. Aplicando la identidad del paralelogramo (prop. 2) a los vectores $\xraya - \yraya_\upsilon$, $\xraya-\yraya_\mu$ conseguimos:
\\ \\
\textbf{Proposición 9.}\\
Sea $H$ un espacio prehilbertiano. Sea $M$ un subespacio de $H$. $M$ es distinguido si y solo si $H$ es la suma directa de $M$ y $M^{\perp}:H=M\oplus M^{\perp}$. \\
\underline{Demostración.}\\
\begin{enumerate}[a)]
\item Supongamos que $H=M \oplus M^{\perp}$. Entonces todo vector $\xraya \en H$ se representa únicamente en la forma $\xraya=\xraya_1+\xraya_2$ con $\xraya_1 \en M$, $\xraya_2 \en M^{\perp}$. Así pues $\xraya-\xraya_1=\xraya_2 \perp M$. Por conseguimiento $\xraya_1$ es la proyección ortogonal de \xraya \phantom{} sobre $M$. Así viene demostrado que $M$ es distinguido.
\item Sea $M$ arbitrario. Ya que evidentemente $M \cap M^{\perp}=\lbrace \overrightarrow{0} \rbrace$, basta demostrar que $H$ es la suma de $M$ y $M^\perp$ o sea que todo $\xraya \en H$  puede representarse en la forma $\xraya=\xraya_1+\xraya_2$ con $\xraya_1 \en M$, $\xraya \en M^{\perp}$. Sea $\xraya_1$ la proyección ortogonal de $\xraya$ sobre $M$ y sea $\xraya_2=\xraya-\xraya_1$. Por definición de la proyección ortogonal $\xraya_2 \en M^{\perp}$.

\end{enumerate}

\underline{Corolario.}\\
Si $M$ es un subespacio distinguido de $H$, $M^\perp$ es también un subespacio distinguido. \\
En efecto, para $\xraya \en H$ arbitrario, escribamos $\xraya=\xraya_1+\xraya_2$ con $\xraya_1 \en M$, $\xraya_2 \en M$. Entonces $\xraya-\xraya_2=\xraya_1 \perp M$. Luego $\xraya_2$ es la proyección ortogonal de $\xraya$ sobre $M^{\perp}$.\\ \\
\textbf{Proposición 10.} \\
Si $M$ es un subespacio distinguido de un espacio prehilbertiano $H$ se tiene $M^{\perp \perp}=M$. \\
\underline{Demostración.}\\
Es claro que $M\subset M^{\perp \perp}$. Demostremos la inclusión inversa $M^{\perp \perp} \subseteq M$. Sea pues $\xraya$ se puede representar únicamente en la forma: $\xraya=\xraya_1+\xraya_2$ con $\xraya_1 \en M$, $\xraya_2 \en M^{\perp}$. Se tiene:\\
$$
0=(\xraya | \xraya_2)=(\xraya_1 | \xraya_2) + (\xraya_2 | \xraya_2)= (\xraya_2 | \xraya_2)$$.
Luegp $\xraya_2=0$ , de donde $\xraya=\xraya_1 \en M$.  \\
\phantom{sssssssssssssssssssssssssssssssssss sasdasdasdasdadadssada} c. q. d \\ \\
\underline{Corolario.} \\
Todo subespacio distinguido de un espacio prehilbertiano es un subespacio cerrado. \\ \\
\textbf{Proposición 11.} \\
Si $H$ es un espacio hilbertiano, un subespacio $M$ de $H$ es distinguido si y sólo si $M$ es cerrado en $H$. \\
\underline{Demostración.} \\
Ya sabemos que la condición es necesaria. Pero si $M$ es cerrado en el espacio completo $H$, $M$ es completo. Luego $M$ es distinguido por la prop. 8. \\
\phantom{sssssssssssssssssssssssssssssssssss sasdasdasdasdadadssada} c. q. d \\ \\

\textbf{Proposición 12 y definición.} \\
Sea $H$ un espacio prehilbertiano. Sea $M$ un subespacio distinguido de $H$, $M \neq \lbrace 0 \rbrace$. $\todo x \en H$ designemos por $\pi \xraya$ a la proyección ortogonal de $x$ sobre $M$. Entonces $\pi$ es una aplicación lineal continua de $H$ en $H$ con $ || \overrightarrow{\pi}||=1$. $\pi$ satisface: \\
$\pi ^2=\pi$ y $(\pi \xraya | \yraya)=(\xraya | \pi \yraya)$ $\todo \xraya , \yraya \en H$. \\
$\pi$ se llama la PROYECCIÓN sobre $M$. \\
\underline{Demostración.}\\
Si $\xraya \en H$, $\xraya- \pi \xraya \perp M$. Si $\alpha \pi \xraya \en M$ Y $\alpha \xraya - \alpha \pi x \perp M$. Luego $\alpha \pi \xraya = \pi (\alpha \xraya )$. \\
Si $\xraya, \yraya \en H$, $\pi \xraya + \pi \yraya \en M$ y $(\xraya + \yraya)-(\pi \xraya + \pi \yraya) \perp M$ luego $\pi \xraya + \pi \yraya=\pi(\xraya + \yraya)$. Así pues $\pi$ es una aplicación lineal de $H$ en $H$. \\
Se tiene $\todo \xraya \en H$: $||\pi \xraya ||^2=||\xraya ||^2-|| \xraya -\pi \xraya ||^2$ luego $|| \pi \xraya || \leq || \xraya ||$. Así pues $\pi$ es continua y $|| \pi || \leq 1$. Pero si $\xraya \en M$, $\xraya \neq 0$, se  verifica $\xraya = \pi \xraya$, luego $|| \xraya ||=|| \pi \xraya|| \leq 1$ de donde $|| \pi || \geq 1$. Finalmente $|| \pi || =1$.\\
La relación $\pi^2=\pi$ es evidente. \\
Sea ahora $\xraya, \yraya \en H$. Ya que $\pi \xraya \en M$ e $\yraya-\pi \xraya \perp M$ se tiene $(\pi \xraya, \yraya-\pi \yraya)=0$, de donde 
$$
(\pi \xraya | \yraya)=(\pi \xraya | \pi \yraya).
$$
Intercambiando $\xraya$ e $\yraya$ se consigue \\
$(\pi \yraya | \xraya)=(\pi \yraya |\pi \xraya)$.\\
Así pues $(\pi \xraya | \yraya)=\overline{(\pi \yraya | \xraya)}=(\xraya | \pi \yraya)$.
\phantom{sssssssssssssssssssssssssssssssssss sasdasdasdasdadadssada} c. q. d \\ \\

\textbf{Proposición 13.}\\
Sea $H$ un espacio prehilbertiano. Sea $\pi$ una aplicación de $H$ en $H$ que $\pi^2=\pi$ y que 
$(\pi \xraya | \yraya)=( \xraya | \pi \yraya)$ $\todo \xraya, \yraya \en H$. \\
Entonces existe un único subespacio distinguido $M$ de $H$ tal que $\pi$ es la proyección sobre $M$.  \\
\underline{Demostración.} \\
$M$, si existe, es necesariamente la imagen de $H$ de bajo $\pi$ o sea $M=\lbrace \pi \yraya | \yraya \en H \rbrace$. Definamos $M$ por esta relación. Debemos probar que $\todo \xraya \en H$, $\xraya - \pi \xraya \perp M$ o sea $(\xraya-\pi \xraya | \pi \yraya)=0$ $\todo \xraya, \yraya \en H$.\\
Ahora bien: $((\xraya-\pi \xraya | \pi \yraya)=(\xraya | \pi \yraya )- (\pi \xraya | \pi \yraya)=(\xraya | \pi \yraya) - (\xraya | \pi^2 \yraya )=(\xraya| \pi \yraya)-(\xraya | \pi \yraya)=0$.
\phantom{sssssssssssssssssssssssssssssssssss sasdasdasdasdadadssada} c. q. d \\ \\
\subsection{Autodualidad de un espacio hilbertiano}
Sea $H$ un espacio prehilbertiano. Designamos por $H^*$ el dual topológico de $H$ es decir el espacio vectorial de todas las formas lineales continuas sobre $H$. Si $\omega$ es una lineal sobre $H$ y $\xraya \en H$, designaremos por $< \xraya, \omega>$ al valor de $\omega$ sobre $\xraya$. La norma sobre $H^*$ se define por $|| \omega ||=M\acute{i}n\lbrace a | \phantom{s} <\xraya, \omega > | \leq a || \xraya||, \phantom{s} \todo \xraya \en H \rbrace$. (Cf. Cap. I, observación después de la prop. 7). Esta norma hace de $H*$ un espacio de Banach (cap. I, observación después de la prop. 22).  \\ \\
\textbf{Proposición 14.} \underline{(Autodualidad de un espacio hilberiano)} \\
Sea $H$ un espacio hilbertiano $H\neq \lbrace 0 \rbrace$. A todo $\yraya \en H$ asocíemosle la forma lineal $G\yraya$ sobre $H$ dada por: 
\begin{equation*}
\boxed{<\xraya,G\yraya >=: (\xraya | \yraya) \phantom{s} \todo \xraya \en H.}
\end{equation*}
Entonces $G$ es una biyección semilineal de $H$ sobre $H*$ que preserva las normas. \\
\underline{Demostración.} \\
\begin{enumerate}[a)]
\item Por la desigualdad de Schwarz obtenemos:
\begin{equation}
| <\xraya, G\yraya>|=|(\xraya | \yraya) | \leq || \xraya || \phantom{s} || \yraya || \phantom{ss} \todo \xraya \en H
\end{equation}
Luego $G\yraya \en H*$ y $||G \yraya || \leq || \yraya ||$ \\
Pero si $\yraya \neq 0$, $|| \yraya ||^2=| < \yraya , G\yraya >| \leq || G\yraya || \phantom{s} ||\yraya||$, luego $||G \yraya || \geq || \yraya ||$, de ahí que $|| G\yraya || = || \yraya ||$. \\
Se comprueba de inmediato que $G$ es una aplicación semilineal de .. Por el hecho de preservar las normas, $G$ es una aplicación inyectiva.\\

Notemos las relaciones: \\
\begin{equation}
 \left\{ \begin{array}{lcc}
           \yraya \en {(Ker(G\yraya))}^\perp   & \todo \yraya \en H \\
             \\  || \yraya ||^2= <\yraya,G\yraya > \\
             \end{array}
   \right.
\end{equation}
(Si $\omega$ es una forma lineal sobre $H$, $Ker(\omega)$ designa el núcleo de $\omega$ o sea $\lbrace \xraya | \xraya \en H, <\xraya, \omega>=0 \rbrace$). \\
\item Queda por mostrar que $G$ es superyectiva. Sea pues dada $\omega \en H*$, $\omega \neq 0$. Debemos buscar $\yraya \en H$ tal que $G\yraya=\omega$. Por las condiciones (2.8) $\yraya$ buscando debe satisfacer:
\end{enumerate}
\begin{equation}
 \left\{ \begin{array}{lcc}
           \yraya \en {(Ker(G\omega))}^\perp \\
             \\  || \yraya ||^2= <\yraya,\omega > \\
             \end{array}
   \right.
\end{equation}
Notemos que, por ser $\omega$ continua, $Ker \omega$, imagen inversa por $\omega$  del punto $\lbrace 0 \rbrace$ de \K , es un subespacio cerrado de $H$. Puesto que $H$ es completo, $Ker(\omega)$ es un subespacio distinguido, luego, por la prop. 9 , $H=(Ker(\omega)) \oplus {(Ker(\omega))}^\perp$. De ahí sigue que la restricción de $\omega$ a  ${(Ker(\omega))}^\perp$ es un isomorfismo de ${(Ker(\omega))}^\perp$ sobre \K , por lo tanto $dim{(Ker(\omega))}^\perp=1$. \\
Sea $\overrightarrow{z}$ un generador de $(Ker(\omega))^{\perp}$. Por la primera condición (2.9) debemos tener $\yraya=\alpha \overrightarrow{z}$ con $\alpha \en \K$. La segunda condición (2.9) implica $| \alpha|^2 || \overrightarrow{z}||^2=\alpha < \zraya,\omega>,$ de donde $\alpha=\overline{\frac{<\zraya,\cdot>}{|| \zraya ||^2}}$, o sea 
\begin{equation}
\yraya=\overline{\frac{<\zraya,\omega>}{|| \zraya ||^2}\zraya}
\end{equation}
Queda por probar que $\yraya$ definido por la fórmula (2.30) efectivamente satisface $G\yraya=\omega$ o sea $(\xraya | \yraya )=< \xraya, \omega >$ $\todo \xraya \en H$\\
Todo $\xraya \en H$ puede representarse únicamente en la forma $\xraya=\xraya_1+\beta \zraya$ con $\xraya_1 \en Ker(\omega)$. Luego, utilizando (2.30): \\
$$
(\xraya | \yraya)=(\beta\zraya | \yraya)=\beta \frac{<\zraya,\omega>}{|| \zraya ||^2} ||\zraya ||^2=\beta <\zraya , \omega>=<\xraya,\omega>.
$$
\phantom{sssssssssssssssssssssssssssssssssss sasdasdasdasdadadssada} c. q. d \\ \\
\underline{Observación.} \\
Nada cambiaría en la demostración, si en vez de suponer $H$ hilbertiano, supusiéramos que $H$ es un espacio prehilbertiano, en el cual todo subespacio cerrado es distinguido. Pero la conclusión del teorema afirma que en tal caso de $H$ es (semilinealmente) isométrico al espacio completo $H*$. Luego $H$ es completo. Tenemos pues el siguiente recíproco de la prop. 11. \\ \\
\textbf{Proposición 15.} \\
Si en un espacio prehilbertiano $H$ todo subespacio cerrado es distinguido, $H$ es hilbertiano. \\ \\
\textbf{Proposición 16.}\underline{(Autodualidad de $L_2(S,\K)$.)}\\
A todo $g\en L_2(S,\K)$ asociémosle la forma lineal $\phi_g$ sobre $ L_2(S,\K)$ definida por:
\begin{equation*}
\boxed{\phi_g (f)=\int_S fg \phantom{s} \todo f\en L_2(S,\K).}
\end{equation*}
Entonces la aplicación $g \flecha \phi_g$ es una isometría lineal de $L_2(S,\K)$ sobre su dual topológico $L_2(S,\K)^{*}$.\\
\underline{Demostración.} \\
$\todo h \en L_2$ sea $\phi_h$ la forma lineal sobre $L_2$ definido por $\Psi_h (f)=:(f|h)=\int_S f\overline{h}$ $\todo f \en L_2$. Ya que $L_2$ es un espacio hilbertiano, la prop. 4. , implica que la aplicación $h \flecha \Psi_h$ es una isometría semilineal de $L_2$ sobre $L_2^*$. \\
Siendo $\phi_g=\Psi_{\overline{g}}$ $\todo g \en L_2$, aplicación $g \flecha \phi_g$ es la compuesta de la isometría semilineal $g \flecha \overline{g}$ de $L_2$ sobre $L_2$ con la isometría semilineal $h \flecha \Psi h$ de $L_2$ sobre $L_2^*$. Luego $g \flecha \phi_g$ es una isometría lineal de $L_2$ sobre $L_2^*$.
\phantom{sssssssssssssssssssssssssssssssssss sasdasdasdasdadadssada} c. q. d \\ \\

\subsection{Familias ortonormales. Clasificación de los espacios hilbertianos.}
Recordemos al lector que repase los resultados sobre familias sumables de números no negativas (prop. 13 y 14 del cap. II.) \\
\underline{Definición.}\\
Una familia $(u_\alpha)_{\alpha \en \Omega}$ de números completos (o reales) se dice FAMILIA SUMABLE si $(|u_y|_{\alpha \en \Omega}$ es una familia sumable de números no negativos. \\ \\
\textbf{Proposición 17 y definición.} \\
Sea $(u_\alpha)_{\alpha \en \Omega}$ una familia sumable de números complejos.
\begin{enumerate}[1)]
\item Sea $\Omega_o=\lbrace \alpha| u_\alpha \neq 0 \rbrace$. Entonces $\Omega_o$ es a lo sumo numerable.
\item Si $i \flecha \alpha (i)$ es una biyección arbitraria de $N$ sobre $\Omega_o$, la serie $\lbrace \lbrace u_\alpha (i) \rbrace\rbrace_{i \en N}$ es absolutamente convergente. La suma $s$ de esta serie es independiente de la biyección elegida. $s$ se llama la SUMA de la FAMILIA SUMABLE $(u_\alpha)_{\alpha \en \Omega}$. Se escribe
$$
s=\sum_{\alpha \en \Omega}u_\alpha
$$

\end{enumerate}

\underline{Demostración.}\\
De la prop. 13. del cap. II se sigue inmediatamente que $\Omega_o$ es a lo suma numerable. Sea $i \flecha \alpha (i)$ una biyección de $N$ sobre $\Omega_o$. De la misma prop. 13. del cap. II resulta que la serie $\lbrace \lbrace u_{\alpha (i)} \rbrace\rbrace_{i \en N}$ es convergente. Luego la serie $\lbrace \lbrace u_{\alpha (i)} \rbrace\rbrace$ es absolutamente convergente. Sea $i \flecha \beta (i)$ otra biyección de $N$ sobre $\Omega_o$. Sean:
$$
s=:\sum_{i=u}^{+ \infty}u_{\alpha (i)} \phantom{s}, \phantom{ss} t=:\sum_{i=1}^{+ \infty}u_{\beta (i)} \phantom{s}
$$
Sea $\epsilon >0$ dado. Por absoluta convergencia de nuestras series \\
$\exists n_o \en N$ tal que $\sum_{i>n_o}| u_{\alpha (i)} | < \frac{\pi}{3}$ y $\sum_{i>n_o} | u_{\beta (I)}| < \frac{\pi}{3}$. \\
También: \\
$\exists n_1 \geq n_o$ tal que $\lbrace {\alpha (1)}, \ldots , \alpha (n_o) \rbrace \subset \lbrace \beta (1), \ldots, \beta (n_1) \rbrace$ \\
Escribimos:
\begin{equation}
|s-t| \leq |s-\sum_{i=1}^{n_o}u_{\alpha (i)} | + |\sum_{i=1}^{n_o}u_{\alpha (i)}-\sum_{i=1}^{n_1}u_{\beta (i)}|+ | \sum_{i=1}^{n_1}u_{\beta (i)}-t|
\end{equation}
El primero y el último término a la derecha de (2.31) son $< \frac{\epsilon}{3}$. En el término del medio quedan después de la reducción solamente unos $u_{\alpha (i)}$ con $i> n_o$. Luego este término es también $<\frac{\epsilon}{3}$. Se sigue pues de (2.31): $|s-t| < \epsilon$ y, ya que $\epsilon$ es arbitrario $s=t$.
\phantom{sssssssssssssssssssssssssssssssssss sasdasdasdasdadadssada} c. q. d \\ \\

\underline{Definiciones.} \\
Sea $\Omega$ un conjunto. 
\begin{enumerate}[1)]
\item Se designa por $L_1(\Omega ,\K)$ al conjunto de todas las funciones $f: \Omega \flecha \K$ tales que $(f(\alpha))_{\alpha \en \Omega}$ sea una familia sumable. 
\item Se designa por $L_2(\Omega ,\K)$ al conjunto de todas las funciones  $f: \Omega \flecha \K$ tales que $(f(\alpha))_{\alpha \en \Omega}^2$ sea una familia sumable. 
\item Se define $N_1:L_1(\Omega ,\K)\flecha \R$ por 
\begin{equation*}
\boxed{N_1 (f)=: \sum_{\alpha \en \Omega}| f(\alpha)|}
\end{equation*}
\item Se define $N_2:L_2(\Omega ,\K)\flecha \R$ por
\begin{equation*}
\boxed{N_2 (f)=:(| \sum_{\alpha \en \Omega}f(\alpha) |^2)^{\frac{1}{2}}}
\end{equation*}
\end{enumerate}
\textbf{Proposicion 18.}\\
$l_1 (\Omega,\K)$ es un espacio vectorial sobre $\K$ y $N_1$ es una norma sobre él. Además si $f,g \en l_1 (\Omega,\K)$ se tiene
\begin{equation*}
\sum_{\alpha \en \Omega}(f+g)(\alpha)= \sum_{\alpha \en \Omega}f(\alpha)+\sum_{\alpha \en \Omega}g(\alpha)
\end{equation*}
\underline{Demostración.} \\
Sean $f,g \en l_1 (\Omega,\K)$. \\
Si I es una parte finita de $\Omega$, se tiene:
$$
\sum_{\alpha \en I}| f+g| (\alpha) \geq \sum_{\alpha \en I}|f(\alpha)|+ \sum_{\alpha \en I} | g(\alpha) \leq N_1 (f)+N_1(g)
$$
Esto prueba que $f+g\en l_1 (\Omega,\K)$ y 
$$
N_1 (f+g) \leq N_1 (f) + N_1(g).
$$
Designemos por $F (\Omega)$ al conjunto de todas las partes finitas de $\Omega$. Si $f \en l_1 (\Omega,\K)$ y $\lam \en \K$ se tiene
$$
 \stackbin[I \en N (\Omega)]{}{Sup}  \sum_{\alpha \en I}| \lam f| (\alpha)=| \lam |  \stackbin[I \en N (\Omega)]{}{Sup}  \sum_{\alpha \en I} | f(\alpha)| = | \lam | N_1 (f).
$$
Luego $\lam \en N_1 (\Omega ,\K)$ y $N_1 (\lam f)=| \lam | N_1 (f)$. \\
Sea $f \en l_1 (\Omega ,\K)$ tal que $N_1 (f)=0$. Entonces \\
$\todo \alpha \en \Omega$: $| f(\alpha) \leq N_1 (f)=0$ de donde $f=0$.\\
Finalmente sean $f,g \en N_1 (\Omega , \K)$. Sean $\Omega_1 = \lbrace \alpha | f(\alpha) \neq 0 \rbrace$, 
Finalmente sean $f,g \en N_1 (\Omega , \K)$. Sean $\Omega_2 = \lbrace \alpha | g(\alpha) \neq 0 \rbrace$ y $\Omega_o =: \Omega_1 + \Omega_2$. $\Omega_o$ es a lo sumo numerable. Sea $i \flecha \alpha (i)$ una biyección de $N$ sobre $\Omega_o$. \\
Entonces \\
\begin{equation*}
\sum_{\alpha \en \Omega}(f+g)(\alpha)=\sum_{i=1}^{+\infty} (f+g) (\alpha (i))= \sum_{i=1}^{+\infty} f(\alpha (i))+ \sum_{i=1}^{+\infty}  g(\alpha (i))
\end{equation*}
$$
=\sum_{\alpha \en \Omega}f(\alpha) + \sum_{\alpha \en \Omega}g(\alpha)
$$
\phantom{sssssssssssssssssssssssssssssssssss sasdasdasdasdadadssada} c. q. d \\ \\
Dejamos al lector la tarea de demostrar que $L (\Omega ,\K)$ es un espacio de Banach. \\ \\
\textbf{Proposición 19.} \\
Si $f,g \en L_2 (\Omega,\K)$, entonces $fg \en N_1 (\Omega ,\K)$ y se tiene 
\begin{equation*}
\boxed{N_1 (fg) \leq N_2 (f) N_2(g)}
\end{equation*}

\underline{Demostración.}\\
Sean $f,g \en L_2 (\Omega ,\K)$. Por la desigualdad de Schwarz sumas finitas se tiene $\todo I \en P (\alpha)$:
\begin{equation*}
\sum_{\alpha \en I}| f(\alpha) g(\alpha)| \leq \biggl( \sum_{\alpha \en I}| f(\alpha)|^2 \biggr)^\frac{1}{2} \biggl( \sum_{\alpha \en I}| g(\alpha)|^2 \biggr)^\frac{1}{2} \leq N_2(f) N_2(g).
\end{equation*}
Esto demuestra que $fg \en L_2 (\Omega ,\K)$ y $N_1(fg) \leq N_2 (f) N_2(g)$. \\ \\
\textbf{Proposición 20.} \\
$L_1(\Omega , \K)$ es un espacio vectorial sobre \K . $\todo f,g \en L_2 (\Omega,\K)$ definimos:
\begin{equation*}
\boxed{(f|g)=: \sum_{\alpha \en \Omega}f(\alpha) \overline{g(\alpha)}}
\end{equation*}
Entonces la aplicación $(f,g) \flecha (f | g)$ es un producto escalar. Hace de $l_2 (\Omega ,\K)$ un espacio hilbertiano. $N_2$ es la norma en este espacio.\\
\underline{Demostración} \\
\begin{enumerate}[a)]
\item Si $f\en l_2 (\Omega , \K)$ y $\lam \en \K$ se tiene $\todo I \en P (\Omega):$
$$
\sum_{\alpha \en I} | \lam f(\alpha)|^2=| \lam |^2 \sum_{\alpha \en I} | f(\alpha )|^2 \leq | \lam |^2 N_2 (f)^2
$$
Luego $\lam f \en l_2 (\Omega ,K)$ \\
Si $f,g \en l_2 (\Omega , K)$ se tiene $\todo I \en P(\Omega)$:
$$
\sum_{\alpha \en I}| f(\alpha) + g(\alpha ) |^2 =\sum_{\alpha \en I}| f(\alpha )|^2+\sum_{\alpha \en I}| g(\alpha )|^2+2Re \sum_{\alpha \en I}| f(\alpha )\overline{g(\alpha)}
$$

$$
\leq \sum_{\alpha \en I}| f(\alpha )|^2+\sum_{\alpha \en I}| g(\alpha )|^2 + 2 \sum_{\alpha \en I}| f(\alpha ) g(\alpha)| 
$$

$$
\leq N_2(f)^2+N_2(g)^2+2N_1(fg).
$$
Esto muestra que $f+g \en l_2 (\Omega,\K)$. Así pues $l_2 (\Omega,\K)$ es un espacio vectorial.
\item La definición $(f|g)=: \sum_{\alpha \en \Omega}f(\alpha) \overline{g(\alpha)}$ tiene sentido pues por la prop. 19 $fg\en l_1 (\Omega,\K)$.
\end{enumerate}

Se comprueba sin dificultad que  $(f,g) \flecha (f | g)$ es un producto escalar. La norma inducida por este producto escalar es $(f|f)^\frac{1}{2}=\biggl( \sum_{\alpha \en \Omega}|f(\alpha )|^2 \biggl)^\frac{1}{2}=N_2 (f)$.\\
Queda por probar que el espacio $l_2 (\Omega , \K)$ provisto de la norma $N_2$ es completo. \\
Sea $\lbrace f_\upsilon \rbrace$ una sucesión de Cauchy en $l_2 (\Omega , \K)$. \\
Sea $\epsilon >0$ dado. Entonces:
\begin{equation}
\exists n_0 \en \N, p, q \geq n_0 \flecha N_2 (f_p - f_q) < \epsilon.
\end{equation}
Para todo $\alpha \en \Omega$ fijo se tendrá pues $|f_p (\alpha)-f_q(\alpha) | < \epsilon$ si $p,q \geq n_0$; en otras palabras la sucesión $\lbrace f_v (\alpha ) \rbrace$ es una sucesión de Cauchy en \K , luego convege a un límite que designamos por $f(\alpha)$. \\
De (2.12) se deduce la implicación: \\
$p,q \geq n_o, \phantom{s} I \en P( \Omega)\flecha \sum_{\alpha \en I}|f_p(\alpha)-f_q(\alpha)|^2 < \epsilon^2.$\\
Fijando $q \geq n_o$ y pasando al límite para $p \to +\infty$ se sigue: \\
\begin{equation}
 \sum_{\alpha \en I}|f_p(\alpha)-f_q(\alpha)|^2 \leq \epsilon^2
\end{equation}
Luego $f-f_q \en l_2 (\Omega ,\K)$, de ahí que $f \en l_2 (\Omega ,\K)$. \\
Finalmente (2.13) conseguimos 
$$
N_2 (f-f_q) \leq \epsilon
$$
Siempre que $q \geq n_o$, demostrando así que $f=\lim_{\upsilon + \infty} f_\upsilon$ en $l_2 (\Omega , \K)$.  \\
\phantom{sssssssssssssssssssssssssssssssssss sasdasdasdasdadadssada} c. q. d \\ \\
\underline{Definición.} \\
Una familia $(\overrightarrow{u_{\alpha}})_{\alpha \en \Omega}$ de vectores de un espacio prehilbertiano se llama FAMILIA ORTONORMAL si $\todo \alpha, \beta \en \Omega$ 
\begin{equation*}
(\overrightarrow{u_{\alpha}}|\overrightarrow{u_{\beta}}= \delta_{\alpha \beta}=:  \left\{ \begin{array}{lcc}
           1 & si & \alpha=\beta \\
             \\  0 & si & \alpha \neq \beta \\
             \end{array}
   \right.
\end{equation*}
Recordemos que una familia de vectores se linealmente independiente, si toda subfamilia finita de ésta es linealmente independiente. \\
 \\
 \textbf{Proposición 21.} \\
 Toda familia ortonormal en un espacio prehilbertiano es linealmente independiente.
 \\
 \underline{Demostración.}  \\
 Basta demostrarlo para una familia ortonormal finita $(\overline{u_1}, \ldots, \overline{u_n})$ . Pero si $\alpha_1 , \ldots, \alpha_r \en \K$ se tiene por el teorema de Pitágoras $|| \alpha_1 \overline{u_1}+ \ldots + \alpha_r \overline{u_r}||^2=\sum_{i=1}^r | \alpha_1|^2$, de donde la conclusión sigue inmediatamente. \\
 Recordemos que todo espacio normado de dimensión finita es completo. Luego todo espacio prehilbertiano de dimensión finita es hilbertiano. Todo subespacio de dimensión finita de un espacio prehilbertiano es completo, luego distinguido. \\ \\
 \textbf{Proposición 22.}\\
 Todo espacio hilbertiano de dimensión finita admite una base ortonormal.
 \\
 \underline{Demostración.} \\
 Razonemos por inducción sobre la dimensión $n$ del espacio $H$. Si $n=1$ y $\overline{u}$ es un vector no nulo de $H$, $\lbrace \frac{\overrightarrow{u}}{|| \overrightarrow{u} ||}$ es una base ortonormal de $H$.\\
 
 Admitamos el resultado para dimensión $n-1$ y sea $dim H=n$. Si $\overrightarrow{u}$ es un vector no nulo de $H$, sea $\overrightarrow{e_1}=: \frac{\overrightarrow{u}}{|| \overrightarrow{u} ||}$. Sea $M$ el subespacio $\overrightarrow{e_1}$ engendrado por $\overrightarrow{e_1}$. Ya que $H=M\oplus M^\perp$, $M^\perp$ es de dimensión $n-1$. Sea $(\overrightarrow{e_2}, \ldots , \overrightarrow{e_n})$ una base ortonormal de $M^\perp$. Entonces $(\overrightarrow{e_1}, \ldots , \overrightarrow{e_n})$ es una base ortonormal de $H$. \\
\phantom{sssssssssssssssssssssssssssssssssss sasdasdasdasdadadssada} c. q. d \\ \\

\textbf{Proposición 23.} \\
Sea $H$ un espacio prehilbertiano, sea $M$ un subespacio de dimensión finita de $H$ y sea $(\overrightarrow{e_1}, \ldots , \overrightarrow{e_n})$ una base ortonormal de $M$. Si $\xraya \en H$, la proyección ortogonal $\xraya_o$ de $\xraya$ sobre $M$ es:
\begin{equation*}
\boxed{\xraya_o=\sum_{i=1}^r (\xraya| \overrightarrow{e_i})\overrightarrow{e_i}}
\end{equation*} 
 
 \underline{Demostración.} \\
 Definimos $\xraya_0$ por esta fórmula. Se tiene:\\
 $$
 (\xraya-\xraya_o | \overrightarrow{e_k}=(\xraya - \sum_{i=1}^{r} (\xraya | \overrightarrow{e_i} )\overrightarrow{e_i} | \overrightarrow{e_k})= ( \xraya |  \overrightarrow{e_k})-(\xraya | \overrightarrow{e_k})=0 .
 $$
 De esto sigue inmediatamente que $\xraya- \xraya_o \perp M$ o sea $\xraya_o$ es la proyección ortogonal de $\xraya$ sobre $M$.
 \\
 \underline{Demostración.}\\
 Sea $H$ un espacio prehilbertiano y sea $(\\overrightarrow{u}_{\alpha})_{\alpha \en \Omega}$ una familia ortonormal en $H$. $\todo \xraya \en H$ definamos la función $\widehat{x}: \Omega \flecha \K$ por 
 \begin{equation*}
 \widehat{x}(\alpha)=:(\xraya|\overrightarrow{u}_{\alpha}) \phantom{s} \todo \alpha \en \Omega
 \end{equation*}
 Los números $\widehat{x}(\alpha)$ se llaman COEFICIENTES DE FOURIER de $\xraya$ con respecto a la familia ortonormal $(\overrightarrow{u}_\alpha) _{\alpha \en \Omega}$.\\ \\
 \textbf{Proposición 24.}\\
 Sea $H$ un espacio prehilbertiano y sea $(\overrightarrow{u}_\alpha) _{\alpha \en \Omega}$ una familia ortonormal en $H$. Entonces $\todo \xraya \en H$, $\widehat{x} \en l_2 (\Omega,\K)$. Además \\
 \begin{equation*}
 \boxed{N_2(\widehat{x}) \leq || \xraya ||}
 \end{equation*}
 Es decir
  \begin{equation*}
 \boxed{\sum_{\alpha \en \Omega} | \widehat{x} (\alpha) |^2 \leq || \xraya || ^2}
 \end{equation*}
 (DESIGUALDAD DE BESSEL)\\ 
 \underline{Demostración} \\
 Sea $\xraya \en H$. Sea $I \en P(\Omega)$. Sea $M_I=: L((\overrightarrow{u}_\alpha) _{\alpha \en \Omega})$ es decir, $M_I$ es el subespacio engendrado por la familia $(\overrightarrow{u}_\alpha) _{\alpha \en \Omega}$. \\
 Por la proposición 23 la proyección ortogonal de $\xraya$ sobre $M_I$ es $\xraya_o = \sum_{\alpha \en I} \widehat{x} (\alpha) \overrightarrow{u}_{\alpha}$. Luego
 $$
 \sum_{\alpha \en I} | \widehat{x}(\alpha) |^2=|| \xraya_o ||^2=|| \xraya ||^2 - d(\xraya,M_I)^2 \leq || \xraya ||^2 
 $$
 Esto demuestra que $\widehat{x} \en l_2 (\Omega,\K)$ y $N_2 (\xraya) \leq || \xraya ||$. \\ \\
 \underline{Corolario 1.} \\
 La aplicación $\xraya \flecha \widehat{x}$ es una aplicación lineal continua de $H$ en $l_2 (\Omega,\K)$ de norma $ \leq 1$. \\ \\
 \underline{Corolario 2.}\\
 Si $(\overrightarrow{u}_\alpha)_{\alpha \en \Omega}$ es una familia ortonormal en $H$ y $\xraya \en H$, el conjunto $\lbrace \alpha | \alpha \en \Omega, \phantom{s} (\xraya | \overrightarrow{u}_\alpha)\neq 0 \rbrace$ es a lo suma numerable. \\ \\
 \textbf{Proposición 25.} \underline{(Teorema de RIESZ-FISCHER).}\\
 Sea $H$ un espacio prehilbertiano y sea $(\overrightarrow{u}_\alpha)_{\alpha \en \Omega}$ una familia ortonormal en $H$. Se supone que el subespacio cerrado $M=\overline{L ((\overrightarrow{u}_\alpha)_{\alpha \en \Omega})}$ es completo. (Esto será siempre el caso si $H$ es hilbertiano). \\
 Entonces la aplicación $\xraya \flecha \widehat{x}$ es una superyección de $H$ sobre $l_2 (\Omega,\K)$.\\
 
Más precisamente $\todo \varphi l_2 (\Omega,\K)$ existe un único $\xraya_o \en M$ tal que $\widehat{x_o}=\varphi$. Si $\xraya \en H$, se tiene $\widehat{x}=\varphi$ si y sólo si $\xraya -\xraya_o \perp M$.\\
\underline{Demostración.} \\
\begin{enumerate}[a)]
\item Sea dado $\psi \en l_2 (\Omega ,\K)$. Sea $\Omega_o = \lbrace \alpha | \alpha \en \Omega, \phantom{s} \psi (\alpha) \neq 0 \rbrace$. $\Omega_o$ es a lo sumo numerable. Haremos la demostración en el caso más difícil de ser $\Omega_o$ numerable. Sea $i \flecha \alpha (i)$ una biyección de $\N$ sobre $\Omega_o$. Se tiene
\begin{equation}
\sum_{i=1}^{+\infty}| \psi (\alpha (i))|^2=N_2(\psi )^2 < \infty;
\end{equation}
Si $q>p: || \sum_{i=p}^q \psi (\alpha (i)) \overrightarrow{u}_{\alpha (i)}||^2=\sum_{i=p}^q | \psi (\alpha (i)) |^2$ \\
En virtud de (2.14): $\lim_{p,q \to +\infty} \sum_{i=p}^q | \psi (\alpha (i))|^2=0$.\\
Se cumple pues la condición de Cauchy para la convergencia de la serie $ \lbrace \lbrace \psi (\alpha (i)) \overrightarrow{u}_{\alpha (i)} \rbrace \rbrace_{i \en \N}$ en $M$. \\
Siendo $M$ completo existe \\
$$
\xraya_o =: \sum_{i=1}^{+\infty} \psi (\alpha (i)) \overrightarrow{u}_{\alpha (i)}, \phantom{s} '\xraya_o \en M
$$
Fijemos $k \en N$. $\todo m \geq k$ se tiene 
$$
\biggl( \sum_{i=1}^m \psi(\alpha (i)) \overrightarrow{u}_{\alpha (i)} | \overrightarrow{u}_{\alpha (k)} \biggl)= \psi (\alpha (k))
$$
 
 
 Pasando aquí al límite para $m \to +\infty$ conseguimos:
 $$
 (\xraya_o | \overrightarrow{u}_{\alpha (k)}=\psi (\alpha (k))
 $$
o sea \\
$$
\widehat{x_o}(\alpha )= \psi (\alpha) \phantom{s} \todo \alpha \en \Omega_o.
$$ 
Pero si $\alpha \en \Omega - \Omega_o$ se tiene $\todo m$: 
$\biggl( \sum_{i=1}^m \psi(\alpha (i)) \overrightarrow{u}_{\alpha (i)} | \overrightarrow{u}_{\alpha (k)} \biggl)=0 $, de donde, pasando al límite para $m \to +\infty$
$$
\widehat{x_o}(\alpha)= (\xraya_o | \overrightarrow{u}_\alpha )=0=\psi (\alpha).
$$
 Finalmente $\todo \alpha \en \Omega$: 
 $$
 x_o (\alpha )= \psi (\alpha).
 $$
 \item  Sea $\xraya \en H$ tal que $x=\widehat{\psi}$. Entonces $\widehat{x-x_o}=0$ o sea $(\xraya-\xraya_o |\overrightarrow{u}_{\alpha})=0$ $\todo \alpha \en \Omega$. Luego $\xraya-\xraya_o \perp L (\overrightarrow{u}_{\alpha})_{\alpha \en \Omega}).$  Ya que todo elemento de $M$ es límite de una sucesión de elementos de $L (\overrightarrow{u}_{\alpha})_{\alpha \en \Omega})$ se sigue que $\xraya-\xraya_o \perp M$. Viceversa si  $\xraya-\xraya_o \perp M$ es inmediato que $\widehat{x}=\widehat{\psi}$. \\
  
 \end{enumerate}
 Si en particular $\xraya \en M$, la condición $\xraya - \xraya_o \perp M$ implica $\xraya=\xraya_o$, de donde la unicidad de $\xraya_o$. \\
 \phantom{sssssssssssssssssssssssssssssssssss sasdasdasdasdadadssada} c. q. d \\ \\
 \underline{Observación.} \\
 La unicidad de $\xraya_o$ implica que $\sum_{i=1}^{+\infty} \psi (\alpha (i))  \overrightarrow{u}_{\alpha (i)}$ es independiente de la biyección $\alpha$ de $\N$ sobre $\Omega_o$. Designaremos esta suma simplemente por $\sum_{\alpha \en \Omega} \psi (\alpha )  \overrightarrow{u}_\alpha$. Podemos pues escribir:
 \begin{equation*}
 \boxed{\xraya_o=\sum_{\alpha \en \Omega} \psi (\alpha)\overrightarrow{u}_{\alpha} }
 \end{equation*}
 \underline{Definición.}\\
 Una familia ortonormal en un espacio prehilbertiano se llama FAMILIA ORTONORMAL MAXIMAL, si no existe una familia ortonormal que la contenga propiamente. \\
 Una condición equivalente es la siguiente: una familia ortonormal $(\overrightarrow{u}_{\alpha})_{\alpha \en \Omega}$ es maximal si y sólo si:
 $$
 \xraya \perp \overrightarrow{u}_{\alpha} \phantom{s} \todo \alpha \en \Omega \flecha \xraya =0
 $$
 \underline{Lema.} \\
 Sean $H$, $H'$ espacios prehilbertianos y sea $L$ una isometría lineal de $H$ en $H'$. Entonces $L$ preserva los productos escalares, es decir: \\
$$
(L\xraya | L\yraya)= (\xraya | \yraya ) \phantom{s} \todo \xraya , \yraya \en H
$$
 \underline{Demostración.} \\
Sean $\xraya , \yraya H$. Se tiene $\todo \lam \en \K$: \\
 \begin{equation}
 || \lam \xraya +\yraya ||^2= | \lam |^2 || \xraya ||^2 + || \yraya ||^2 + 2Re (\lam (\xraya | \yraya ))
 \end{equation}
 
 \begin{equation}
|| L (\lam \xraya + \yraya ) ||^2= || \lam L\xraya + L\yraya ||^2=| \lam |^2 || \L\xraya ||^2 + ||L\yraya ||^2
 \end{equation}
 $$
 +2Re(\lam(L\xraya | L\yraya))
 $$
Los primeros miembros de (2.15) y (2.16) son iguales por hipótesis. Igualmente los últimos miembros y sirviéndose de nuevo de la hipótesis se obtiene: 
$$
Re (\lam (\xraya | \yraya))=Re(\lam (L \xraya | L \yraya ))
$$
Haciendo aquí primero $\lam=1$, luego $\lam=-i$ se obtiene:
$$
Re(\xraya | \yraya)=Re(L \xraya | L \yraya ), \phantom{s} Im(\xraya | \yraya)=Im (L \xraya | L \yraya )$$
De ahí que $(\xraya | \yraya )=(L \xraya | L \yraya )$\\
 \phantom{sssssssssssssssssssssssssssssssssss sasdasdasdasdadadssada} c. q. d \\ \\
 
\textbf{Proposición 26.}
Sea $H$ un espacio hilbertiano. Sea $(\overrightarrow{u}_\alpha)_{\alpha \en \Omega}$ una familia ortonormal en $H$. Entonces las siguientes condiciones son equivalentes a pares:
\begin{enumerate}[1)]

\item  $(\overrightarrow{u}_\alpha)_{\alpha \en \Omega}$ es maximal.
\item $\overline{L ( (\overrightarrow{u}_\alpha)_{\alpha \en \Omega})}=H$.
\item $\todo \xraya \en H$, $N(\widehat{x})= || \xraya ||$ osea $\sum_{\alpha \en \Omega}| \widehat{x}(\alpha)|^2=|| \xraya ||^2$
\item $\todo \xraya,\yraya \en H$, $(\widehat{x} | \widehat{y})=(\xraya | \yraya)$. \\
Osea $\sum_{\alpha \en \Omega} \overline{\widehat{y}(\alpha)}=(\xraya | \yraya).$

\end{enumerate}
La identidad 3) o lo equivalente identidad 4) se llama IDENTIDAD DE PARSEVAL. \\
\underline{Demostración.}
\begin{enumerate}[a)]
\item Probemos $1) \to 2)$. Supongamos que 2) es falso y que $M=:\overline{L ( (\overrightarrow{u}_\alpha)_{\alpha \en \Omega})}$. Entonces $M \neq H$. Sea $\xraya \en H$, $\xraya \not \in M$. Puesto que $H$ es hilbertiano, el subespacio cerrado $M$ es distinguido. Sea $\xraya_o$ la proyección ortogonal de $\xraya$ sobre $M$. Se tiene $\xraya - \xraya_o \neq 0$, pero $(\xraya-\xraya_o | \overrightarrow{u}_\alpha)=0$ $\todo \alpha \en \Omega$. Luego 1) es falso.

\item Probemos $2) \to 3)$. Supongamos que se cumple 2) sea dado $\xraya \en H$ y sea $\epsilon >0$ arbitrario. Por hipotesis exste $I \en P (\Omega)$ y un elemento $\sum_{\alpha \en I} \lam _\alpha \overrightarrow{u_\alpha}$ tal que $|| \xraya - \sum_{\alpha \en I} \lam_\alpha \overrightarrow{u}_\alpha || < \epsilon$. Sea $M_I$ el subespacio de dimension finita engendrado por $(\overrightarrow{u}_\alpha)_{\alpha \en \Omega})$. Entonces $d(\xraya ,M_I) < \epsilon$. \\
Por prop. 23 la proyeccion ortogonal $\xraya_I$ de $\xraya$ sobre $M_I$ es $\xraya_I=\sum_{\alpha \en I} x(\alpha) \overrightarrow{u}_\alpha$. Luego
$$
d(\xraya,M_I)^2=|| \xraya ||^2 - || \xraya_I ||^2 < \epsilon^2
$$
De donde 
$$
N_2(\widehat{x})^2 \geq \sum_{\alpha \en I}| \widehat{x}(\alpha)|^2=||\xraya ||^2>|| \xraya ||^2-\epsilon^2.
$$
Puesto que $\epsilon$ es arbitrario, se sigue $N_2(\widehat{x}) \geq || \xraya ||$. Combinando con la desigualdad de Bessel $N_2(\widehat{x}) \leq || \xraya ||$, se obtiene bien $N_2(\widehat{x})=|| \xraya ||$.
\item $3) \to 4)$ por el lema que precede esta proposición. 
\item Mostremos $4)\to 1)$. Supongamos que 1) es falso. Entonces existe $\xraya \en H$ tal que $\xraya \neq 0$ y $\todo \alpha \en \Omega:$ $\widehat{(\alpha)}=(\xraya | \overrightarrow{u}_{\alpha}=0$. Luego $(\widehat{x} | \widehat{x})= \sum_{\alpha \en \Omega} \widehat{x}(\alpha) \overline{\widehat{x}(\alpha)=0}$, pero $(\xraya | \xraya) \neq 0$. Así pues también 4) es falso. 
\end{enumerate}
 \phantom{sssssssssssssssssssssssssssssssssss sasdasdasdasdadadssada} c. q. d \\ \\
 
 \underline{Observación.}\\
 Resulta de la prop. 26 que si $(\overrightarrow{u}_\alpha)_{\alpha \en \Omega})$ es una familia ortonormal maximal en un espacio hilbertiano $H$, aplicación $\xraya \to \widehat{x}$ es una isometría de $H$ en $l_2(\Omega,\K)$. Pero por el teorema de Riesz-Fischer dicha aplicación es superyactiva. Así pues:
\underline{La aplicación $\xraya \to \widehat{x}$ es una isometría de $H$ sobre $l_2(\Omega,\K)$.} \\
En virtud de la observación después de la prop. 25, conociendo $\widehat{x}$ se obtiene $\xraya$ por la fórmula
\begin{equation*}
\boxed{\xraya=\sum_{\alpha \en \Omega} \widehat{x}(\alpha) \overrightarrow{u}_\alpha}
\end{equation*}
\underline{... de familias ortonormales maximales.}\\
Vamos a necesitar conceptos de la teoría de conjuntos que procedemos a recordar: \\
Una relación $"\prec"$ en un conjunto $S$ se llama RELACIÓN DE ORDEN, si  
\begin{enumerate}[1)]
\item $x\prec x$ $\todo x \en S$ (reflexividad)
\item $x\prec x$ e $y\prec z \to x\prec z$
\item $x\prec x$ e $y\prec x \to x =y$ \\
La relación  de orden se dice RELACIÓN DE ORDEN TOTAL si cumple también la condición: 
\item $\todo x,y \en S$ se tiene por lo menos una de las relaciones  $x\prec x$ ó $y\prec x$.

\end{enumerate}
Si $" \prec "$ es una relación de orden, resp. una relación de orden total, en $S (S, \prec)$ se dice CONJUNTO ORDENADO resp. TOTALMENTE ORDENADO. \\
P. ej. la relación $\subset$ en el conjunto de todas las partes de cierto conjunto $X$ es una relación de orden "parcial" (es decir no total). La relación $\leq$ en \R \phantom{s} es una relación de orden total. \\
Si $X$ es un subconjunto de un conjunto ordenado $S$, un elemento $a$ de $S$ se llama MAYORANTE del conjunto $X$ si $x \prec a$ $\todo x \en X$. \\
Un subconjunto $X$ de $S$ se llama una CADENA si la relacion $\prec$ da a $X$ hace de $X$ un conjunto totalmente ordenado. (En otras palabras $\todo x, y \en X$ se tiene por lo menos una de las relaciones $x \prec y$ ó $y \prec x$). \\
Un elemento $a \en S$ se dice ELEMENTO MAXIMAL si $a \prec x \to x=a$. Finalmente un conjunto ordenado $S$ se dice INDUCTIVO si toda cadena en $S$ admite un mayorante. \\
Recordemos el siguiente principio de la teoría de conjuntos: \\
\underline{Lema de ZORN.} \\
Si $S$ es un conjunto ordenado inductivo, existe por lo menos un elemento maximal en $S$. \\ \\
\textbf{Proposición 27.}
En todo espacio prehilbertiano existe una familia ortonormal maximal. \\
\underline{Demostración.}\\
Consideremos el conjunto $\Re$ de todas las familias ortonormales en el espacio prehilbertiano $H$. La relación de inclusión $"\subset "$ es una relación de orden en $\Re$. \\
Sea $(P_i)_{i \en I}$ una cadena en $\Re$. Es decir $(P_i)_{i \en I}$ es una familia de sisrtemas ortonormales $P_i$, tal que $\todo i ,j \en I$ se tiene $P_i \subset P_j$ ó $P_j \subset P_i$. Pongamos $P=: u_{i \en I}P_i$. Sean $\overrightarrow{u}_\alpha,\overrightarrow{u}_\beta \en P$. \\
Puesto que $(P_i)_{i \en I}$ es una cadena, existe un indice i tal que ... pertenecen a $P_i$. Luego $(\overrightarrow{u}_\alpha | \overrightarrow{u}_\beta)=...$. Se sigue que $P$ es un sistema ortonormal: $P \en \Re$. $P$ es evidentemente mayorante de la cadena $(P_i)_{i \en I}$. Así demostramos que $\Re$ es un conjunto ordenado inductivo. \\
Por el lema de Zorn, existe en $\Re$ un elemento maximal $g$. $g$ es una familia ortonormal maximal en $H$. \\
 \phantom{sssssssssssssssssssssssssssssssssss sasdasdasdasdadadssada} c. q. d \\ \\
 
\textbf{Proposición 28.} \\
Todas las familias ortonormales maximales de un espacio hilbertiano $H$ tienen la misma cordinalidad.  \\
\underline{Demostración} \\
Sean $A,B$ dos familias ortonormales maximales en un espacio hilbertiano $H$. 
\begin{enumerate}[a)]
\item Supongamos que $A$ es finito: sea $A= \lbrace \uraya_1,\ldots, \uraya_n \rbrace$. Por la prop. 26, $H= \overline{L (\uraya_1,\ldots, \uraya_n)}=L(\uraya, \ldots, \uraya)$. Luego $H$ es de dimensión $n$ y una base de $H$. Puesto que $B$ es linealmente independiente, $B$ es también finito y es otra base de $H$. Por lo tanto 
$$
card (A)=card(B)
$$
\item Supongamos ahora $A$ y $B$ infinitos. \\
... sea $B_{\overrightarrow{a}}=:\lbrace \overrightarrow{b} | \overrightarrow{b} \en B, (\overrightarrow{b}| \overrightarrow{a}) \neq 0 \rbrace.$ Sabemos que $B_{\overrightarrow{a}}$ es a lo sumo numerable. \\
Si $\overrightarrow{b} \en B$ se tiene por la identidad de Parseval
\end{enumerate}
$l= || \overrightarrow{b} ||^2=\sum_{\overrightarrow{a} \en A}(\overrightarrow{b}|\overrightarrow{a})|^2$, luego $ \exists \overrightarrow{a} \en A \pitchfork \overrightarrow{b} \en B_{\overrightarrow{a}}$. \\
Se sigue $B=\cup{\overrightarrow{a} \en A}B_{\overrightarrow{a}}$, de ahí que:
$$
card(B) \leq card(A) N_o=card(A)
$$
 \phantom{sssssssssssssssssssssssssssssssssss sasdasdasdasdadadssada} c. q. d \\ \\
 
 \textbf{Proposición 29.} \\
 Todo espacio hilbertiano es isométrico a un espacio $l_2(\Omega, \K)$. Los espacios $l_2 (\Omega,\K)$ y $l_2 (\Omega',\K)$ son isométricos si y sólo si $\Omega, \Omega'$ tienen la misma cardinalidad. \\
 \underline{Demostración.}\\
 Sea $H$ un espacio hilbertiano. Por la prop. 27 existe en $H$ una familia ortonormal maximal $(\uraya_\alpha)_{\alpha \en \Omega}$. Por la observación después de la prop. 26, $H$ es isométrico a $l_2(\Omega,\K)$. \\
 Sea $\Omega$ un conjunto. $\todo \alpha \en \Omega$ sea $\varphi_\alpha$ la aplicacion $\Omega \flecha \K$ dada por $\varphi_\alpha (\beta)=...$ $\todo \beta \en \Omega$. Se comprueba inmediatamente que $(\varphi_\alpha)_{\alpha \en \Omega}$ es una familia ortonormal maximal en $l_2 (\Omega, \K)$ de la misma cardinalidad de $\Omega$. Si  $\Omega$ y $\Omega'$ son conjuntos de cardinalidades distintas, se sigue pues de la prop. 28 que $l_2 (\Omega, K)$ y $l_2 (\Omega ',K)$ no pueden ser isométricos.\\
 Finalmente supongamos $card \Omega=card \Omega '$. Sea $h$ una biyección de $\Omega$ sobre $\Omega '$. A todo $f \en l_2(\Omega ',\K)$ asociémosle la función $f_o h: \Omega \flecha \K$. Se comprueba inmediatamente que la aplicación $f \flecha f_o h$ es una isometría de $l_2 (\Omega ',\K)$ sobre $l_2 (\Omega,\K)$. \\
  \phantom{sssssssssssssssssssssssssssssssssss sasdasdasdasdadadssada} c. q. d \\ \\
 Recordemos que un espacio métrico $M$ se llama SEPARABLE, si existe en $M$ un subconjunto numerable denso.  \\ \\
 \textbf{Proposición 30.} \\
 Para que un espacio hilbertiano $H$ sea separable es necesario y suficiente que exista en $H$ una familia ortonormal maximal a lo sumo numerable. \\
 \underline{Demostración.}  \\
 
 \begin{enumerate}[a)]
 \item Supongamos que existe en $H$ una sucesión $(\uraya_i)_{i \en \N}$ ortonormal maximal. Un número complejo $a$ se dirá racional si $Re$ $a$ e $Im a \en \Q$. Sea $s$ el conjunto de todas las combinaciones lineales finitas, de los $\uraya_i$ con coeficientes racionales, es decir, el conjunto de todos los vectores $\sum_{i \en \N}\alpha_i \uraya_i$ donde $\alpha_i$ es racional $\todo i$ y $\alpha_i=0$ salvo para un número finito de los índices. $S$ es un subconjunto numerable de $H$. Para mostrar que $H$ es separable basta probar que $\overline{S}=H$. \\
 Sea $\xraya \en H$ y sea $\epsilon >0$. Por la prop. 26 existe $k \en N$ y una sucesión $(\lam_1,\ldots, \lam_k)$ de elementos de $\K$ tal que:
  \begin{equation}
  || \xraya - \sum_{i=1}^k \lam_i \uraya_i || < \frac{\epsilon}{2}
  \end{equation}
 Para $i=1,\ldots, k$ existe un elemento racional $\alpha_i$ de $\K$ tal que $| \lam_i - \alpha_i| < \frac{\epsilon}{2k}$. Se tendrá pues: 
\begin{equation}
  || \sum_{i=1}^k \lam_i \uraya_i - \sum_{i=1}^k \alpha_i \uraya_i || \leq \sum_{i=1}^k | \lam_i - \alpha_i | < \frac{\epsilon}{2}
\end{equation} 
De (2.17) y (2.18) se sigue: \\
$|| \xraya - \sum_{i=1}^k \alpha \uraya_i || < \epsilon$. Luego $\overline{S}=H$. 
 \item Recíprocamente supongamos que $H$ es separable. Sea $(\xraya_i)_{i\en N}$ una sucesión densa en $H$. Sea $(\uraya_\alpha)_{\alpha \en \Omega}$ un sistema ortonormal maximal en $H$. \\
 $\todo i \en N$ definamos $A_I=: \lbrace \alpha | \alpha \en \Omega, (\uraya_\alpha | \xraya_i) \neq 0 \rbrace$.\\
 Sabemos que $A_i$ es a lo sumo numerable. Sea $A=: \cup_{i=1}^{+\infty}A_i$. \\
 Entonces $A$ es también a lo sumo numerable. Bastará demostrar que $A= \Omega$.\\
 En efecto sea $\alpha \en \Omega$. Puesto que $(\xraya_i)_{i \en \N}$ es denso en $H$, existe $i \en N$ $|| \uraya_\alpha - \xraya_i || <1$. Si tuviéramos $(\uraya_\alpha | \xraya )=0$, seguiría del teorema de Pitágoras: $|| \uraya_\alpha - \xraya_i ||^2 \leq || \uraya_\alpha ||^2 + || \xraya || ^2=1 +|| \xraya ||^2 \geq 1$ que es una contradicción. Luego $(\uraya_\alpha | \xraya_i ) \neq 0$ o sea $\alpha \en A_i \subset A$.
 Así pues $A=\Omega$. \\
  \phantom{sssssssssssssssssssssssssssssssssss sasdasdasdasdadadssada} c. q. d \\ \\
 \end{enumerate}
 \underline{Observación. } \\
 Combinando la prop. 30. con la observación después de la prop. 26 obtenemos el resultado siguiente:\\
 Un espacio hilbertiano de dimensión infinita es separable si sólo si es isométrico a $l_2(\N, \K)$.\\
 Este espacio, llamado simplemente $l_2$, fue él considerado inicialmente por Hilbert. \\
 Si $(\uraya_i)_{i \en N}$ es una familia ortonormal maximal en $H$, una isometría de $H$ sobre $l_2$ está por $\xraya \flecha \xraya$ donde $\widehat{x}(i)=(\xraya | \uraya_i)$ $\todo i \en \N$. La isometría inversa es
 $$
 \varphi \flecha \sum_{i=1}^{+\infty} \varphi (i) \uraya_i
 $$
\underline{Separabilidad de $L_2 (S,\K)$.}\\
Vamos a mostrar, más generalmente:\\ \\
\textbf{Proposición 31.} \\
Sea $S$ un subconjunto medible de \Rn . Para $1 \leq p < \infty$, $L_p (S,\K)$ es un espacio de Banach separable. \\
\underline{Demostracion. }\\
Se puede considerar $L_p (S,\K)$ como subespacio de $L_p(\Rn,K)$. Ya que todo subespacio de un espacio métrico separable es separable, basta probar que, para $1 \leq p < \infty$, $L_p(\Rn, \K)$ es separable. \\
Sea $P=I_1x \cdots x I_n$ un tarugo acotado en \R . Sean $a_k , b_k$ los extremos del intervalo $I_k$ para $k=1, \ldots , n$. $P$ se llamará un tarugo racional, si todos los números $a_k$ y todos los números $b_k$ son racionales. Una función escalonada $\sum_{i=1}^r \alpha_i x_{P_i}$ se llamará función escalonada racional si todos los tarugos $P_i$ y todos los $\alpha_i$ son racionales. Sea $\epsilon '$ el conjunto de todas las funciones escalonadas \Rn \flecha \K . $\epsilon '$ es un conjunto numerable. Bastará demostrar que  $\epsilon '$ es denso en $L_p (\Rn , \K)$. \\
Sea dada $f \en L_p (\Rn, \K)$ y sea $\epsilon >0$ dado. Ya que el espacio, de las funciones escalonadas es denso en $L_p (\Rn, \K)$ (prop. 15 del cap. VII) $\exists \varphi \en \epsilon$ tal que: 
\begin{equation}
N_p(f-\varphi)< \frac{\epsilon}{2}
\end{equation}
Sea $\varphi=\sum_{k=1}^r \lam_k x_{Q_k}$ donde $\lam_k \en \K$, $\lam_k \neq 0$ y $Q_k$ son tarugos acotados disjuntos a pares. Sea, para $k=1, \ldots ,r, P_k$ un tarugo racional de volumen positivo que contiene $Q_k$ tal $vol (P_k)-vol(Q_k) < \biggl( \frac{\epsilon}{4r | \lam_k|} \biggl)^p$ y sea $\alpha_k$ un elemento racional de $\K$ tal que $| \alpha_k - \lam_k | < \frac{\epsilon}{4r N_p(x_{P_k})}$. Notemos que  $N_p(x_{P_k}-x_{Q_k})=(vol(P_k)-vol(Q_k)^{\frac{1}{p}}<\frac{\epsilon}{4r | \lam_k |}.$ \\
Sea $\Psi=: \sum_{k=1}^r \alpha_k x_{P_k}.$ Entonces $\Psi \en \epsilon '$ y se tiene:
\begin{equation}
N_p (\Psi - \varphi)=N_p(\sum_{k=1}^r \alpha_k x_{P_k}-\sum_{k=1}^r \lam_k x_{Q_k}=
\end{equation}
 
\begin{equation*}
=N_p(\sum_{k=1}^r (\alpha_k - \lam_k )x_{P_k} + \sum_{k=1}^r \lam_k (x_{P_k} - x_{Q_k})) \leq 
\end{equation*} 

\begin{equation*}
\leq \sum_{k=1}^r | \alpha_k - \lam_k | N_p (x_{P_k})+ \sum_{k=1}^r | \lam_k | N_p (x_{P_k}-x_{Q_k}) < \frac{\epsilon}{4}+ \frac{\epsilon}{4}=\frac{\epsilon}{2}
\end{equation*}

De (2.19) y (2.20) se sigue. \\
$N_p(f-\Psi) < \epsilon$, o sea $\epsilon ' $ es denso en $L_{p '}$. \\
  \phantom{sssssssssssssssssssssssssssssssssss sasdasdasdasdadadssada} c. q. d \\ \\
 
 \subsection{$L_\infty$ como dual de $L_1$}
 \underline{Lema.} \\
 Sea $f:\Rn \flecha \F$ una función medible. Entonces existe un subconjunto despreciable $Z$ de $\Rn$ tal que $f(\Rn - Z)$ es separable. \\
 \underline{Demostración.} \\
 Por definición de una función medible existe una sucesión $\lbrace \varphi_\upsilon \rbrace$ de funciones escalonadas $\Rn \flecha \F$ y un subconjunto despreciable $Z$ de $\Rn$ tal que $f(x)= \lim_{\upsilon \to +\infty}\varphi_\upsilon (x)$ $\todo x \en \Rn - Z$.\\
Sea $S_\upsilon =: \varphi_\upsilon (\Rn)$ y $S=: \cup_{\upsilon = 1}^{+ \infty}S_\upsilon \cdot S_\upsilon$ es finito, luego $S$ es numerable. Por lo tanto $\overline{S}$ es separable. Si $x \en \Rn -Z$, entonces $f(x) \en \overline{S}$, luego $f(\Rn -Z ) \subset \overline{S}$. Ya que todo subespacio de un espacio métrico separable, es separable, $f(\Rn -Z )$ es separable, c. q. d.
\\  \\
\underline{Corolario.} \\
Si $f: \Rn \flecha \F$ es una función medible, existe una función $g: \Rn \flecha \F$ tal que $g=f$ c. t. p. y $g(\Rn)$ es separable. \\ \\
\textbf{Proposición 32.} ("LEMA DE LOS PROMEDIOS"). \\
Sea $E$ un subconjunto medible de \Rn . Sea $f:E \flecha \F$ una función medible sobre $E$ integrable en todo subconjunto de $E$ de medida finita. Sea $S$ un subconjunto cerrado de $\F$. Se supone que para todo subconjunto medible $A$ de $E$ tal que $0 < vol (A) < \infty $ se tiene $\frac{1}{vol (A)}\int_A f \en S$. Entonces $f(x) \en S$ para casi todo $x\en E$. \\
\underline{Demostración. } \\
\begin{enumerate}[a)]
\item Sea $C$ una bola cerrado de radio positivo en $F$ tal que $C_{\cap} S=\varnothing$. Sea $\uraya$ el centro, $r$ el radio de $C$. Afirmamos que el conjunto $f^{-1}(C)= \lbrace x|x \en E, || f(x)- \uraya || \leq r \rbrace$ es despreciable. \\
Sea $P_k$ el cubo $\lbrace x | -k \leq x_i \leq k, i=1, \ldots, n \rbrace$. Puesto que $E= \cup_{k=1}^{+ \infty} (E_{\cap}P_k)$, basta demostrar que el conjunto $A_k =: f^{-1} (C )_{\cap}P_k$ es despreciable $\todo k$. \\
Ahora bien $A_k$ es un conjunto medible de medida finita. Si fuera $vol(A_k) \neq 0$ se tendría por hipótesis:
\begin{equation}
\frac{1}{vol (A_k)}\int_{A_k}f \en S
\end{equation}
Pero 
$$
|| \frac{1}{vol(A_k)}\int_{A_k}f - \uraya ||=||  \frac{1}{vol(A_k)}\int_{A_k}(f-\uraya) || \leq 
$$

$$
\leq  \frac{1}{vol(A_k)}\int_{A_k} | f-\uraya | \leq r
$$
Ya que $f(x) \en C$ $\todo x \en A_k$. Se tendría pues: \\
$ \frac{1}{vol(A_k)}\int_{A_k}f \en C$, en contradicción con (2.21).
Así, pues $vol(A_k)=0$, como afirmamos. 
\item Valiéndonos del corolario al lema precedente, podemos suponer sin pérdida de generalidad que $f(E)$ es separable. También el conjunto $f(E)-S$, el subconjunto de éste, es separable. $\todo \uraya \in f(E)-S$ existe una bola cerrada $C_{\uraya}$ de centro $u$, de radio positivo, tal que $C_{\uraya} \cap S=\varnothing$. La familia de abiertos $(C_{\uraya})_{\uraya \en f(E)-S}$ recubre $f(E)-S$. Pero sabemos que todo recubrimiento abierto de un conjunto separable contiene un subrecubrimiento numerable ("Teorema de Lindelof"). Sea pues $(C_\upsilon)_{\upsilon \en N}$ un subrecubrimiento numerable del recubrimiento precedente. También las bolas cerradas $C_\upsilon$ recubren $f(E)-S$. Se tiene pues:
$$
f^{-1}(F-S)=f^{-1}(f(E)-S)=f^{-1} (\cup_{\upsilon=1}^{+\infty}f^{-1}(C_\upsilon)
$$

\end{enumerate}
 Pero por la parte a) de esta demostración sabemos que $f^{-1}(C_\upsilon)$ es un conjunto despreciable. Luego $f^{-1}(\F -S)$ es un conjunto despreciable. \\
  \phantom{sssssssssssssssssssssssssssssssssss sasdasdasdasdadadssada} c. q. d \\ \\
   \underline{Corolario 1.}\\
Sea $f:E \flecha \F$ como en la prop. 32. Si $\int_A f=0$ para todo subconjunto medible $A$ de $E$ de medida finita, entonces $f=0$, c. t. p. \\
Este corolario es un caso particular del siguiente: \\ \\
\underline{Corolario 2.} \\
Sea $f:E \flecha \F$ como en la prop. 32. Si existe un número $b$ tal que $|| \int_A f|| \leq b \phantom{s} vol(A)$ para todo subconjunto medible $A$ de $E$ de medida finita, entonces $| f | \leq b$, c. t. p. \\ \\
\textbf{Proposición 33.}\\
Sea $E$ un subconjunto medible de \Rn . Para todo $g \en L_\infty (E, \K )$ la aplicación $\Phi_g : f \flecha \int_E fg$ es una forma lineal continua sobre $L_1 (E,\K)$. La aplicación $g \flecha \Phi_g$ es una isometría de $L_\infty (E,\K)$ sobre el dual topológico $L_1 (E, \K)^*$ de $L_1 (E, \K)$. \\
\underline{Demostración.} \\
\begin{enumerate}[a)]
\item Sea $g \en L_\infty (E,\K)$. Pongamos $\Phi_g (f)=: \int_E fg$ $\todo f \en L_1 (E,\K)$. Es claro que $\Phi_g (f)$ depende considerar $\Phi_g$ como una forma lineal sobre $L_1 (E,\K)$. Se tiene $| \Phi_g (f)| \leq N_1 (f) N_\infty (g)$. Luego $\Phi$ es una forma lineal continua sobre $L_1$, siendo: 
\begin{equation}
|| \Phi_g || \leq N_\infty (g)
\end{equation}
También para todo subconjunto integrable $A$ de $E$ se verifica 
$$
|| \int_A g |=| \int_E x_A g |=| \Phi_g (x_A)| \leq || \Phi_g || N_1 (x_A)= || \Phi_g || \phantom{s} vol(A)
$$
Luego, por el corolario 2. del lema de los promedios:\\
$|g(x)| \leq || \Phi_g ||$ para casi todo $x$ en $E$. Esto entraña:\\
\begin{equation}
N_\infty(g) \leq || \Phi_g ||
\end{equation}
De (2.22) y (2.23) deducimos:
$$
|| \Phi_g || =N_\infty (g)
$$
Patentamente $\Phi_g$ depende solamente de la clase de equivalencia de $g$. Podemos pues considerar la aplicación $g \flecha \Phi_g$ como una aplicación lineal de $L_\infty$ en $L_1^*$. Por lo demostrado dicha  aplicación lineal de $L_\infty$ en $L_1^*$. (En particular es automáticamente inyectiva). 
\item  Hagamos la hipótesis adicional de que $E$ es de medida finita y probemos que en este caso nuestra isometría es superyectiva. \\
Sea dada $\Phi \en L_1^*$. Puesto que en nuestro caso $L_2 (E,\K) \subset L_1 (E,\K)$ (prop. 22 del cap. VII)., $\Phi |_{L_2}$, la restricción de $\Phi$ a $L_2$ es una forma lineal sobre $L_2$. Siendo además $\frac{1}{vol (E)}N_1 (f) \leq \frac{1}{(vol(E))^\frac{1}{2}}N_2(f)$ por la prop. 22. del cap. VII, se tiene $\todo f \en L_2:$

$$
| \Phi (f) | \leq || \Phi || N_1 (f) \leq || \Phi || (vol(E))^\frac{1}{2}N_2(f).
$$
Esto prueba que $\Phi |_{L_2}$ es una forma lineal continua sobre el espacio de Hilbert $(L_2, N_2)$. Por la prop. 16 ("autodualidad de $L_2$") existe un único $g \en L_2$ tal que: 
\begin{equation}
\Phi (f)=\int_E fg \phantom{s} \todo f \en L_2.
\end{equation} 
Afirmamos que de hecho $g \en L_\infty$. (Recuerde que en nuestro caso $L_\infty \subset L_2$). \\
En efecto, tomando por $f$ en (2.24) la función característica $x_A$ de un subconjunto integrable de $E$, se obtiene:
 $|\int_A g |=|\Phi (x_A)| \leq || \Phi || \phantom{s} vol(A)$, de ahí que, por el corolario 2. del lema de los promedios: $|g| \leq || \Phi ||$ c. t. p en $E$. Esto prueba bien que $g 
 \en L_\infty$. \\
 Ahora los dos miembros de (2.24) están definidos $\todo f \en L_1$. Tanto $\Phi$ como $\Phi_g:\flecha \int_E fg$ son formas lineales continuas sobre $L_1$. La relación (2.24) es cierta en particular si $f$ es una función medible acotada. Como el espacio de las funciones medibles acotadas es denso en $L_1$, por continuidad la relación (2.4) es cierta $\todo f \en L_1$. Así probamos que $\Phi=\Phi_g$. 
 \item Abandonemos ahora la hipótesis de que $E$ es de medida finita. $E$ puede representarse en la forma $E=\cup_{k=1}^{+\infty}E_k$, donde $E_k$ son conjuntos integrables, disjuntos a pares. Sea dada $\upsilon \en L_1(E,\K)^*$. Si $f \en L_1 (E,\K)$, pongamos $f_k=fx_{E_k}$ $\todo k \en \N$. Entonces $f=\sum_{k=1}^{+\infty}f_k$ en todo punto de $E$ y $\todo n\en \N$:
\end{enumerate}
 $| \sum_{k=1}^n f_k | \leq |f|$. Ya que $f$ es una función integrable independiente de $n$, se sigue del teorema de Lebesgue que $\lim_{n \to +\infty} N_1 (f-\sum_{k=1}^n f_k)=0$. La continuidad de $\Phi$ implica ahora:
 \begin{equation}
 \Phi (f)=\sum_{k=1}^{+\infty}\Phi (f_k)
 \end{equation}
 Sea $\Phi_k$ la restricción de $\Phi$ a funciones integrables, nulas fuera de $E_k$. $\Phi_k$ se puede considerar como forma lineal continua sobre $L_1 (E_k,\K)$ de norma $|| \Phi_k || \leq || \Phi ||$. Por la parte b) de esta demostración existe una función $g_k: E \flecha \K$ medible, acotada, nula fuera de $E_k$ tal que $\Phi_k=\Phi_{g_k}$. Luego
 $$
 \Phi (f_k)=\Phi_k (f_k) = \int_E f_k g_k=\int_E fg_k
 $$
 Por lo tanto (2.25) se convierte en:
 
 \begin{equation}
 \Phi (f)= \sum_{k=1}^{+\infty}\int_E fg_k.
 \end{equation}
 Sea $g:E \flecha \K$ la funcion que, para todo $k$, se reduce a $g_k$ sobre $E_k$, o sea $g=\sum_{k=1}^{+\infty}g_k$ en todo punto de $E$. Para casi todo $x$ en $E_k$ se tiene $|g(x)|=|g_k(x)| \leq || \Phi_k || \leq || \Phi ||$. Así pues para casi todo $x$ en $E$, $|g(x)| \leq || \Phi ||$, de ahí que $g \en L_\infty (E,\K)$. \\
 Finalmente, por una parte: \\
 
$fg=\sum_{k=1}^{+\infty}fg_k$ en todo punto de $E$. \\
Por otra parte \\
$$
\sum_{k=1}^{n}fg_k | =\sum_{k=1}^{n}|f| |g_k| \leq |f| \phantom{s} |g|
$$
De donde $|f|$ $|g|$ es una función integrable independiente de $n$. Se sigue pues del teorema de Lebesgue:  $\sum_{k=1}^{+\infty}\int_E fg_k=\int_E fg$. Substituyendo en (2.25) se consigue:
$$
\Phi (f)=\int_E fg
$$
\\
  \phantom{sssssssssssssssssssssssssssssssssss sasdasdasdasdadadssada} c. q. d \\ \\
  
\textbf{Nota}\\
Generalizando las proposiciones 16. y 33. se puede demostrar que si $l \leq p <\infty$, el dual topológico $L_p (E,\K)^*$ es canónicamente isométrico con $L_{p^*} (E,\K)$. La isometría  de que se trata asocia a todo $g \en L_{p^*} (E,\K)$ la forma lineal $\Phi_g$ sobre $L_p$ dada por
$$
\Psi_g (f)=\int_E fg \phantom{s} \todo f \en L_p
$$
Nos parece mejor posponer la demostración de este resultado a un curso sobre la teoría general de medida e integración.



%%%%%%%%% SIGUIENTE CAPITULO %%%%%%%%%%%%%

\chapter{Convolución en $L_1$}

\textbf{Proposición 1 y definición} \\
Sean $f,g \en l_1 (\Rn,K)$. \\
\begin{enumerate}[1)]
\item Para casi todo $x$ en $\Rn$ la función $y \to f(y) g(x-y)$ es integrable en \Rn.\\
Se pone 
\begin{equation*}
\boxed{(f*g)(x)=:\int_{\Rn}f(y)g(x-y)dy}
\end{equation*}
La función $f*g$ definida c.t.p en $\Rn$ se llama CONVOLUCIÓN de $f$ y $g$.
\item $f*g \en l_1 (\Rn,\K)$.
\item $\int_{\Rn}f*g=\int_{\Rn}.\int_{\Rn}g$
\item $N_1 (f*g) \leq N_1 (| f| * |g|)=N_1(f)N_1(g).$
\end{enumerate}
\underline{Demostración.} \\
Consideremos la función $(x,y)\flecha f(y) g(x-y)$ de $\Rn \times \Rn$ en $\K$. Ya que $f$ es medible, la función $(x,y) \flecha f(y)$ es medible (corolario de la prop. 13 del cap. IV). Sea $h: \Rn \times \Rn \flecha \K$ dado por $h(u,v)=:g(v)$ $\todo (u,v) \en \Rn \times \Rn$. Ya que $g$ es medible, $h$ es medible. Sea $\Phi:\Rn \times \Rn \flecha \Rn \times \Rn$ dado por $\Phi (x,y)=(x,x-y)$. $\Phi$ es un isomorfismo $C^{\infty}$ de $\Rn \times \Rn$ sobre $\Rn \times \Rn$.\\
Por el corolario 2. de la prop. 17 del cap. V la función $(x,y) \flecha (h_o \Phi)(x,y)=g(x-y)$ es medible. Por consiguiente es medible la función $(x,y) \flecha f(y) g(x-y)$.\\
También:
\begin{equation}
\int_{\Rn}|f(y)|dy\int_{\Rn}|g(x-y)|dx=N_1(f)N_1(g)<\infty
\end{equation}
(Para evitar la integral interior se hace, para $y$ fijo el cambio de variables $x=y+z$ $\to$ $z=x-y$ en \Rn. El jacobiano de este cambio de variables es 1. Luego $\int_{\Rn}|g(x-y)|dx=\int_{\Rn}|g(z)|dx=N_1 (g)$).\\
La relación (3.1) implica por el teorema de Tonelli que la función $(x,y) \flecha f(y)g(x-y)$ es integrable en $\Rn \times \Rn$. Por el teorema de Fubini, para casi todo $x \en \Rn$, la función $y\flecha f(y)g(x-y)$ es integrabe. Esto prueba la afirmación 1).\\
Por el propio teorema de Fubini, la función $f*g$ definida por $(f*g)(x)=:\int_{\Rn}f(y)g(x-y)dy$ es integrable (afirmación 2).\\
Además, intercambiando el orden de las integraciones y haciendo un cambio de variables como arriba, se consigue:

$$
\int_{\Rn} (f*g)(x)dx=\int_{\Rn}dx\int_{\Rn} f(y)g(x-y)dy=
$$
$=\int_{\Rn}f(y)dy\int_{\Rn}g(x-y)dx=\int_{\Rn}f.\int _{\Rn}g.$ probando la afirmación 3). \\
Reemplazando $f,g$ por $|f|$, $|g|$ se obtiene  \\
$(|f| * |g|)(x)=\int_{\Rn}|f(y)| |g(x-y)|dy \leq 0$ para casi todo $x$ en $\Rn$. \\
Luego:
\begin{equation}
N_1 (|f| * |g|)=\int_{\Rn}|f| * |g|=\int_{\Rn}|f| \int_{\Rn}|g|=N_1 (f)N_1 (g) 
\end{equation}
Finalmente
\begin{equation}
N_1 (f*g)=\int_{\Rn} dx | \int_{\Rn} f(y)g(x-y)dy| \leq 
\end{equation}
$$
 \leq \int_{\Rn}dx \int_{\Rn} |f(y)| |g(x-y)|dy=\int_{\Rn}(|f| * |g|)(x)dx=N_1(|f| * |g|)
$$
Las relaciones (3.2) y (3.3) demuestran la afirmación 4). \\
  \phantom{sssssssssssssssssssssssssssssssssss sasdasdasdasdadadssada} c. q. d \\ \\
  
  \underline{Demostración}
  La función $f*g$ está definida solamente c. t. p. en \Rn. La clase de equivalencia de $f*g$ es por el contrario, definida sin ambigüedad. Es ... clase de equivalencia depende solamente de las clases de equivalencia de $f$ y $g$. \\
  Podemos pues considerar la convolución $(f,g) \flecha f*g$ como aplicacion de $L_1 (\Rn, \K) \times L_1 (\Rn,\K)$ en $L_1 (\Rn, \K)$. \\
  Puesto que $N_1 (f*g) \leq N_1 (f) N_1 (g)$ \underline{esta aplicación es} \\
  \underline{bilineal continua.} (Cf. ejercicio). \\ \\
  \textbf{Proposición 2.} \\
  \begin{enumerate}[1)]
  \item La convolución $(f,g) \flecha f*g$ es una aplicacion bilineal de $L_1 \times L_1$ en $L_1$.
  \item La convolución es conmutativa: $f*g=g*f$ $\todo f,g \en L_1$.
  \item La convolución es asociativa: $(f*g)*h=f*(g*h)$ $\todo f,g,h \en L_1$.
  \end{enumerate}
  \underline{Demostración.}
  La comprobación de la bilinealidad es inmediata. Probemos la conmutatividad y la asociatividad. 
  \begin{enumerate}[a)]
\item Sean $f,g \en L_1$. Tenemos para casi todo $x$ en $\Rn$. 
$$
(g*f)(x)=\int_{\Rn} g(y)f(x-y)dy
$$
  (para $x$ fijo) el cambio de variables $y=x-z$ a $z=x-y$, de jacobiano $\Rn$. Se consigue:
$$
(g*f)(x)=\int_{\Rn} g(x-z)f(z)dz=(f*g)(x).
$$  
  
 Luego $g*f=f*g$.
 
 \item Sean $f,g,h \en L_1$.
 
  \end{enumerate}
  
Sea $Z$ un subconjunto despreciable de $\Rn$ tal que, si $x\en \Rn - Z$, existe $((|f| * |g|)*|h|)(x)$. Para $x$ fijo en $\Rn - Z$ consideremos la función:

\begin{equation}
(y,z) \flecha h(x-y)f(z)g(y-z)
\end{equation}
  de $\Rn \times \Rn$ en $\Rn$. \\
  La función $y \flecha h(x-y)$ es medible, como se ve por el cambio de variables $y=x-u$ $\to$ $x-y=u$. Por el corolario de la prop. 13 del cap. IV. también la función $(y,z) \flecha h(x-y)$ es medible. Por el razonamiento al comienzo de la prop. 1. la función$(y,z) \flecha f(z)g(y-z)$ es medible. Por consiguiente la función (3.4) es medible como producto de funciones medibles.\\
Además 
$$
\int_{\Rn}|h(x-y)|dy \int_{\Rn}|g(y-z)f(z)|dz=\int_{
\Rn}  (|f|*|g|)(y)|h(x-y)|dy=
$$
$$
=((|f|*|g|)*|h|)(x) < \infty
$$
El teorema de Tonelli implica ahora que la función (3.4) es integrable. Por una parte:
\begin{equation}
\int_{\Rn \times \Rn} h(x-y)f(z)g(y-z)dydz=\int_{\Rn} h(x-y)dy\int_{\Rn}g(y-z)f(z)dz	
\end{equation}
$=((f*g)*h)(x)$.
Por otra parte:
$$
\int_{\Rn \times \Rn} h(x-y)f(z)g(y-z)dydz=\int_{\Rn} f(z)dz\int_{\Rn}g(y-z)h(x-y)dy.	
$$
Hagamos el cambio de variables $y=u+z$ a $u=y-z$ en la integral interior. Se obtiene: 
\begin{equation}
\int_{\Rn \times \Rn} h(x-y)f(z)g(y-z)dydz=\int_{\Rn} f(z)dz \int_{\Rn}g(u)h(x-z-u)du
\end{equation}
$=\int_{\Rn}f(z)(g*h)(x-z)dz=(f*(g*h))(x).$\\
Al comparar (3.14) y (3.15) resulta:
$((f*g)*h)(x)=(f*(g*h))(x)$ $\todo x \en \Rn -Z$, o sea \\
$((f*g)*h)=f*(g*h)$ c. t. p.  \\
  \phantom{sssssssssssssssssssssssssssssssssss sasdasdasdasdadadssada} c. q. d \\ \\
  
\underline{Definición.}   \\
Un espacio de Banach $(E,| \phantom{s}|,| \phantom{s}|)$ toma el nombre ÁLGEBRA DE BANACH si está definida una aplicación $(\xraya,\yraya) \flecha \xraya \yraya$ de $ExE$ en $E$ ("multiplicación") que es bilineal y asociativa (o sea hace de $E$ un álgebra asociativa) cumpliéndose además:
$$
|| \xraya \yraya || \leq || \xraya || ||\yraya || \phantom{s} \todo \xraya, \yraya \en E
$$
Si la multiplicación es además conmutativa el álgebra de Banach se llama CONMUTATIVA. \\ \\
\underline{Ejemplos.}
\begin{enumerate}[1)]
\item El campo $\K$ mismo provisto de su estructura vectorial, de su multiplicación usual y de la norma $|$ $|$ es un álgebra de Banach conmutativa. 

\item Sea $S$ un conjunto arbitrario. El espacio vectorial $B(S,\K)$ de todas las funciones acotadas $S \flecha \K$ provisto de la "norma uniforme" (cf. ejemplo 3) después de la prop. 1 del cap. I) y de la multiplicación "usual": $(fg)(s)=:f(s).g(s)$ $\todo s \en S$ es un álgebra de Banach conmutativa.

\item Si $S$ es un espacio métrico, el subespacio ... de $B(S,\K)$ constituido por las funciones continuas acotadas es un álgebra de Banach conmutativa. 

\item El subespacio $\varphi_o (\Rn , \K)$ de $\varphi (\Rn, \K)$ constituido por las funciones continuas "nulas en el infinito" (es decir que tienden a cero en el infinito) es un álgebra de Banach conmutativa.

\item Sea $(E,| \phantom{s} |,| \phantom{s}|)$ es un espacio de Banach. El espacio de Banach End E de todos los endomorfismos lineales continuos de E en E provisto de la multiplicación $(A,B) \flecha A o B$ es un álgebra de Banach no conmutativa (cf. prop. 8 del cap. I).\\

\end{enumerate}
  
  Finalmente, de las prop. 1 y 2. sigue sin más: \\ \\
  
  \textbf{Proposición 3.} \\
El espacio de Banach $L_1 (\Rn, \K)$ provisto de la "multiplicación" : $(f,g) \flecha f*g$ es un álgebra de Banach conmutativa. La integral $\int$ es un homomorfismo continuo de este álgebra de Banach en \K. \\
\subsection{Convolución entre diferentes espacios $L_p$.}

\textbf{Lema.} \\
Sean $P_1, \ldots,P_r$ números positivos tales que $\frac{1}{P_1}+\ldots+\frac{1}{P_r}=1$. (De esto sigue que, de hecho, $P_1,\ldots,P_r$ son $>1$). Sean $f_1 \en L_{p_1} (\Rn, \K),\ldots, f_r \en L_{p_r}(\Rn ,\K)$. Entonces $f_1 \ldots f_r \en L_1 (\Rn,\K)$ y 

$$
N_1 (f_1 \ldots f_r) \leq N_{p_1} (f_1) \ldots N_{p_r} (f_r)
$$
(Desigualdad de Holder generalizada). \\ \\
\underline{Demostración.}\\

Hagamos la demostración por inducción sobre $r$. Sabemos que el resultado es cierto para $r=2$ (Desigualdad de Holder). \\
Supongámoslo cierto para $r-1$. Sean $p_1, \cdots, p|_r$ y $f_1, \cdots ,f_r$ como en el enunciado. Se tiene 
\begin{equation}
\frac{1}{P_r *}=\frac{1}{P_1}+ \cdots+ \frac{1}{P_{r-1}}
\end{equation}

Sea $q_k=\frac{P_k}{p_r *}$ para $k=1,\ldots, r-1$, Entonces por (3.7): 
\begin{equation}
\frac{1}{q_1}+\cdots + \frac{1}{q_{r-1}}=1.
\end{equation}
  Para $k=1, \ldots, r-1$ se tiene $|f_k|^{p_k} \en L_1$, equivalentemente $(|f_k|^{p_r * q_k} \en L_1$ o $|f_k|^{p_r *} \en L_{q_k}$. Tomando en cuenta esto y (3.8) obtenemos por la hipótesis de inducción en primer lugar: \\
 $|f_1 \cdots f_{r-1}|^{P_r *} \en L_1$ o sea $f_1 \cdots f_{r-1} \en L_{p_r *}$ \\
 En segundo lugar:
 $$
 N_{p_r *} (f_1 \cdots f_{r-1})^{p_r *}= N_{p_r *} (|f_1 \cdots f_{r-1}|^{P_r*}) \leq 
 $$
 $$
 \leq N_{q_1} (|f_1|^{p_r *}) \cdots N_{q_{r-1}} (|f_{r-1}|^{p_r *})=(\int |f_1|^{p_1})^{\frac{p_r *}{p_1}} \cdots \int |f_{r-1} |^{P_r-1})^{\frac{p_r *}{P_{r-1}}}
 $$
 $=(N_{p_1} (f_1) \cdots N_{P_{r-1} (f_{r-1}^{p_r *}}$, Así pues: 
 \begin{equation}
 N_{p_r *}(f_1 \cdots f_{r-1}) \leq N_{p_1} (f_1) \cdots N_{p-1} (f_{r-1})
 \end{equation}
 Apliquemos ahora la desigualdad de Holder a las funciones $f_1 \cdots f_{r-1} \en L_{p_r *}$ y $f_r \en L_{p_r}$. Se tendrá en primer lugar: $f_1 \cdots f_{r-1} \en L_1$ y en segundo lugar, valiéndose de (3.9):
 $$N_1 (f_1 \cdots f_{r-1} f_r) \leq N_{p_r *} (f_1 \cdots f_{r-1})N_{p_r} (f_r) \leq$$
 
 $$
  \leq N_{p_1}(f_1) \cdots N_{p_{r-1}}(f_{r-1})N_{p_r}(f_r)
 $$\\
  \phantom{sssssssssssssssssssssssssssssssssss sasdasdasdasdadadssada} c. q. d \\ \\

\textbf{Proposición 4.} \\
Sean $p,q$ números tales que $p \geq 1$, $q \geq 1$ y $\frac{1}{p} + \frac{1}{q} >1$. Definamos $r$ por $\frac{1}{r}=: \frac{1}{p} + \frac{1}{q}=1$. Sean $f \en L_p (\Rn, \K)$, $g \en L_p (\Rn, \K)$. Entonces: 
\begin{enumerate}[1)]
\item Para casi todo $x$ en $\Rn$ la función $y \flecha f(y) g(x-y)$ es integrable. Luego la función $f *g$ definida por\\
$(f*g) (x)=: \int_{\Rn} f(y) g(x-y)dy$ está definida c. t. p. en \Rn. 
\item $f *g=g*f \en L_r (\Rn,\K)$.
\item $N_r (f*g) \leq N_p(f) N_q (f)$
\end{enumerate}
\underline{(DESIGUALDAD DE YOUNG)}. \\ \\
\underline{Demostración. }
Notemos que, siendo $\frac{1}{p} \leq 1$, $\frac{1}{q} \leq 1$ de donde $\frac{1}{p}+\frac{1}{q} \leq 2$ se tiene $\frac{1}{r} \leq 1$ o sea $r \geq 1$. También $\frac{1}{q}-\frac{1}{r}=1-\frac{1}{q} \geq 0$ y $\frac{1}{q}-\frac{1}{r}=1-\frac{1}{p} \geq 0$. La función $f$ es medible. También la función $y \flecha g(x-y)$ es medible, como se ve por un cambio de variables. Luego es medible la función $y \flecha f(y) g(x-y)$.  \\
Escribamos 
\begin{equation}
|f(y) g(x-y)|=\biggl( |f(y)|^{P} |g(x-y)|^q \biggr)^\frac{1}{r}  \biggl(|f(y)|^P \biggr)^{\frac{1}{P}-\frac{1}{r}} \biggl( |g(x-y)|^q  \biggr)^{\frac{1}{q}-\frac{1}{r}}
\end{equation}
Supongamos por un momento que $p>1, q>1$, luego $\frac{1}{p}-\frac{1}{r}>0$ y $\frac{1}{q}-\frac{1}{r}>0$.\\
Definamos $\alpha, \beta, \gamma$ por $\frac{1}{\alpha}=:\frac{1}{r}$ $\frac{1}{\beta}=:\frac{1}{p}-\frac{1}{r},$ $\frac{1}{\gamma}=:\frac{1}{q}-\frac{1}{r},$
Entonces $\frac{1}{\alpha}+\frac{1}{\beta}+\frac{1}{\gamma}=1$. El primer factor a la derecha de (3.9) considerado como función de $y$ está en $L_{\alpha}$ para casi todo $x$ (a saber para aquellos $x$, para los cuales existe $(|f|^P * |g|)(x,.)$Los otros dos factores están respectivamente en $L_{\beta}$ y $L_{\gamma}$. De (3.10) y del lema precedente se concluye que la función $y \flecha f(y).g(x-y)$ es integrable para casi todo $x$. Luego existe $(f*g)(x)$ para casi todo $x$. 
Si p. ej. $q=1$ y $p>1$, entonces $p=r$ y (3.10) se reduce a: 
\begin{equation}
|f(y) g(x-y)| \leq \biggl( |f(y)|^P |g(x-y)| \biggl)^{\frac{1}{P}|g(x-y)|^{1-\frac{1}{P}}}
\end{equation}
Ahora el primer factor a la derecha (3.11) considerado como función de $y$ está en $L_p$ para casi todo $x$ y el segundo está en $L_p$. Luego la conclusión subsiste. \\
La relación $f*g=g*f$ se demuestra exactamente como en la  prop. 2. \\
La función $f*g$ es medible por la prop. 10 del cap. IV.  \\
Volviendo el caso $p>1$, $q>1$ obtenemos de (3.11) por el lema, para casi todo $x$. 
\begin{equation}
|(f*g)(x)| \leq (|f|*|g|)(x)=\int_{\Rn}|f(y)g(x-y)| dy \leq
\end{equation}
$$
\leq \biggl( \int_{\Rn}|f(y)|^P |g(x-y)|^q dy  \biggl)^{\frac{1}{r}} \biggl( \int_{\Rn}|f(y)|^P dy \biggl)^{\frac{1}{p}-\frac{1}{r}} \biggl( \int_{\Rn} |g(x-y)|^q dy \biggl)^{\frac{1}{q}-\frac{1}{r}}
$$

$$
= \biggl( |f|^P * |g|^q)(x)^{\frac{1}{r}}N_p(f)^{1-\frac{p}{r}}N_q (g)^{1-\frac{q}{r}}
$$
Si  $q=1$, $p>1$, $r=p$ obtenemos de (3.11) por la desigualdad de Holder, para casi todo $x$:
$$
|(f*g)(x)| \leq (|f| *|g|)(x)=\int_{\Rn}|f(y)g(x-y)|dy \leq
$$

$$
\leq \biggl( \int_{\Rn} |f(y)|^P |g(x-y)|dy \biggl)^{\frac{1}{p}}\biggl(\int_{\Rn} |g(x-y)| \biggl)^{1-\frac{1}{p}}
$$
Que es un caso particular de (3.12). \\
De (3.12) se sigue:

\begin{equation}
|(f*g)(x)|^r \leq (|f|*|g|)(x)^r \leq (|f|^P *|g|^q)(x) N_p (f)^{r-p}N_q (g)^{r-q}
\end{equation}
Ahora bien por la prop. 1. la función $|f|^p *|g|*q$ definida c. t. p. es integrable. La relación (3.13) entraña pues que $|f*g|^r$ es integrable. Por lo tanto $f*g \en L_r$.\\
Finalmente, integrando (3.13) resulta:
$$N_r(f*g)^r \leq N_1 (|f|^p * |g|^q )N_p (f)^{r-p}N_q (g)^{r-q}=$$

$$
N_1 (|f|^p)N_1(|g|^q)N_p(f)^{r-p}N_q(g)^{r-q}=N_p(g)^p N_q (g)^q N_p (g)^q N_p (f)^{r-p}N_q (g)^{r-q}
$$

$$
=N_p (f )^r N_q(g)^r
$$
Así pues $N_r (f*g) \leq N_p (f) N_q (g)$. \\
  \phantom{sssssssssssssssssssssssssssssssssss sasdasdasdasdadadssada} c. q. d \\ \\
  \underline{Corolario.} \\
  La aplicación $(f,g) \flecha f*g$ es una aplicación bilineal continua de $L_p (\Rn, \K) \times L_q (\Rn, \K)$ en $L_r(\Rn,\K)$. \\ \\
  \textbf{Nota.} \\
 El caso particular $q=1$ (luego $r=p$) es importante:
Si $f \en L_p$, $g \en L_1$, entonces $f *g \en L_p$ y 

$$
N_p (f*g) \leq N_p (f) N_1 (g)
$$
Si deseamos generalizar la prop. 4 al caso $\frac{1}{p}+\frac{1}{q}=1$ (o sea $q=p*$), debemos poner $r=\infty$. De hecho, tenemos: 
\\ \\
\textbf{Proposición 5.} \\
Sea $1 \leq p \leq \infty$. Sean $f \en L_p (\Rn,\K)$ y $g\en L_{p*}(\Rn,\K)$. Entonces para todo $x\en \Rn$ (no, como antes, para casi todo) existe $(f*g)(x)=: \int_{\Rn} f(y)g(x-y)dy$. Se tiene $f*g=g*f$.\\
$f*g$ es una función medible acotada y 
$$
N_\infty (f*g) \leq N_p (f) N_{p*}(g)
$$
\underline{Demostración.} \\
La función $y \flecha g(x-y)$ es medible y pertenece a $L_{}p*$. Su seminorma $N_{p*}$ es la misma que la de $g$. Luego, $\todo x \en \Rn$, la función $y \flecha f(y) g(x-y)$ pertenece a $L_1$.\\
Sea $(f*g)(x)=\int_{\Rn} f(y)g(x-y)dy$ $\todo x \en \Rn$.\\
Exactamente como en la prop... se demuestra .... $f*g$ es medible por la prop. 10 del cap. IV. \\
Además, por al desigualdad de Holder, se tiene $\todo x$:
$$ |(f*g)(x)| \leq \int_{\Rn} |f(y) g(x-y)| dy \leq N_p (f) N_{p*}(g)$$.
Luego $f*g$ es acotada y $N_\infty (f*g) \leq N_p (f) N_{p*}(g)$.  \\
  \phantom{sssssssssssssssssssssssssssssssssss sasdasdasdasdadadssada} c. q. d \\ \\

\underline{Corolario.}  \\
La aplicación $(f,g) \flecha f*g$ es una aplicación bilioneal continua de $L_p(\Rn, \K) \times L_{p*}(\Rn,\K)$ en $L_\infty (\Rn, \K)$.  \\
De hecho, la aplicación $(f,g) \flecha f*g$ es una aplicación bilineal continua de $L_p (\Rn,\K) \times L_{p*}(\Rn,\K)$ en el espacio vectorial de las funciones medibles acotadas (aun de las funciones continuas acotadas, cf. prop. 6) provisto de la norma uniforme.
\\ \\

\textbf{Proposición 6.} \\
Sea $l \leq p \leq \infty$. Sean $f \en L_p (\Rn, \K)$ y $g\en L_{p*}(\Rn,\K)$. \\
Entonces $f*g$ es uniformemente continua en \Rn. \\
\underline{Demostración.}\\
Intercambiando, si hace falta, $f$ con $g$, podemos suponer $p* < \infty$. Se tiene, valiéndose de la desigualdad de Holder: 
$$
|(f*g)(x_1)-(f*g)(x_2)|=| \int_{\Rn}f(y)(g(x_1-y)-g(x_2-y))dy| \leq 
$$
  
$$
 \leq \int_{\Rn}|f(y)|. |g(x_1 - y)-g(x_2-y)|dy \leq N_p (f) \biggl(\int_{\Rn}|g(x_1-y)-g(x_2-y)|^{p*}dy \biggl)^{\frac{1}{p*}}
$$  
O sea, por un cambio de variables:
\begin{equation}
(f*g)(x_1)-(f*g)(x_2)| \leq N_p (f) \biggl(\int_{\Rn}|g(x_1+y)-g(x_2+y)|^{p*}dy \biggl)^{\frac{1}{p*}} 
\end{equation}  
 
 $$
 =N_p (f) N_{p*}(g_{x_1}-g_{x_2})
 $$
Donde $g_{x_1}(y)=: g(x_1+y)$ $\todo y \en \Rn$ y $g_{x_2}(y)=g(x_2+y)$ $\todo y \en \Rn$. \\
Por la prop. 17. del cap. VII. la aplicación $t \flecha g_t$ es una aplicación uniformemente continua en \Rn \phantom{} en $L_{p*}$. La conclusión ahora inmediatamente de  .
\\ \\
\textbf{Proposición 7.} \\
Sea $1 < p < \infty$ (luego también $1<p*<\infty$). Sean $f \en L_p (\Rn, \K)$, $g \en L_{p*} (\Rn,\K)$. Entonces 
$$
\lim_{x \to \infty}(f*g)(x)=0
$$
\underline{Demostración.} \\
Designaremos por $||$ $||$ a una norma en \Rn. \\
Sea $M$ una constante positiva por determinar más adelante. \\
Tenemos para todo $x \en \Rn$:
$$
|(f*g)(x)| \leq \int_{\Rn}|f(y)g(x-y)|dy=\stackbin[||y|| \leq M]{}\int{|f(y)g(x-y)|dy}+
$$

$$
+ \stackbin[||y||> M]{}\int{|f(y)g(x-y)|dy}.
$$
de donde por la desigualdad de Holder:
$$|(f*g)(x)| \leq N_p (f) \biggl(\stackbin[||y|| \leq M]{}\int{|g(x-y)|^{p*}}\biggl)^{\frac{1}{p*}}+\biggl(\stackbin[||y|| > M]{}\int{|f|} \biggl)^{\frac{1}{p}}N_{p*}(g)
$$
\begin{enumerate}[a)]
\item Designemos por $B_M$ la bola $\lbrace y| \phantom{s} ||y|| \leq M \rbrace$. Podemos escribir: 
\begin{equation}
\stackbin[||y||>M]{}\int{|f|^p}=\int_{\Rn}|f|^p (1-x_{B_M}).
\end{equation}
Para $M \to +\infty$, el integrando a la derecha de (3.15) tiende a cero en todo punto. Además $|f|^p (1-x-{B_M}) \leq |f|^p =$función integrable independiente de $M$. Por el teorema de Lebesgue $lim_{M \to \infty}\stackbin[||y|| > M]{}\int{|f|^p}=0$. En efecto, el teorema de Lebesgue implica que si $\lbrace M_k \rbrace$ es una sucesión de números positivos que tiende a $+\infty$, entonces 
$$
\lim_{k \to \infty}\stackbin[||y|| > M_k]{}\int{|f|^p}=0
$$
Sea $\epsilon>0$ dado. Por lo dicho podemos fijar $M$ de suerte que se verifique:
\begin{equation}
\biggl(\stackbin[||y|| > M]{}\int{|f|^p}\biggl)^{\frac{1}{p}}<\epsilon
\end{equation}
\item $M$ así fijado, consideremos el primer término a la derecha de (3.16). Por el cambio de variables $y=x-z$, $z=x-
y$, obtenemos:
\begin{equation}
\stackbin[||y|| \leq M]{}\int{|g(x-y)|^{p*}dy}=\stackbin[||z-x|| \leq M]{}\int{|g(z)|^{p*}}
\end{equation}
Por el mismo razonamiento que en la parte a) existe $R>0$ tal que $\biggl(\stackbin[||z|| > R]{}\int{|g(z)|^{p*}dz}\biggl)<\epsilon$. Tomemos $x$ tal que $||x|| >R+M$. \\
Entonces si $||z-x|| \leq M$, se tiene
$$
||z||=||x-(x-z)|| \geq ||x||-||x-z|| >(R+M)-M=R
$$
Tenemos pues
$$
\biggl(\stackbin[||z-x|| \leq M]{}\int{|g(z)|^{p*}dz}\biggl)\leq \biggl(\stackbin[||z|| \geq R]{}\int{|g(z)|^{p*}dz}\biggl)^{\frac{1}{p*}}<\epsilon
$$
\end{enumerate}
Por (3.17) rige la implicación: 
\begin{equation}
||x|| >R+M \flecha \stackbin[||y|| \leq M]{}\int{|g(x-y)|^{p*}dy}<\epsilon
\end{equation}
Finalmente, en virtud de (3.14), (3.16) y (3.18) tenemos la implicación: 
$$
||x|| >R+M \flecha |(f*g)(x)|<\epsilon (N_p (f)+N_{p*}(g))
$$\\
  \phantom{sssssssssssssssssssssssssssssssssss sasdasdasdasdadadssada} c. q. d \\ \\
  La prop. 7 no se generaliza al caso $p=1$,$p*=\infty$. P. ej. sea $f\en L_1$ tal que $\int_{\Rn}f\neq 0$ y sea $g$ la constante 1. Entonces $f*g$ es la constante no nula $\int_{\Rn}f$. Sin embargo tenemos \\ \\
  \textbf{Proposición 8.}
  Sean $f \en L_1 (\Rn,\K)$ y $g\en L_{\infty}(\Rn,\K)$. Se supone que $\lim_{x \to \infty}g(x)=0$. Entonces $\lim_{x \to +\infty}(f*g)(x)=0$. \\
  \underline{Demostración.} \\
  Sea $M$ una constante positiva por determinar. Escribamos: 
 \begin{equation}
 |(f*g)(x)| \leq  \stackbin[||y|| \leq M]{}\int{|f(x-y)g(y)|dy}+ \stackbin[||y|| >M]{}\int{|f(x-y)g(y)|dy} \leq 
 \end{equation}

$$
 \leq N_\infty (g)  \stackbin[||z-x|| \leq M]{}\int{|f(z)|dz}+N_1 (f) \stackbin[||y|| >M]{}Sup |g(y)|
$$
Sea $\epsilon>0$ dado. Por la hipótesis sobre $g$, podemos fijar $M$ de suerte que se cumpla:  $\stackbin[||y|| >M]{}Sup |g(y)|$. \\
$M$ así fijado, por el mismo razonamiento que en la demostración precedente: 
$$
\exists \phantom{s}p>0 \phantom{s}\pitchfork ||x||>p \flecha \stackbin[|z-x||\leq M]{}\int |f(z)|dz <\epsilon
$$
Luego, por (3.13):
$$
||x|| >p \flecha |(f*g)(x)| <\epsilon (N_\infty (g)+N_1 (f))
$$
 \phantom{sssssssssssssssssssssssssssssssssss sasdasdasdasdadadssada} c. q. d \\ \\
 En relación con la proposición siguiente hacemos notar que existen funciones $\Rn \flecha \K$ de clase $C^\infty$, de soporte compacto.\\
 Pongamos p. ej. \\
 \begin{equation*}
\rho (x)=\left\{ \begin{array}{lcc}
           exp(\frac{1}{1-||x||^2}) & si & ||x||<1 \\
             \\  0 & si &||x|| \geq 1 \\
             \end{array}
   \right.
\end{equation*}
donde $||$ $||$ es la norma euclidiana. El lector comprobará que $\rho$ es una función de clase $C^\infty$. Claramente Spt $\rho=\lbrace x| \phantom{s} ||x|| \leq 1 \rbrace$ es compacto. \\ \\
\textbf{Proposición 9.} \\
Sea $1 \leq p \leq \infty$ y sea $f \en L_p (\Rn, \K)$. Sea $g: \Rn \flecha \K$ una función de clase $C^r$ de soporte compacto, $r \geq 1$. Entonces $f*g$ es una función de clase $C^r$. Si además $D=: \gamma_{\alpha \cdots \alpha_k} (k \leq r,\alpha_1, \cdots, \alpha_k \en [1,n])$ se tiene: 
$$
\underline{D(f*g)=f*Dg.}
$$
\underline{Demostración.} \\
Notemos que, siendo evidentemente $g \en L_{p*}$, la convolución $f*g$ existe, está bien determinada y continua en todo punto. La misma observación, con $Dg$ en lugar de $g$, asegura que $f*Dg$ es una función continua. Una vez probada la relación
\begin{equation}
D(f*g)=f*Dg
\end{equation}
quedará pues establecido que $f*g$ es de clase $C^r$. \\
Para demostrar (3.20) inductivamente, basta probar
\begin{equation}
\gamma_k (f*g)=f*\gamma_k g \phantom{ss} k \en [1,n]
\end{equation}
Notemos que el resultado (3.21) es inmediato si $p=1$. Sea en efecto, en este caso, $M=:  \stackbin[z \en \Rn]{}{M\acute{a}x} | \gamma_k g(z)|$. Entonces $|f(y) \gamma_k g(x-y)| \leq M|f(y)|$. Ya que $y \flecha M|f(y)|$ es una función integrable independiente de $x$, se tiene por la prop. 2. del cap. VI. (derivación bajo el signo integral) $\todo x \en \Rn$:
$$
\gamma_k (f*g)(x)=\int_{\Rn}f(y)\gamma_k g(x-y)dy=(f*\gamma_k g)(x)
$$
Obsérvese que en esta demostración la sola hipótesis utilizada sobre $\gamma_k g$ es que esta función es acotada en $\Rn$. \\
Pasemos al caso general. \\
Sea $(\overrightarrow{e_1}, \ldots,\overrightarrow{e_n})$ la base natural de $\Rn$ y sea $||$ $||$ una norma en $\Rn$ tal que $||\overrightarrow{e_k}=1$ $\todo k$. (p. ej. la norma euclidiana o la norma cúbica). Sea $d$ la distancia inducida por esta norma. \\
Pongamos de nuevo $M=:  \stackbin[z \en \Rn]{}{M\acute{a}x} | \gamma_k g(z)|$ y sea $K=Spt \gamma_k g$. $K$ es un compacto y:
\begin{equation}
|\gamma_k g(z)| \leq Mx_K (z) \phantom{s} \todo z \en \Rn
\end{equation}
Sea $K'= \lbrace z | d(z,K) \leq 1 \rbrace$. $K'$ es otro compacto. Se comprueba inmediatamente que si $||u|| \leq 1$, entonces $x_K (z+u) \leq x_{K'} (z)$. De ahí y de (3.22)
$$
\gamma_k g(z+u)| \leq Mx_{K'}(z), \phantom{ss} ||u|| \leq 1
$$
Fijemos $x$ en $\Rn$ y sea $\lbrace h_\upsilon \rbrace$ una sucesión de números reales tales que $h_\upsilon \neq 0$ $\todo \upsilon$ y $\lim_{\upsilon \to +\infty}h_\upsilon=0$. Podemos suponer $|h_\upsilon| \leq 1$ $\todo \upsilon$. \\
Se tiene:
\begin{equation}
\frac{(f*g)(x+h_\upsilon \overrightarrow{e_k})-(f*g)(x)}{h_\upsilon}=\int_{\Rn}f(y)\frac{g(x-y+h_\upsilon \overrightarrow{e_k}-g(x-y)}{h_\upsilon}dy
\end{equation}
Para $\upsilon \to +\infty$ el integrando converge a $f(y)\gamma_k g(x-y)$ en todo punto por el teorema de incrementos finitos: 
$$
|\frac{g(x-y+h_\upsilon \overrightarrow{e_k})}{h_\upsilon}| \leq  \stackbin[0 \leq \theta \leq 1]{}{Sup}|\gamma_k g(x-y+ \theta h_\upsilon \overrightarrow{e_k})| 
$$
Pero por :
$$
|\gamma_k g(x-y+ \theta h_\upsilon \overrightarrow{e_k})|  \leq Mx_{K'}(x-y).
$$
Así pues:
$$
|f(y),\frac{g(x-y+h_\upsilon \overrightarrow{e_k})-g(x-y) }{h_\upsilon}| \leq M|f(y)x_{K'}(x-y)|
$$
Puesto que $f \en L_p$ y $x_{K'} \en L_{p'}$, la función $y \flecha |f(y)x_{K'}(x-y)|$ es integrable. Se puede pues aplicar el teorema de Lebesgue y pasar al límite bajo el signo integral en (5). Queda:
$$
\gamma_k (f*g)(x0=\int_{\Rn} f(y) \gamma_k g(x-y)dy=(f*\gamma_k g)(x)
$$
 \phantom{sssssssssssssssssssssssssssssssssss sasdasdasdasdadadssada} c. q. d \\ \\

\subsection{Sucesiones de Dirac}
El álgebra de Banach $L_1 (\Rn, \K)$ no posee un elemento "uno". Es decir no existe una funcion $\delta \en L_1$ tal que $\delta *f=f$ c. t. p. $\todo f \en L_1$. Tampoco hay $\delta \en L_1$ tal que $\delta *f=f$ c. t. p. para todo $f$ en cierto $L_p$. En efecto supongamos que hubiera tal $\delta$. Sea $P$ un tarugo acotado de volumen$>0$. Se tendrá pues $\alpha * X_p=X_p$ c. t. p. Siendo $\delta \en L_1$ y $x_p \en L_\infty$, la función $\delta * X_p$ sería continua en $\Rn$ por la prop. 6. Sea $Z$ un subconjunto despreciable de $\Rn$ tal que si $x \en \Rn -Z$, entonces $(\delta * X_p)(x)=X_p (x)$. Puesto que $Z$ no tiene puntos interiores, $P-Z$ es denso en $P$ y ya que $(\delta * X_p )(x)=0$ $\todo x \en \Rn -P$. Luego $\delta * X_p$ no puede ser continua en \Rn. Esta contradicción prueba que no existe "la función $\delta ".$ \\
Vamos sin embargo a mostrar que existen sucesiones $\lbrace p_\upsilon \rbrace$ de funciones en $L_1$, tales que, para $f$ en un espacio funcional conveniente, $\rho_\upsilon * f$ aproximará en cierto sentido la función $f$. \\ 
\underline{Definición.} \\
Una sucesion $\lbrace \rho_\upsilon \rbrace$ de funciones en $L_1 (\Rn,\K)$ se llama una SUCESIÓN DE DIRAC si cumple las siguientes condiciones:
\begin{enumerate}[1)]
\item $\rho_\upsilon (x) \geq 0$ $\todo x \en \Rn$.
\item $\int_{\Rn} \rho_\upsilon=1$ $\todo \upsilon$.
\item $\todo \delta >0$, $\lim_{\upsilon \to +\infty}  \stackbin[||x|| \leq \delta]{}{\int \rho_\upsilon (x)dx=1.}$
\end{enumerate}
Aquí $||$ $||$ es una norma en \Rn. Ya que dos normas arbitrarias sobre $\Rn$ son equivalentes, se ve fácilmente que la condición 3) es independiente de la norma elegida. \\
La condición 3) es vista de 2) puede reemplazarse por la siguiente:
$$
\underline{\todo \delta>0: \lim_{\upsilon \to +\infty} \stackbin[||x|| > \delta]{}{\int \rho_\upsilon (x)dx=0.}}
$$
 \underline{Ejemplos.} \\
 \begin{enumerate}[1.]
 \item Sea $\rho_\upsilon:\Rn \flecha \R$ la función, cuya gráfica está en la figura abajo.
 \\ \\ \\ \\
Se ve inmediatamente que $\lbrace \rho_\upsilon \rbrace$ es una sucesión de Dirac en $L_1 (\R , \K)$. 
\item Sea $\rho$ una función integrable real no negativa en $\Rn$ tal que $\int_{\Rn}\rho=1$. Pongamos $\todo \upsilon \en \N$: $\rho_\upsilon(x)=:\upsilon^n \rho (\upsilon,x)$. Afirmamos que $\lbrace \rho_\upsilon \rbrace$ es una sucesión de Dirac en $L_1 (\Rn, \K)$. La propiedad 1) es clara. Además, por el cambio de variables $x=\Phi (y) =: \frac{y}{\upsilon}$ con $J \phi (y)=\frac{1}{\upsilon^n}$ se obtiene: 
$$
\int_{\Rn}  \rho_\upsilon =\upsilon^n \int_{\Rn} \rho (\upsilon x)dx=\int_{\Rn} \rho (y)dy=1
$$
Probando 2). \\
 Finalmente, si $\gamma >0$: 
 $$
 \stackbin[||x|| < \delta]{}{\int \rho_\upsilon (x)dx}=\upsilon^n  \stackbin[||x|| \leq \delta]{}{\int \rho(\upsilon x)	dx}=\stackbin[||y|| \leq \delta]{}{\int \rho(y)	dy}
 $$
 que tiende a 1 para $\upsilon \to +\infty$ como se ve fácilmente p. ej. por el teorema de Beppo Levi. \\
 A título auxiliar vamos a necesitarla:
 \\ \\
 
 \end{enumerate}
 \textbf{Proposición 10.}\\
 Sea $E$ un subconjunto medible de \Rn. Sea $\rho$ una función no negativa, integrable en $E$, tal que $\int_E \rho=1$. \\
 Sea $I$ un intervalo de $\R$ y $\varphi$ y una función convexa $I \flecha \R$. Finalmente sea $f$ una función $E \flecha I$ tal que $f$ y $(\varphi_f)$ son integrables en $E$. \\
Entonces $\int_E fp \en I$ y se tiene:
\begin{equation*}
\varphi (\int_E fp) \leq \int_E (\varphi_o f)\rho
\end{equation*}

(DESIGUALDAD DE JENSEN) \\
\underline{Demostración.} 
\begin{enumerate}[a)]
\item Supongamos que para cierto número real $a$, se tiene $a \leq f$ en todo punto de $E$. De ahí sigue $a \leq f \rho$ y por integración $a \leq \int_E f \rho$. Asimismo, si $f \leq b$ en todo punto de $E$, se sigue  $\int_E f \rho \leq b$. Supongamos ahora $a<f$ en todo punto de $E$. De nuevo sigue $\rho a \leq \rho f$ de donde $a \leq \int_E f \rho$. Pero si tuviéramos $\int_E f \rho=a$ o sea $\int_E (\rho f-\rho a)=0$ se seguiría $\rho f=\rho a$ c. t. p. en $E$ o sea $f=a$ c. t. p. en el conjunto $S= \lbrace x | x \en E, \rho (x)>0 \rbrace$ de medida positiva. Esto es una contradicción, luego $a < \int_E f \rho$. Asimismo, se prueba que si $f<b$ en todo punto de $E$, se tiene $\int_E f \rho <b$. \\
De estas consideraciones se sigue inmediatamente que $\int_E f \rho \en I$. \\
\item Sea $c=: \int_E f \rho$. Supongamos $c \en \I$. Si $\beta,t \en I$ cumplen $a<c<t$ se tiene $\frac{\varphi (c)- \varphi (s)}{c-s} \leq \frac{\varphi (t)- \varphi (c)}{t-c}$. \\
Existe pues $\alpha=: \stackbin[s < c]{}{Sup}  \frac{\varphi (c)- \varphi (s)}{c-s}$ \\
Si $s \en I$, $s<c$, se  tiene  $\frac{\varphi (c)- \varphi (s)}{c-s} \leq \alpha$ o sea 
\begin{equation}
\varphi (s) \geq \varphi (c)+\alpha (s-c)
\end{equation}
Si $t \en I$, $t>c$, se tiene $\alpha \leq \frac{\varphi (t)- \varphi (c)}{t-c}$, de donde 
$$
\varphi (t) \geq \varphi (c)+\alpha (t-c)
$$

Así pues la relación (1) es válida $\todo s \en I$. Al substituir en (1) $s$ por $f(x)$ y multiplicar por $\beta (x)$ se obtiene: 
$$
(\varphi  \cdot f) \rho (x) \geq \varphi (c) \rho (x) + \alpha  (f(x)-c)\rho (x) \phantom{s} \todo x \en E
$$
Integrando sobre $E$ se consigue finalmente:
$$
\int_E (\varphi \cdot f) \rho \geq \varphi (c) + \alpha (\int_E f\rho-c)=\varphi (c)=\varphi (\int_E f\rho)
$$
viniendo probada la desigualdad de Jensen. 
\item Queda por examinar el caso de ser $\int_E f \rho$ uno de los extremos de $I$, que entonces pertenece a $I$. Supongamos p. ej. que $\int_E f\rho$ coincide con el extremo izquierdo $a$ de $I$. \\
De la parte $a)$ se sigue que la relacion $\int_E f \rho =a$ implica $f=a$ c. t. p. en el conjunto $S= \lbrace x | x\en E, \rho (x)>0 \rbrace$. Se tiene pues $\int_E (\varphi \cdot f)\rho =\int_S (\varphi \cdot f)\rho=\varphi (a) \int_S \rho =\varphi (a)$. \\
Por lo tanto $\int_E (\varphi \cdot f) \rho = \varphi (\int_E f \rho)$, que es un caso particular de la desigualdad de Jensen.  \\
 \phantom{sssssssssssssssssssssssssssssssssss sasdasdasdasdadadssada} c. q. d \\ \\
 \underline{Ejemplo.} \\
 Sea $f:E \flecha [0, +\infty [$ y sea $\varphi (t)=:t^p$ con $p \geq $... \\
 Entonces 
 \begin{equation}
 (\int_E f \rho )^p \leq \int_E f^p \rho 
 \end{equation}
 \end{enumerate}
Siempre que las integrales existan. Como los integrandos son positivos, es claro que bastará suponer $f\rho$ integrable y $f^p \rho$ medible, a condición de admitir integrales infinitas. \\ \\
\textbf{Proposición 11.} \\
Sea $\lbrace \rho_\upsilon \rbrace$ una sucesión de Dirac. Sea $1 \leq p < \infty$ y sea $f \en L_p (\Rn, \K)$. Entonces $\lbrace \rho_\upsilon *f \rbrace$ converge a $f$ en $p$-promedio. \\
\underline{Demostración.} \\
Recordemos que $\rho_\upsilon * f \en L_p$  $\todo \upsilon$ (cf. nota después de la prop 4). Para casi todo $x$ tenemos:
$(\rho_\upsilon *f)(x)=\int_{\Rn} f(x-y)\rho_\upsilon (y)dy.$ \\
Por la propiedad ii) de sucesiones de Dirac, podemos escribir: \\
$f(x)-(\rho_\upsilon * f)(x)=\int_{\Rn} (f(x)-f(x-y))\rho_\upsilon (y)dy$.\\
De ahí, por la desigualdad de Jensen:
$$
|f(x-(\rho_\upsilon * f)(x)|^p \leq (\int_{\Rn} |f(x)=f(x-y)|\rho_\upsilon (y)dy)^p \leq 
$$
$$
\leq \int_{\Rn} |f(x)-f(x-y)|^p \rho_\upsilon (y)dy
$$
donde el último miembro se trata como integral de una función medible positiva. \\
Al integrar resulta: 
\begin{equation}
N_p (f-\rho_\upsilon * f) \leq \int_{\Rn}dx \int_{\Rn}|f(x)-f(x-y)|^p \rho_\upsilon (y)dy
\end{equation}
Puesto que la función $(x,y) \flecha f(x)-f(x-y)|^p \rho_\upsilon (y)$
es medible positiva, la integral reiterada tiene sentido con la convención después de la prop. 11. del cap. IV). \\
Intercambiando el orden de las integraciones a la derecha de (1) obtenemos: 
\begin{equation}
N_p (f-\rho_\upsilon *f)^p \leq \int_{\Rn} N_p (f-f_{-y})^p \rho_\upsilon (y)dy.
\end{equation}
Aquí $f_{-y}$ es la función $x \flecha f(x-y)$. \\
Sea dado $\epsilon>0$. Puesto que la aplicación $h \flecha f_h$ da $\Rn$ en $L_p(\Rn, \K)$ es continua (prop. 17. del cap. VII): \\
$$
\exists \delta >0 \pitchfork \phantom{s} ||y|| \leq \delta \flecha \phantom{s} N_p (f-f_{-y})<\frac{\epsilon}{2^{\frac{1}{p}}}
$$
Escribamos
\begin{equation}
\int_{\Rn}N_p (f-f_{-y})^p \rho_\upsilon (y)dy=\stackbin[||y|| \leq \delta]{}{\int N_p (f-f_{-y})^p\rho_\upsilon (y)dy}+
\end{equation}

$$
+ \stackbin[||y|| > \delta]{}{\int N_p (f-f_{-y})\rho_\upsilon (y)dy}.
$$
La primera integral a la derecha de (3.28) es $\leq \frac{\epsilon^p}{2}$. La segunda es $ \leq 2^p N_p (f)^p \stackbin[||y|| > \delta]{}{\int \rho_\upsilon (y)dy}$. Por la propiedad iii) de sucesiones de Dirac se sigue que $\exists n_o \en N \phantom{s} \pitchfork v \geq n_o \flecha 2^p N_p (f)^p \stackbin[||y|| > \delta]{}{\int \rho_\upsilon (y)dy} <\frac{\epsilon^p}{2}$. \\
Así pues $v \geq n_o \flecha \int_{\Rn} N_p (f-f_{-y})\rho_\upsilon (y)dy<\epsilon^p$.\\
Por (3.27): $v \geq n_o \flecha N_p (f-\rho_\upsilon * f)<\epsilon$.\\
Con eso probamos que $\lim_{\upsilon \to +\infty}
N_p (f-\rho_\upsilon *f)=0$. \\
 \phantom{sssssssssssssssssssssssssssssssssss sasdasdasdasdadadssada} c. q. d \\ \\
En las dos proposiciones a continuación será cómodo, aunque no indispensable, valerse de la siguiente observación: \\
Si $f \en L_\infty (\Rn, \F)$, entonces en todo punto de continuidad $x$ de $f$ se tiene $||f(x)|| \leq N_\infty (f)$. \\
En efecto, dado $\epsilon>0$ existe una bola abierta $B$ de centro $x$ tal que $\todo y \en B$: $||f(y)|| > ||f(x)|| -\epsilon$. Luego $||f(x)||- \epsilon$ no es mayormente esencial de $f$ de donde $||f(x)||- \epsilon<N_\infty (f)$. Puesto que $\epsilon$ es arbitrario, se tiene bien $||f(x)|| \leq N_\infty (f)$. \\
Asimismo se demuestra que si existe $\lim_x f$ se tiene $||\lim_x f || \leq N_\infty (f)$. \\ \\
\textbf{Proposición 12.} \\
Sea $\lbrace \rho_\upsilon \rbrace$ una sucesión de Dirac. Sea $f\en L_\infty (\Rn , \K)$. \\
Si $f$ es continua en un punto $x \en \Rn$, se tiene
$$
f(x)=\lim_{\upsilon \to +\infty} (\rho_\upsilon *f)(x)
$$
\underline{Demostración.} \\
Notemos que el enunciado tiene sentido. En efecto, siendo $\rho_\upsilon \en L_1$ y $f \en L_\infty$, la convolución $\rho_\upsilon * f$ está bien definida en todo punto. \\
Como en la demostración precedente, tenemos: 
\begin{equation}
|f(x)-(\rho_\upsilon *f)(x)| \leq \int_{\Rn}|f(x)-f(x-y)|\rho_\upsilon (y)dy.
\end{equation}
Sea $\epsilon >0$ dado. Por la continuidad de $f$ en $x$:
$$
\exists \delta >0, \phantom{s} ||h|| \leq \delta \flecha | f(x+h)-f(x)| < \frac{\epsilon}{2}.
$$
Escribamos 
\begin{equation}
\int_{\Rn} |f(x)-f(x-y)| \rho_\upsilon (y)dy=\stackbin[||y|| \leq \delta]{}{\int |f(x)-f(x-y)|\rho_\upsilon (y)dy}+
\end{equation}
$$
\stackbin[||y|| > \delta]{}{\int | f(x)-f(x-y)|\rho_\upsilon (y)dy}
$$
La primera integral a la derecha de (3.29) es $\frac{\epsilon}{2}$. La segunda (en virtud de la observación que precede esta proposición) es $ \leq 2N_\infty (f) \stackbin[||y|| > \delta]{}{\int \rho_\upsilon (y)dy}.$ \\
Por la propiedad iii) de sucesiones de Dirac: 
$$
\epsilon n_o \en \N \phantom{s} \upsilon \geq n_o \flecha 2N_\infty (f) \stackbin[||y|| > \delta]{}{\int \rho_\upsilon (y)dy < \frac{\epsilon}{2}}.
$$
Luego, por (3.27) y (3.28)
$$
\upsilon \geq n_o \flecha |f(x)-(\rho_\upsilon *f)(x)| < \epsilon.
$$ \\
 \phantom{sssssssssssssssssssssssssssssssssss sasdasdasdasdadadssada} c. q. d \\ \\
\textbf{Proposición 13.} \\
Sea $\lbrace \rho_\upsilon \rbrace$ una sucesión de Dirac. Sea $f \en L_\infty (\Rn , \K)$. Se supone $f$ continua en un abierto $\Omega$ de \Rn. \\
Si $K$ es un compacto contenido en $\Omega$, $\lbrace \rho_\upsilon * f \rbrace$ converge a $f$ uniformemente en $x$.\\ 
\underline{Demostración.} \\
Si $\omega \neq \Rn$ sea $\alpha=: d(K, \Rn-\Omega)$. Si $\Omega = \Rn$ se elige $\alpha \geq 0$ arbitrario. Sea $K'=\lbrace z|z \en \Rn, d(z, K) \leq \frac{\alpha}{2} \rbrace$. $K'$ es un pacto y se verifica $K \subset K' \subset \Omega$.\\
Sea $\epsilon >0$ dado. Puesto que $f$ es uniformemente continua en $K'$:
$$
\exists \delta < \frac{\alpha}{2} \pitchfork x_1, x_2 \en K', ||x_2 \cdot x_1 || \leq \delta \flecha |f(x_2)-f(x_1)| < \frac{\epsilon}{2}
$$
Sea $x \en K$ arbitrario. Escribamos:

\begin{equation}
|f(x)-(\rho_\upsilon * f)(x)| \leq \stackbin[||y|| \leq \delta]{}{\int |f(x)-f(x-y)|\rho_\upsilon (y)dy}+
\end{equation}
$$
\stackbin[||y|| > \delta]{}{\int| f(x)-f(x-y)|\rho_\upsilon (y)dy}
$$
La primera integral a la derecha de ... es $\leq \frac{\epsilon}{2}$ (pues si $x \en K$ e $||y|| \leq \delta$, entonces $x-y \en K'$ y $|f(x)-f(x-y)|<\frac{\epsilon}{2})$. La segunda integral es $\leq 2 N_\infty (f)$. $\stackbin[||y|| > \delta]{}{\int \rho_\upsilon (y)dy}.$ \\
Por la propiedad iii) de sucesiones de Dirac:
$$
\exists n_o \en N \phantom{s} \upsilon \geq n_o \flecha 2N_\infty (f) \stackbin[||y|| > \delta]{}{\int \rho_\upsilon (y)dy}<\frac{\epsilon}{2}
$$
Finalmente por ... :
$$
\upsilon \geq n_o \flecha |f(x)-(\rho_\upsilon * f)(x)| <\epsilon \phantom{s} \todo x \en K
$$
\\
 \phantom{sssssssssssssssssssssssssssssssssss sasdasdasdasdadadssada} c. q. d \\ \\
 Las proposiciones 14. y 15. a continuación serán ejemplos de aplicaciones de sucesiones de Dirac. \\ \\
 \underline{Lema} \\
 Sean $f,g$ funciones medibles $\Rn \flecha \K$ tales que $f*g$ está definida. Si $f,g$ son de soporte compacto, $f*g$ es de soporte compacto. \\
 \underline{Demostración.} \\
 Sea $R>0$ tal que $Spt$ $f \subset \lbrace x| \phantom{s} ||x|| \leq R \rbrace$. Entonces para todo $x$, si $f*g$ está definida en todas partes, o para casi todo $x$, si $f*g$ está definida c. t. p. :
 \begin{equation}
 (f*g)(x)= \stackbin[||y|| \leq \delta]{}{\int f(y)g(x-y)dy}
 \end{equation}
Sea $K=\lbrace z|d(z,Spt(g)) \leq R \rbrace$. $K$ es un compacto. Si $x \en \Rn - K$ e $||y|| \leq R$, entonces $x-y\en \Rn-Spt(g)$, luego $g(x-y)=0$. Se sigue pues de ... :
$$
x\en \Rn -K \flecha (f*g)(x)=0
$$
Así pues $Spt(f*g) \subset K$, luego $Spt(f*g)$ es compacto.  \\
 \phantom{sssssssssssssssssssssssssssssssssss sasdasdasdasdadadssada} c. q. d \\ \\

\textbf{Proposición 14.}
Sea $1 \leq p < \infty$. El espacio vectorial $L_c^\infty (\Rn, \K)$ de las funciones $\Rn \flecha \K$ de clase $C^\infty$, de soporte compacto, es denso en $L_p (\Rn, \K)$. \\
\underline{Demostración.} \\
Sea $\rho:\Rn \flecha \Rn$ dada por:

 \begin{equation*}
\rho (x)=\left\{ \begin{array}{lcc}
           \frac{e^{\frac{-1}{1-||x||^2}}}{\int_{\Rn}e^{\frac{-1}{1-||x||^2}}dx} & si & ||x||\leq 1 \\
             \\  0 & si &||x|| > 1 \\
             \end{array}
   \right.
\end{equation*}
donde $||$ $||$ es la norma euclidiana. $\rho$ es de clase $C^\infty$, $Spt(\rho)=\lbrace x| \phantom{s} ||x|| \leq 1 \rbrace$ y $\int_{\Rn}\rho=1$. Consideremos la sucesión de Dirac $\lbrace \rho_\upsilon \rbrace$ donde $\rho_\upsilon (x)=\upsilon^n \rho (\upsilon x)$ $\todo \rho \en N$.\\
$\rho_\upsilon$ es de clase $C^\infty$ y $Spt(\rho_\upsilon)=\lbrace x | \phantom{s} ||x|| \leq \frac{1}{\upsilon} \rbrace$. \\
Sean dados $f \en L_p (\Rn,\K)$ y $\epsilon >0$. Sea $\varphi$ una función escalonada tal que  $N_p (f-\varphi)<\frac{\epsilon}{2}$. Ya que $\lbrace \rho_\upsilon * \varphi \rbrace$ converge en p-promedio a $\varphi$ (prop. 11.) podemos fijar $\upsilon$ de suerte que $N_p (\varphi -\rho_\upsilon * \varphi) < \frac{\epsilon}{2}$. Será pues:

\begin{equation}
N_p (f-\rho_\upsilon * \varphi)<\epsilon.
\end{equation}
Ahora bien $\rho_\upsilon * \varphi$ es de clase $C^\infty$ por la prop. 9 y es de soporte compacto por el lema precedente. Luego la relación ... demuestra la proposición. \\
\underline{Observación. } \\
Sea $\Omega$ un abierto de $\Rn$ y sea $L_c^\infty (\Omega,K)$ el espacio vectorial de las funciones de clase $C^\infty:\Omega \flecha \K$, de soporte compacto contenido en $\Omega$. Entonces $L_c^\infty (\Omega,\K)$ es denso en $L_p (\Omega, \K)$. \\
En efecto, supongamos en la demostración de la prop. 14. que $f$ es nula fuera de $\Omega$. Por el ejercicio se puede suponer que $Spt \varphi \subset \Omega$. Sea $K_\upsilon =: \lbrace z | d(z,Spt(\varphi)) \leq \frac{1}{\upsilon} \rbrace$. Para $\upsilon$ suficientemente grande será $K_\upsilon \subset \Omega$. Ahora bien, por la demostración del lema precedente $Spt(\rho_\upsilon * \varphi) \subset K_\upsilon$. Por consiguiente para $\upsilon$ suficientemente grande $Spt (\rho_\upsilon * \varphi) \subset \Omega$. \\
\underline{Nota}.
Es fácil generalizar la prop. 14 y la observación precedente al caso más general en que $K$ es substituye por un espacio de Banach $\F$. Por cierto estos resultados pueden establecerse por un método análogo al de la prop. 16 del cap. VII sin acudir a las sucesiones de Dirac. \\ \\
\textbf{Proposición 15.}(TEOREMA DE APROXIMACIÓN DE WEIERSTRASS). \\
Sea $[a,b]$ un intervalo compacto de \Rn. Designemos por $L(a,b)$ al espacio vectorial de las funciones continuas: $[a,b] \flecha \K$ provisto de la norma uniforme. Sea $P (a,b)$ el subespacio de $L(a,b)$ constituído por las funciones polinomiales $[a,b] \flecha \K$.\\
Entonces $P(a,b)$ es denso $L(a,b)$. \\
\underline{Demostración.} \\
\begin{enumerate}[a)]
\item Afirmamos que basta probar el teorema en el caso de ser $[a,b]=[0,1]$. En efecto supongámoslo ya establecido para $[0,1]$ y sea $f \en L(a,b)$. Sea $g:[0,1] \flecha \K$ dada por $g(t)=f(a+t(b-a))$ $\todo t \en [0,1]$. Entonces $g \en L(0,1)$. Sea $\epsilon >0$ dado. Por la hipótesis hecha exista una función polinomial $P$ tal que $\stackbin[x \en [a,b]]{}{|f(x)-P(\frac{x-a}{a-b}|}<\epsilon$.\\
Pero $x\flecha P(\frac{x-a}{a-b})$ es una función polinomial. Luego el teorema será cierto para $L(a,b)$. 
\item Sea $\overbrace{L(0,1)}$ el subespacio de $L(0,1)$ constituído por las funciones $f$ tales que $f(0)=f(1)=0$. Afirmamos que basta demostrar que $P(0,1)$ es denso en $\overbrace{L(0,1)}$. En efecto supongamos ya esto demostrado. Sea dado $f \en L(0,1)$ y sea $\epsilon >0$. Sea $h:[0,1]\flecha \K$ definida por $h(x)=:f(x)-f(0)-x(f(1)-f(0))$. Entonces $h \en \overbrace{L}(0,1)$ y por hipótesis habrá una función polinomial $P$ tal que $\stackbin[0 \leq x \leq 1]{}{Sup} |h(x)-P(x)| < \epsilon$. Sea $Q(x)=: f(0)+x(f(1)-f(0))$. Entonces $\stackbin[0 \leq x \leq 1]{}{Sup} |f(x)-P(x)-Q(x)| < \epsilon$. Ya que $P+Q$ es una función polinomial, el teorema vendrá establecido para $\overbrace{L}(0,1)$.
\item Mostremos finalmente que $P(0,1)$ es denso en $\overbrace{L}(0,1)$ $\todo \upsilon \en N$ sea $\rho_\upsilon:\R \flecha \K$ la función dada por:
\end{enumerate}
\begin{equation*}
\rho (x)=\left\{ \begin{array}{lcc}
        \frac{(1-t^2)^\upsilon}{\int_{-1}^1} & si & -1 \leq t \leq 1 \\
             \\  0 & si &||t|| > 1 \\
             \end{array}
   \right.
\end{equation*}
Esta función $\rho_\upsilon$ se conoce como el NÚCLEO de LANDAU. Mostremos que $\lbrace \rho_\upsilon \rbrace$ es una sucesión de Dirac en $L_1 (\R, \K)$. \\
Las propiedades i) y ii) son evidentes. Aprovechando la paridad de $\rho_\upsilon$ para probar iii) basta mostrar que si $0<\delta<1$, entonces 
$$
\lim_{\upsilon \to +\infty}\frac{\int_{\delta}^1 (1-t^2)^\upsilon dt}{\int_{-1}^1 (1-t^2)^\upsilon dt}=0
$$
Ahora bien $\int_{-1}^1 (1-t^2)^\upsilon dt=2\int_0^1 (1-t^2)^\upsilon dt=
$
$$
=2\int_0^1 (1-t)^\upsilon (1+t)^\upsilon dt \geq 2\int_0^1 (1-t)^\upsilon dt=\frac{2}{\upsilon +1}
$$

Luego $\frac{\int_\delta^1 (1-t^2)^\upsilon dt}{\int_{-1}^1 (1-t^2)^\upsilon dt} \leq \frac{\upsilon +1}{2} (1-\delta^2)^\upsilon (1-\delta) \stackbin[\upsilon \to +\infty]{}{\flecha} 0$. Por tanto se verifica también la propiedad iii) de las sucesiones de Dirac. \\
Sea dada $f \en \overbrace{L}(0,1)$. Sea $\overbrace{f}$ la ampliación canonica de $f$ a $\R$ (nula fuera de $[0,1]$). $\overbrace{f}$ es continua y acotada en \R . Por la prop. 13. $\lbrace \rho_\upsilon * \overbrace{f} \rbrace$ converge uniformemente a $f$ en $[0,1]$. Se tiene $(\rho_\upsilon * \overbrace{f} )(x)=\int_0^1 f(t)\rho_\upsilon (x-t)dt$. Pero si $x \en [0,1]$ y $t\en [0,1]$, entonces $x-t \en [-1,1]$, luego $\rho_\upsilon (x-t)=\frac{1}{c_\upsilon}(1-(x-t)^2)^\upsilon$, donde $c_\upsilon =:\int_{-1}^1 (1-\theta^2)^\upsilon d\theta$. \\
Así pues $\todo x \en [0,1]:$ $(\rho_\upsilon * \overbrace{f})(x)=\frac{1}{c_\upsilon}\int_0^1 f(t)(1-(x-t)^2)^\upsilon dt$, de donde se ve que $\rho_\upsilon * \overbrace{f}$ restringida a $[0,1]$ es una función polinomial. \\
Con esto el teorema queda demostrado. 

\subsection{Funciones periódicas en \R.}
Sea $f$ una función $\R \flecha \K$ y sea $T$ un número positivo. Recordamos que $f$ se dice FUNCIÓN PERIÓDICA de PERIODO T si:
$$
f(x+T)=f(x) \phantom{s} \todo x\en \R.
$$
Tal función $f$ está completamente determinada por su restricción a cualquier intervalo $[\alpha,\alpha + T[$, $\alpha \en \R$. \\
Si $f$ es medible, también es medible la función $x_{[\alpha , \alpha +T[}f$, en otras palabras la restricción de $f$ al intervalo $[\alpha , \alpha +T[$ o sea la medibilidad de la función $g=:x_{[\alpha , \alpha +T[}f$. $\todo k \en \Z$ también es medible la función $x \flecha g(x+kT)$ (corolario 2. de la prop. 17. del cap. V). Siendo $f(x)=\sum_{k \en \Z}g(x+kT)$ $\todo x \en \R$, $f$ es medible por el teorema fundamental sobre las funciones medibles. \\
También si $f$ es integrable en un intervalo de longitud $T$, $f$ es integrable en todo intervalo de longitud $T$ y se ve fácilmente que las integrales de $f$ en todos esos intervalos son iguales. \\
Por comodidad consideraremos de preferencia el intervalo $[-\frac{T}{2},\frac{T}{2}[$. \\
\underline{Definiciones.} \\
\begin{enumerate}[1)]
\item Sea $1 \leq p <\infty$. Designamos por $L_p ^T$ el espacio vectorial de las funciones $f:\R \flecha \K$, periódicas de perÍodo $T$, medibles y tales que $|f|^p$ es integrable en $[-\frac{T}{2},\frac{T}{2}[$. Si $f \en L_p^T$, ponemos

$$
N_p f=: ....
$$
\item Designamos por $L_p^T$ al espacio vectorial de las funciones $\R \flecha K$ periódicas de período $T$, medibles y esencialmente acotadas en $\R$ (o equivalentemente en $[-\frac{T}{2},\frac{T}{2}[$. Si $f \en L_p^T$, designamos por $N_\infty (f)$ el supremos esencial de $f$ en $\R$ (equivalentemente en $[-\frac{T}{2},\frac{T}{2}[$. \\
Si conviene distinguir los casos $\K =\R$ y $\K =C$ escribimos $L_p ^T (\R)$, resp. $L_p^T (C)$ $1 \leq p \leq \infty$.\\
La aplicación que a todo $f \en L_p^T (1 \leq p \leq \infty )$, le asocia su restricción a $[-\frac{T}{2},\frac{T}{2}[$ es un isomorfismo lineal de $L_p^T$ sobre $L_p ([-\frac{T}{2},\frac{T}{2}[,\K)$. De esto se sigue que $N_p$ es un seminorma sobre $L_p^T$ y que dicha aplicación es una isometría.\\
Designemos por $L_p^T$ al espacio normado asociado al espacio seminormado $L_p^T$ y por el mismo símbolo $N_p$ la norma en $L_p^T$. La isometría precedente induce una isometría de $L_p^T$ sobre $L_p ([-\frac{T}{2},\frac{T}{2}[,\K)$. Así pues los espacios $L_p^T$ son espacios de Banach. $L_2^T$ es un espacio hilbertiano. \\
Desde luego dos funciones $f,g \en L_p^T$ definen un mismo elemento de $L_p^T$ si y sólo si $f=g$ c. t. p. (en $\R$ o en $[-\frac{T}{2},\frac{T}{2}[)$.
\end{enumerate}

Puesto que la clase de equivalencia de una función $f \en L_p^T$ depende solamente de la restricción de $f$ al intervalo abierto $]-\frac{T}{2},\frac{T}{2}[$, podemos aun identificar $L_p^T$ con $L_p^T$ con $L_p (]-\frac{T}{2},\frac{T}{2}[, \K)$. \\
Finalmente notemos que, en virtud de la prop. 22. del cap. VII si $p,q \en [1,\infty]$ y $q>p$ , entonces $L_p^T \subset L_p^T$. En particular $L_p^T \subset L_1^T$ $\todo p \en [1,\infty]$. \\
\underline{Definiciones.}\\
\begin{enumerate}[1)]
\item Designemos por $\epsilon^T$ al espacio vectorial de las funciones $f:\R \flecha \K$, periódicas de período $T$,tales que $x_{]-\frac{T}{2},\frac{T}{2}[}f$ es una función escalonada.

\item Designamos por $L^T$ al espacio vectorial de las funciones continuas $\R \flecha \K$ periódicas de período $T$.\\
Se ve inmediatamente que si $f\en L^t$, $f$ es uniformemente continua en $\R$. \\ \\
\textbf{Proposición 16.} \\
Sea $1 \leq p \leq \infty$. Los espacios $\epsilon^T$ y $L^T$ son densos en $L_p^T$.\\
\underline{Demostración.} \\
Por el ejercicio, el espacio de las funciones escalonadas de soporte en $]-\frac{T}{2},\frac{T}{2}[$ y el espacio de las funciones continuas de soporte en $]-\frac{T}{2},\frac{T}{2}[$ son densos en $L_p (]-\frac{T}{2},\frac{T}{2}[,\K)$. 

\end{enumerate}
De esto la conclusión sigue inmediatamente. \\ \\
\textbf{Proposición 17.} \\
Sea $1 \leq p < \infty$. La aplicación $h \flecha f_h$ es una aplicación uniformemente continua de $\R$ en $L_p^T$. Además, $N_p(f_h)=N_p(f)$ $\todo h \en\R$ (aqui, como de costumbre, $f_h (x)=f(x+h)$ $\todo x \en \R)$. \\
\underline{Demostración.}\\
La demostración es  análoga a, pero más simple que la de la prop. 17 del cap. VII. \\
Sean $f \en L_p^T$ y $h \en \R$. La función es periódica de período $T$. Además $f_h$ es medible. Luego $|f_h|^p$ es medible positiva. Por el cambio de variables $x+h=y$ se obtiene:
$$
\int_{-\frac{T}{2}}^{\frac{T}{2}}|f_h|^p =\int_{-\frac{T}{2}}^{\frac{T}{2}}|f(x+h)|^p dx=\int_{-\frac{T}{2}}^{\frac{T}{2}}|f(y)|^p dy=\int_{-\frac{T}{2}}^{\frac{T}{2}}|f|^p.
$$
Así pues $f_h \en L_p^T$ y $N_p(f_h)=N_p(f)$. \\
Sea $\epsilon >0$ dado. \\
Por la prop. 16 $\exists g \en L^T \pitchfork N_p (f-g) < \epsilon$. \\
Puesto que $g$ es uniformemente continua en $\R$. \\
$\exists \delta>0$ $\pitchfork |x_1 - x_2| <\delta$ $\flecha |g(x_1)-g(x_2)|<\frac{\epsilon}{T^\frac{1}{p}}$.\\
Tomemos $s,t \en \R$ tales que $|s-t|<\delta$. Entonces
$$
N_p(g_s-g_t)=\biggl(\int_{-\frac{T}{2}}^{\frac{T}{2}} g(x+s)-g(x+t)|^p dx\biggl)^\frac{1}{p} \leq \biggl( \frac{\epsilon^p}{T}\cdot T \biggl)^\frac{1}{p}=\epsilon.
$$

Finalmente:
$$
N_p(f_s-f_t) \leq N_p(f_s - g_s) +N_p(g_t - f_t) < 3\epsilon
$$
 \\
 \phantom{sssssssssssssssssssssssssssssssssss sasdasdasdasdadadssada} c. q. d \\ \\
 La CONVOLUCION de las FUNCIONES $f,g$ PERIÓDICAS de periodo $T$ se define, si existe, como la función $f*g$ dada por:
 \begin{equation*}
 \boxed{(f*g)(x)=:\int_{-\frac{T}{2}}^{\frac{T}{2}}  f(y)g(x-y)dy.}
 \end{equation*}
 Las proposiciones a continuación que no estarán seguidas por demostraciones deberán ser probadas por lector a título de ejercicio. \\ \\
 \textbf{Proposición 18.}\\ \\
 Sean $f,g \en L_1 ^T$. Para casi todo $x\en \R$ está definida la convolucion $(f*g)(x)=\int_{-\frac{T}{2}}^{\frac{T}{2}}  f(y)g(x-y)dy$. Se tiene $f*g \en L_1^T$ y $\int_{-\frac{T}{2}}^{\frac{T}{2}}  f*g=\int_{-\frac{T}{2}}^{\frac{T}{2}} f. \int_{-\frac{T}{2}}^{\frac{T}{2}} g. $ \\
 La aplicación $(f,g) \flecha f*g$ hace de $L_1^T$ un álgebra de Banach conmutativa. En particular se tiene $N_1 (f*g) \leq N_1 (|f| * |g|)=N_1 (f) N_1 (g)$ .\\ \\
 \textbf{Proposición 19.} \\ 
 Sean $p,q \en [1,+\infty [$ tales que $\frac{1}{p}+ \frac{1}{p}>1$. Definimos $r$ por $\frac{1}{r}=:\frac{1}{p}+\frac{1}{q}-1$. Sean $f \en L_p^T$, $g \en L_q^T$. Entonces
 \begin{enumerate}[1)]
 \item Para casi todo $x \en \R$ existe la convolucion. 
 
 $$
 (f*g)(x)=\int_{-\frac{T}{2}}^{\frac{T}{2}}  f(y)g(x-y)dy
 $$
 \item $f*g=g*f \en L_r^T$.
 
 \item $N_r (f*g) \leq N_p (f) N_q (g)$.
 \end{enumerate}
 
 En particular si $f \en L_p^T$ y $g \en L_1^T$, entonces $f*g \en L_p^T$ y $N_p (f*g) \leq N_p (f) N_1 (g)$. \\ \\
 \textbf{Proposición 20.}\\ 
 Sea $1 \leq p < \infty$. Sean $f \en L_p^T, g\en L_{p*}^T$. Entonces para todo $x$ en $\R$ está definida la convolución $(f*g)(x)=\int_{-\frac{T}{2}}^{\frac{T}{2}} f(y)g(x-y)dy$. \\
 Se tiene $f*g=g*f$. Finalmente $f*g \en L^t$. \\
 \underline{Definición.}\\
 Una sucesión $\lbrace \rho_\upsilon \rbrace$ de funciones en $L_1^t$ se llama SUCESIÓN de DIRAC PERIÓDICA de PERIODO T (mientras no haya riesgo de confusión diremos simplemente "sucesión de Dirac"), si cumple las siguientes condiciones:
 
 \begin{enumerate}[i)]
 \item $\rho_\upsilon \geq 0$.
 
 \item $\int_{-\frac{T}{2}}^{\frac{T}{2}} \rho_\upsilon =1$.
 
 \item Si $0 < \delta < \frac{T}{2}, \lim_{\upsilon \to +\infty} \int_{-\delta}^\delta \rho_\upsilon =1$
 \end{enumerate}
 
 En vista de II) la condición III) puede reemplazarse equivalentemente por:
 $$
 \lim_{\upsilon \to +\infty} \biggl( \int_\delta^{\frac{T}{2}} \rho_\upsilon + \int_{-\frac{T}{2}}^{-\delta}\rho_\upsilon \biggl)=0
 $$
$\lbrace \rho_\upsilon \rbrace $ se llama SUCESIÓN de DIRAC FUERTE si $\rho_\upsilon \en L_\infty^T$ $\todo \upsilon$, se verifican las condiciones I), II) y la condición:

iii') Si $0<\delta < \frac{T}{2}$, $\lim_{\upsilon \to +\infty} \stackbin[\delta \leq |t| \leq \frac{T}{2}]{}{Sup} ess \rho_\upsilon (t)=0$
Esta condición iii') implica la condición III). \\
En efecto supongamos realizada la condición iii'). Sean $0<\delta<\frac{T}{2}$ y $M_\upsilon (\delta)=:\stackbin[\delta \leq |t| \leq \frac{T}{2}]{}{ess \rho_\upsilon (t)}= 0$. Entonces:
$$
\int_{\delta}^{\frac{T}{2}} \rho_\upsilon + \int_{-\frac{T}{2}}^\delta \rho_\upsilon \leq (T-2\delta )M_\upsilon (\delta) \stackbin[\upsilon \to +\infty]{}{\flecha} 0.
$$
\\ \\
\textbf{Proposición 21.} \\
Sea $\lbrace \rho_\upsilon \rbrace$ una sucesión de Dirac en $L_1^T$.
\begin{enumerate}[1)]
\item Si $1 \leq p < \infty$ y $f \en L_p$, se tiene
$$
\underline{\lim_{\upsilon \to +\infty}N_p (f-\rho_\upsilon * f)=0}
$$
\item Si $f \en L_\infty^T$ y $f$ es continua en un punto $x$, entonces
$$
\lim_{\upsilon \to +\infty} (\rho_\upsilon * f)(x)=f(x).
$$
\item Si $f \en L_\infty^T$ y $f$ es continua en un conjunto $J$ abierto con respecto a $[-\frac{T}{2},\frac{T}{2}], \lbrace \rho_\upsilon * f \rbrace$ converge a $f$ uniformemente en todo compacto contenido en $J$.
\item Si $f \en 	L^T$, $\lbrace \rho_\upsilon * f \rbrace$ converge a $f$ uniformemente en $\R$. En el caso de ser $\lbrace \rho_\upsilon \rbrace$ una sucesión de Dirac fuerte, las conclusiones de 2) y 3) subsisten al suponer solamente $f \en L_1^T$.
\end{enumerate}

\underline{Demostración.}\\
Como muestra probaremos solamente 3) en la hipótesis que $\lbrace \rho_\upsilon \rbrace$ es una sucesión de Dirac fuerte y $f \en L_1^T$. \\
La función $\rho_\upsilon * f$ existe y está bien determinada (y continua) en todo punto. \\

Sea $K$ un compacto contenido en $J$. Pongamos $\alpha=: d(K,[-\frac{T}{2},\frac{T}{2}]-J)$. Sea $K'= \lbrace x| d(x,K) \leq \frac{\alpha}{2} \rbrace$. $K'$ es un compacto tal que $K \subset K' J$. Sea dado $\epsilon >0$. Puesto que $f$ es uniformemente continua en $K' \exists \delta$ $\pitchfork 0<\delta < \frac{\alpha}{2}$ y $x_1, x_2 \en K'$ $\flecha |f(x_1)-f(x_2)| < \frac{\epsilon}{2}$. \\
Sea $x$ un punto arbitrario de $K$. Se tiene: 
\begin{equation}
|f(x)-(\rho_\upsilon * f)(x)| \leq \int_{-\delta}^\delta |f(x)-f(x-y)| \rho_\upsilon (y)dy+
\end{equation}
$$
\stackbin[\delta \leq |y| \leq \frac{T}{2}]{}{\int}|f(x)-f(x-y)|\rho_\upsilon (y)dy.
$$
Si $-\delta < y < \delta$, el punto $x-y$ está en $K'$. Por lo tanto, la primera integral a la derecha de (3.34) es $\leq \frac{\epsilon}{2}$. La segunda está mayorizada por $M_\upsilon (\delta) \biggl(A(T-2\delta)+ \int_{-\frac{T}{2}}^{\frac{T}{2}} |f(x-y)|dy \biggl)=M_\upsilon (\delta) \biggl( A(T-2\delta) +N_1 (f) \biggl)$, donde $M_\upsilon (\delta)=: \stackbin[\delta \leq |y| \leq \frac{T}{2}]{}{ess \phantom{s}  \rho_\upsilon (y)}$ y $A=: \stackbin[x \en K]{}{M \acute{a}x} |f(x)|$\\
Puesto que $\lbrace \rho_\upsilon \rbrace$ es una sucesión de Dirac fuerte, se tiene $\lim_{\upsilon \to +\infty }M_\upsilon (\delta)=0$. Existe pues $n_o \en \N$ tal que:
$$
\upsilon \geq n_o \flecha M_\upsilon \biggl(A(T-2\delta)+N_1 (f) \biggl) < \frac{\epsilon}{2}
$$
Por () será 
$$
\upsilon \geq n_o \flecha |f(x)-(\rho_\upsilon * f)(x)| < \epsilon \phantom{s} \todo x \en K
$$
 \\
 \phantom{sssssssssssssssssssssssssssssssssss sasdasdasdasdadadssada} c. q. d \\ \\
 La parte 2) de la prop. 21. tiene la siguiente generalización:
\\ \\
\textbf{Proposición 22.} \\ 
Sea $\lbrace \rho_\upsilon \rbrace$ una sucesión de Dirac en $L_1^T$. Se supone que para todo $\upsilon$ la función $\rho_\upsilon$ es par. \\
Sea $f \en L_\infty^T$. Se supone que en un punto $x \en \R$ existen $f(x^-)$ (el límite por la izquierda de $f$ en $x$) y $f(x^{+})$ (el límite por la derecha de $f$ en $x$). Entonces: 
$$
\lim_{\upsilon \to +\infty} (\rho_\upsilon * f)(x)=\frac{1}{2}(f(x^{-}+f(x^{+})).
$$
La conclusión subsiste si $\lbrace \rho_\upsilon \rbrace$ es una sucesión de Dirac fuerte y $f \en L_1^T$.  \\
\underline{Demostración.} \\
De nuevo haremos la demostración en la hipótesis de que $\lbrace \rho_\upsilon \rbrace$ es una sucesión de Dirac fuerte y $f \en L_1^T$. \\
Se tiene: 
\begin{equation}
(\rho_\upsilon * f)(x)=\int_{-\frac{T}{2}}^0 f(x-y)\rho_\upsilon (y)dy+\int_{0}^\frac{T}{2}f(x-y)\rho_\upsilon (y)dy.
\end{equation}
Hagamos en la primera integral el cambio de variables $z=-y$, usemos la paridad de $\rho_\upsilon$ y escribamos de nuevo y en vez de $z$. La relación (3.35) se convierte en
\begin{equation}
(\rho_\upsilon * f)(x)=\int_{0}^{\frac{T}{2}} (f(x+y)+f(x-y))\rho_\upsilon (y)dy
\end{equation}
De la paridad de $\rho_\upsilon$ sigue también:

\begin{equation}
\frac{1}{2}=\int_0^{\frac{T}{2}}\rho_\upsilon (y)dy
\end{equation}
Multipliquemos los dos miembros de (3.57) por $f(x^{+})+f(x^-)$ y restemos miembro por miembro de (3.56). Queda:

\begin{equation}
(\rho_\upsilon * f)(x)-\frac{1}{2} (f(x^+)+f(x^-))=\int_{0}^\frac{T}{2} (f(x+y)-f(x^+))\rho_\upsilon (y)dy+
\end{equation}
$$
+\int_{0}^\frac{T}{2} (f(x-y)-f(x^-))\rho_\upsilon (y)dy
$$
Basta demostrar que las dos integrales a la derecha de (3.58) tienden a cero para $\upsilon \to +\infty$. Probésmolo p. ej. para la primera integral. \\
Sea $\epsilon>0$ dado. Por definición de $f(x^+)$ existe $\delta$, tal que $0<\delta<\frac{T}{2}$ y que $0<h<\delta$ $\flecha |f(x+h)-f(x^+)|<\frac{\epsilon}{2}$.  \\
Escribamos:
\begin{equation}
|\int_0^{\frac{T}{2}}(f(x+y)-f(x^+))\rho_\upsilon (y)dy| \leq \int_0^\delta |f(x+y)-f(x^+)|\rho_\upsilon (y)+
\end{equation}
$$
+\int_{\delta}^{\frac{T}{2}}|f(x+y)-f(x^+)|\rho_\upsilon (y)dy.
$$

La primera integral a la derecha de (3.59) es $\leq \frac{\epsilon}{2}$. La segunda está mayorizada por $M_\upsilon (\delta) (|f(x^+)|(\frac{T}{2}-\delta)+N_1 (f))$ donde $M_\upsilon (\delta)=:
\stackbin[\delta \leq y \leq \frac{T}{2}]{}{Sup}ess(\rho_\upsilon)(y)$. Puesto que $\lbrace \rho_\upsilon \rbrace$ es una sucesión de Dirac fuerte $\lim_{\upsilon \to +\infty}M_\upsilon (\delta)=0$. Luego
$$
\exists n_o \en N \pitchfork \phantom{s} \upsilon \geq n_o \flecha M_\upsilon (\delta)(|f(x^+)|(\frac{T}{2}-\delta)+N_1 (f))<\frac{\epsilon}{2}
$$
Luego por (3.59):
$$
\upsilon \geq n_o \flecha |\int_{0}^{\frac{T}{2}}f(x+y)-f(x^+))\rho_\upsilon (y)dy| < \epsilon.
$$
 \\
 \phantom{sssssssssssssssssssssssssssssssssss sasdasdasdasdadadssada} c. q. d \\ \\
 Las proposiciones 23 y 24. a continuación constituyen unas aplicaciones importantes de sucesiones de Dirac periódicas. \\
 \underline{Definición.}\\
Sea $(E,||$ $||)$ un espacio vectorial normado. Sea $S$ un subconjunto de $E$ y sea $L(S)$ el subespacio de $E$ engendrado por $S$ (es decir el conjunto de todas las combinaciones lineales finitas de los elementos de $S$). Se dice que el SUBCONJUNTO S es COMPLETO en E, si $L(S)$ es denso en E.\\
A continuación se considerará funciones periódicas de periódo $2\pi$. Esto es una normalización que no causa pérdida de generalidad. (cf. observación después de la prop. 25.) \\
\underline{Definiciones.}\\
Se llama SISTEMA TRIGONOMETRICO REAL $L_\R$ el conjunto de las funciones $\R \flecha \R$ constituído por:
\begin{enumerate}[1)]
\item Todas las funciones $x\flecha cos (kx)$ para $k=0,1,2,\ldots$ (para $k=0$ la función de valor $1$).
\item Todas las funciones $x \flecha sen(kx)$ para $k=1,2,\ldots$ se llama SISTEMA TRIGONOMETRICO COMPLEJO $L_C$ al conjunto de las funciones $x \flecha e^{ikx}$, $k \en \Z$, de $\R$ en $\C$. (Aquí $i=\sqrt{-1}$). \\
Las combinaciones lineales finitas de elementos de $L_\R$ con coeficientes reales se llaman POLINOMIOS TRIGONOMETRICOS REALES. 
\end{enumerate}
Las combinaciones lineales finitas de elementos de $L_\R$ con coeficientes complejos son también aquellas de $L_C$ y viceversa. \\
Se llaman POLINOMIOS TRIGONOMETRICOS COMPLEJOS. \\ \\
\textbf{Proposición 23.} \\
\begin{enumerate}[1)]
\item Si $1 \leq p <\infty$ el sistema $L_\R$ es completo en $L_p^{2\pi}(\R)$. $L_\R$ (equivalentemente $L_C$) es completo en $L_p^{2 \pi}(\R)$. \\
\item $L_\R$ es completo en el espacio $L^{2\pi} (\R)$ provisto de la norma uniforme.
\end{enumerate}
$L_\R$ (equivalentemente $L_C$) es completo en el espacio $L^{2\pi}(\C)$ provisto de la norma uniforme.\\
\underline{Demostración.}\\
Pongamos $\todo \upsilon \en \N$:
$$
\rho_\upsilon (x)=:\frac{(1+cos(x))^\upsilon}{\int_{-\pi}^\pi (1+cos(t))^\upsilon dt}=\frac{cos^{2\upsilon}\frac{x}{2}}{\int_{-\pi}^\pi cos^{2\upsilon}\frac{t}{2}dt}
$$
Afirmamos que $\lbrace \rho_\upsilon \rbrace$ es una sucesion de Dirac fuerte de período $2\pi$. Las propiedades 1) y 2) son evidentes. Comprobemos la propiedad iii'). Sea $\delta$ tal que $0<\delta<\pi$. 
$$
\int_{-\delta}^{\delta}cos^{2\upsilon}\frac{t}{2}dt=4\int_0^\frac{\pi}{2} cos^{2\upsilon}t dt \geq 4 \int_{0}^{\frac{delta}{4}}cos^{2\upsilon}t dt \geq \delta cos^{2\upsilon}\frac{\delta}{4}.
$$
Luego $\todo x \en [\delta,\pi]$: 
$$
\rho_\upsilon (x) \leq \frac{1}{\delta}\frac{cos^{2 \upsilon}\frac{\delta}{2}}{cos^{2 \upsilon}\frac{\delta}{4}} \stackbin[\upsilon \to +\infty]{}{\flecha}0
$$
En vista de la paridad de $\rho_\upsilon$ esto demuestra iii').\\
De la proposición 21. sigue:
\begin{enumerate}[a)]
\item Si $f \en L_P^{2 \pi} (\K)$ $(1 \leq p <\infty)$, se tiene:
$$
\lim_{\upsilon \to +\infty}N_p (f-\rho_\upsilon * f)=0
$$
\item Si $f \en L^{2 \pi}(\K)$ , $\lbrace \rho_\upsilon * f \rbrace$ converge a $f$ uniformemente en \R. El teorema quedará pues probado, si mostramos que $\rho_\upsilon *f$ es un polinomio trigonométrico (real si $\K=\R$, complejo si $\K=\C$). \\
Ahora bien
\end{enumerate}
\begin{equation}
(\rho_\upsilon * f)(x)=(f*\rho_\upsilon )(x)=\frac{1}{c_\upsilon}\int_{-\pi}^\pi f(y)(1+cos(x-y))^\upsilon dy,
\end{equation}



  %


\end{document}
